%!TEX root = ../main.tex
%%%%%%%%%%%%%%%%%%%%%%%%%%%%%%%%%%
% Links:
%https://leetcode.com/problems/valid-parenthesis-string/
%https://leetcode.com/problems/valid-parenthesis-string/discuss/582134/C%2B%2B-2-Pointer-Approach-Explained-O(1)-Space-O(N)-Time
%https://leetcode.com/explore/challenge/card/30-day-leetcoding-challenge/530/week-3/3301/discuss/582074/C%20%20-Using-Two-Stacks
%https://leetcode.com/problems/valid-parenthesis-string/discuss/543521/Java-Count-Open-Parenthesis-O(n)-time-O(1)-space-Clean-Explain
%https://leetcode.com/explore/challenge/card/30-day-leetcoding-challenge/530/week-3/3301/discuss/582134/C++-2-Pointer-Approach-Explained-O(1)-Space-O(N)-Time
% Difficulty:
% Companies: 
%%%%%%%%%%%%%%%%%%%%%%%%%%%%%%%%%%

\chapter{Validate Parenthesized String}
\label{ch:valid_parenthesis}

\section{Problem statement}
\begin{exercise}
 Given a string \textit{s} containing only three types of characters:
 \begin{enumerate}
	 \item \textit{(}
	 \item \textit{)}
	 \item \textit{*}
 \end{enumerate}
 write a function to check whether a string is valid. A string is valid if the following  holds:

  \begin{itemize}
	\item  Any left parenthesis \textit{(} must have a corresponding  right parenthesis \textit{)}.
    \item Any right parenthesis \textit{)}  must have a corresponding left parenthesis \textit{)}.
    \item Left parenthesis \textit{(} must appear before the corresponding right parenthesis \textit{)}.
    \item The character \textit{*} could be treated as a jolly, and can be modified into a single right parenthesis  \textit{)} or a single left parenthesis \textit{)} or deleted.
 \end{itemize}  


	\begin{example}
		\hfill \\
		Given the input string \textit{s="(**))"} the function returns \textbf{true} because it is possible to obtain from \textit{s} the string \textit{(())} by deleting the first \textit{*} and by turning the second one into a left parenthesis \textit{)}.
	\end{example}

	\begin{example}
		\hfill \\
		Given the input string s=\textit{"*(*)()(()"} the function returns \textbf{false} because no matter how the 	\inline{*} are arranged there is no way to obtain a well balanced string of parenthesis.
	\end{example}
\end{exercise}

\section{Clarification Questions}

\begin{QandA}
	\item Is the empty string considered valid?
	\begin{answered}
		\textit{An empty string is also valid.}
	\end{answered}
	
\end{QandA}

\section{Discussion}
\label{valid_parenthesis:sec:discussion}
This is an extremely interesting and quite challenging problem that can be solved in different ways. We will start by having a look at the brute-force solution in Section \ref{valid_parenthesis:sec:bruteforce} will will allow us to develop a dynamic programming solution that works much better. Section \ref{valid_parenthesis:sec:linear} presents a totally different solution that improves dramatically the time and space complexity compared to the previous approaches. Section \ref{valid_parenthesis:sec:twostacks} presents a linear time and space clever solution based on stacks.

We will highly advice to use the solution shown in Section \ref{valid_parenthesis:sec:linear} as a reference and use that during an actual interview.

\subsection{Brute-force}
\label{valid_parenthesis:sec:bruteforce}
If the input string does not contains  wild-cards, this problem is quite trivial and becomes easily solvable by using a stack. When wild-cards are present things can get more complicated because now for each of them there are three options. In the brute-force approach we will try all possible options for all wild-cards. 
The idea is that the input string $s$ is traversed from left to right. As we traverse the string we will keep track of how many open \inline{open} and closed \inline{closed} parenthesis we have encountered. We do this because if at any moment we find that the number of closed parenthesis is greater than the number of open ones, the string is invalid (it violates the constraint that any left parenthesis should appear before any right one). Depending on the character $c$ we are processing:
\begin{enumerate}
	\item If $c$is a \textit{(} then we increase the number of open parenthesis \inline{open++}found so far and we recursively check the rest of the string. 
	\item Similarly, if $c$ is a \textit{)} then we increase the number of closed parenthesis and proceed checking the rest of the string. 
	\item If the current character is a \textit{*} then we have the option to:
	\begin{itemize}
		\item consider it as an open parenthesis
		\item  consider it as a closed parenthesis
		\item ignore it
	\end{itemize}
\end{enumerate}

The recursion terminates when either:
 \begin{itemize}
	\item the number of closed parenthesis is larger than the number of open ones
	\item  we have processed the whole string. In this case we return true only if the number of open parenthesis so far is equal to the closed ones (necessary condition for a well balanced string).
\end{itemize}


Listing \ref{list:valid_parenthesis:bruteforce} shows a possible recursive implementation of the idea above. The complexity of this approach is exponential in the number of \textit{*}, i.e. $O(3^{n})$, where $n$ is the length of $s$.

\lstinputlisting[language=c++, caption={Brute-force, exponential time solution to the problem of validating a string of parenthesis with wild-cards.},label=list:valid_parenthesis:bruteforce]{sources/valid_parenthesis/valid_parenthesis_solution1.cpp}

\subsection{Dynamic Programming}
\label{valid_parenthesis:sec:dp}
Another way of solving this problem is to still try all possibilities like in the brute-force solution, but doing it in a smart way, making sure no work is done more than once. In a string of length $n$ there are $O(n^2)$ possible substring. 
Given a substring starting at $i$ and ending at $j$, from now on identified by $s(i,j)$ we can solve this problem by processing one character $c$ at the time. A substring $s(i,j)$ is valid when
\begin{itemize}
	\item if $c$ is an \textit{*} and $s(i+1,j)$ is valid. We try to ignore the character $c$.
	\item if $c$ is either \textit{*} or \textit{(}, then we search for a character $k$ in $s(i+1,k)$ s.t. it can be turn into a closing parenthesis. If $k$ exists (in case multiple $k$ exists, then we try all of them) then, $s(i,j)$ is valid if $s(i+1,k-1)$ and $s(k+1,j)$ are valid. Basically what we do here is to match an open parenthesis with a closing one that appears further in the range. Remember that each open parenthesis must be paired up with a closing one.
	\item if $c$ \textit{)} we return false because we have in this case an unmatched closing parenthesis.
\end{itemize}


\lstinputlisting[language=c++, caption={Sample Caption},label=list:valid_parenthesis]{sources/valid_parenthesis/valid_parenthesis_solution2.cpp}
\subsection{Linear time}
\label{valid_parenthesis:sec:linear}

\lstinputlisting[language=c++, caption={Sample Caption},label=list:valid_parenthesis]{sources/valid_parenthesis/valid_parenthesis_solution3.cpp}

\subsection{Stack based solution}
\label{valid_parenthesis:sec:twostacks}

\lstinputlisting[language=c++, caption={Sample Caption},label=list:valid_parenthesis]{sources/valid_parenthesis/valid_parenthesis_solution4.cpp}




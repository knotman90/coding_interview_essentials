%!TEX root = ../main.tex
%%%%%%%%%%%%%%%%%%%%%%%%%%%%%%%%%%
% Links:
%
% Difficulty:
% Companies: 
%%%%%%%%%%%%%%%%%%%%%%%%%%%%%%%%%%

\chapter{Validate Parenthesised String}
\label{ch:valid_parenthesis}

\section{Problem statement}
\begin{exercise}
 Given a string $s$ containing only three types of characters:
 \begin{enumerate}
	 \item \textit{(}
	 \item \textit{)}
	 \item \textit{*}
 \end{enumerate}
 write a function to check whether a string is valid. A string is valid if the following  holds:

  \begin{itemize}
	\item  Any left parenthesis \textit{(} must have a corresponding  right parenthesis \textit{)}.
    \item Any right parenthesis \textit{)}  must have a corresponding left parenthesis \textit{)}.
    \item Left parenthesis \textit{(} must appear before the corresponding right parenthesis \textit{)}.
    \item The character \textit{*} could be treated as a jolly, and can be modified into a single right parenthesis  \textit{)} or a single left parenthesis \textit{)} or deleted.
 \end{itemize}  

\end{exercise}


\begin{example}
	\hfill \
	
\end{example}

\begin{example}
	\hfill \
	
\end{example}

\section{Clarification Questions}

\begin{QandA}
	\item Is the empty string considered valid?
	\begin{answered}
		\textit{An empty string is also valid.}
	\end{answered}
	
\end{QandA}

\section{Discussion}
\label{valid_parenthesis:sec:discussion}


\subsection{Brute-force}
\label{valid_parenthesis:sec:bruteforce}

\lstinputlisting[language=c++, caption={Sample Caption},label=list:valid_parenthesis]{sources/valid_parenthesis/valid_parenthesis_solution1.cpp}


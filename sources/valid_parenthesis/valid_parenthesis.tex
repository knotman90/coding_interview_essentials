%!TEX root = ../main.tex
%%%%%%%%%%%%%%%%%%%%%%%%%%%%%%%%%%
% Links:
https://leetcode.com/problems/valid-parenthesis-string/
https://leetcode.com/problems/valid-parenthesis-string/discuss/582134/C%2B%2B-2-Pointer-Approach-Explained-O(1)-Space-O(N)-Time
https://leetcode.com/explore/challenge/card/30-day-leetcoding-challenge/530/week-3/3301/discuss/582074/C%20%20-Using-Two-Stacks
https://leetcode.com/problems/valid-parenthesis-string/discuss/543521/Java-Count-Open-Parenthesis-O(n)-time-O(1)-space-Clean-Explain
https://leetcode.com/explore/challenge/card/30-day-leetcoding-challenge/530/week-3/3301/discuss/582134/C++-2-Pointer-Approach-Explained-O(1)-Space-O(N)-Time
% Difficulty:
% Companies: 
%%%%%%%%%%%%%%%%%%%%%%%%%%%%%%%%%%

\chapter{Validate Parenthesized String}
\label{ch:valid_parenthesis}

\section{Problem statement}
\begin{exercise}
 Given a string \textit{s} containing only three types of characters:
 \begin{enumerate}
	 \item \textit{(}
	 \item \textit{)}
	 \item \textit{*}
 \end{enumerate}
 write a function to check whether a string is valid. A string is valid if the following  holds:

  \begin{itemize}
	\item  Any left parenthesis \textit{(} must have a corresponding  right parenthesis \textit{)}.
    \item Any right parenthesis \textit{)}  must have a corresponding left parenthesis \textit{)}.
    \item Left parenthesis \textit{(} must appear before the corresponding right parenthesis \textit{)}.
    \item The character \textit{*} could be treated as a jolly, and can be modified into a single right parenthesis  \textit{)} or a single left parenthesis \textit{)} or deleted.
 \end{itemize}  


	\begin{example}
		\hfill \\
		Given the input string \textit{s="(**))"} the function returns \textbf{true} because it is possible to obtain from \textit{s} the string \textit{(())} by deleting the first \textit{*} and by turning the second one into a left parenthesis \textit{)}.
	\end{example}

	\begin{example}
		\hfill \\
		Given the input string s=\textit{"*(*)()(()"} the function returns \textbf{false} because no matter how the 	\inline{*} are arranged there is no way to obtain a well balanced string of parenthesis.
	\end{example}
\end{exercise}

\section{Clarification Questions}

\begin{QandA}
	\item Is the empty string considered valid?
	\begin{answered}
		\textit{An empty string is also valid.}
	\end{answered}
	
\end{QandA}

\section{Discussion}
\label{valid_parenthesis:sec:discussion}


\subsection{Brute-force}
\label{valid_parenthesis:sec:bruteforce}

\subsection{Dynamic Programming}
\label{valid_parenthesis:sec:twostacks}


\subsection{Stack based solution}
\label{valid_parenthesis:sec:twostacks}

\lstinputlisting[language=c++, caption={Sample Caption},label=list:valid_parenthesis]{sources/valid_parenthesis/valid_parenthesis_solution1.cpp}


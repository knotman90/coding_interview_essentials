%!TEX root = ../main.tex
%%%%%%%%%%%%%%%%%%%%%%%%%%%%%%%%%%
% Links:
%
% Difficulty: Companies: 
%%%%%%%%%%%%%%%%%%%%%%%%%%%%%%%%%%

\chapter{First positive missing}
\label{ch:first_positive_missing}
\section*{Introduction}

This chapter addresses a fairly common problem posed during on-site interviews for which there are a number of solutions which vary widely in terms of time and space complexity.

Finding what most interviewers would consider the \quotes{best} solution in terms of asymptotic complexity can be challenging therefore in needs a more in depth analysis than some other problems posed in this book.

It is common for interviewers to pose this problem using a short and purposely vague statement.
It is, therefore,  important to ask questions to ensure all aspects of the problem are well understood before attemping a solution.  \footnote{ We think a good way of doing this is to repeat out loud a summary of your understanding of the problem to the interviewer.}

\section{Problem statement}
\begin{exercise}
	Write a function that, given an unsorted integer array $A$, returns the smallest positive integer not contained in $A$.

	\begin{example}
		\hfill \\
		Given $A=\{ 1, 0, -1, -2\}$ the answer is $2$.
	\end{example}
	
	\begin{example}
		\hfill \\
		Given $A=\{ 2, 3, -7, 6, 8, 1, -10, 15\}$ the answer is $4$.
	\end{example}
	
	\begin{example}
		\hfill \\
		Given $A=\{ 1, 0, -1, -2\}$ the answer is $2$.
	\end{example}
\end{exercise}
	
\section{Clarification Questions}

\begin{QandA}
	\begin{questionitem} \begin{question} Are the input numbers always positive?  \end{question} 	 
    \begin{answered}
		\textit{No, the array contains positive and negative numbers.}
	\end{answered} \end{questionitem}

	\begin{questionitem} \begin{question} Are all the elements distinct?  \end{question} 	 
    \begin{answered}
		\textit{No, the array might contains duplicates.}
	\end{answered} \end{questionitem}
	
	\begin{questionitem} \begin{question} Can the input array be modified?  \end{question} 	 
    \begin{answered}
		\textit{Yes.}
	\end{answered} \end{questionitem}

	\begin{questionitem} \begin{question} Can the size of the array be zero? In other words, can the array be empty?  \end{question} 	 
    \begin{answered}
		\textit{No, the input array contains at least one element.}
	\end{answered} \end{questionitem}

	\begin{questionitem} \begin{question} Is $0$ a valid output?  \end{question} 	 
    \begin{answered}
		\textit{No, only strictly positive numbers should be returned.}
	\end{answered} \end{questionitem}

\end{QandA}

\section{Discussion}
\label{first_positive_missing:sec:discussion}
This problem has a solid real-life application.
Consider how an OS might assign PID\footnote{A number used (in UNIX part of the process control block) to uniquely identify a process within the OS. In Unix, process IDs are usually allocated on a sequential basis, beginning at 0 and rising to a maximum value (usually 65535) which varies from system to system. Once this limit is reached, allocation restarts at zero and again increases. However, for this and subsequent passes any PIDs still assigned to processes are skipped.}s to processes. One approach would be to keep a list of all the PIDs for all the processes running, and once a new one is fired up, the OS will assign it the smallest PID \textbf{not already assigned} to any other process. 

In a highly dynamic environment like the OS with thousands of applications active at the same time; the focus of the solution should be speed as you want the process to be up and running as quickly as possible. 


\subsection{Brute-force}
One of the simplest approaches is to simply search $A$ for the missing number incrementally,  starting from $1$. 
The practical reality of this approach is that we have to perform a search operation in $A$ for each number from $1$ onward \textbf{until the search fails}.
This algorithm is always guaranteed to return the smallest missing number given that we perform the searches in order, with the smallest numbers being searched first. 

Listing \ref{list:first_positive_missing_bruteforce} shows an implementation of this using \inline{std::find()} as a means to do the actual search in the $A$. 
 
\lstinputlisting[language=c++, caption=Two bruteforce solution implementations the problem of finding the smallest missing positive integer in an array.,label=list:first_positive_missing_bruteforce]{sources/first_positive_missing/first_positive_missing_solution4.cpp}

This is often considered a poor solution (as a rule of thumb, in the context of coding interviews, brute-force solutions always are) as it has a
complexity of $O(n^2)$ time and $O(1)$ space.

It does, however, have some advantage in being  easy and fast to write, and avoiding implementations mistakes due to simple logic and small amount of code involved.


\subsection{Sorting}
\label{first_positive_missing:sec:sorting}

The second most intuitive approach (after brute-force)is sorting the input as having the numbers sorted is helpful for easily coming up with a faster solution.

When the $A$ is sorted, we know the positive numbers in it will all be appearing \textbf{in an ordered fashion} from a certain index $k\geq 0$ onwards (the positions from index $0$ to $k-1$ are occupied by negatives or zeros). 

We also know that, if no number is missing in $A[k \ldots n-1]$, then we would expect to see: 
\begin{itemize}
	\item $A[k]=1$
	\item $A[k+1]=2$
	\item $A[k+2]=3$
	\item $\ldots$
	\item $A[n-1]=n-k+1$
\end{itemize}

i.e. all numbers from $1$ onward appear in their natural order $(1,2,3, \ldots (n-k+1))$ from the cell at index $k$ to the end of $A$. 
If any of these numbers are missing then we would not be able to see such a sequence.

The goal of this problem is to find the first number that is missing from that sequence.  We can do that by finding the first element among $A[k \ldots n-1]$  where the condition $A[k+i]=i+1$ with $(i=0,1,2, \ldots)$ is \textbf{false}. When this happens, we can conclude the missing number is $i+1$.
If such a cell does not exist (every cell satisfies the condition above), then we know that the missing number is $A[n-1]+1$.

For example, consider the array $A=\{ 9,-7,0,4,5,2,0,1\}$. When sorted, the array
becomes $A=\{ -7,0,0,1,2,4,5,9\}$. The positives start at index $k=3$:
\begin{itemize}
	\item $A[3+0] = 1$ (test passes)
	\item $A[3+1] = 2$ (test passes)
	\item $A[3+2] = 4$ (\textbf{test fails})
\end{itemize}
As we can see, the test fails after three tries and therefore we can conclude the missing number is $3$.


Now let's consider the array $B=\{ 3,-7,0,4,5,2,0,1\}$ which is exactly the same as in the previous example, with the exception that we have swapped a $9$ for a $3$. When sorted, the array
becomes $B=\{ -7,0,0,1,2,3,4,5\}$ which contains no gaps between any of the positive numbers. As before, the positives start at index $k=3$ but this time every element passes the test:
\begin{itemize}
	\item $A[3+0] = 1$ (test passes)
	\item $A[4+1] = 2$ (test passes)
	\item $A[4+2] = 3$ (test passes)
	\item $A[4+3] = 4$ (test passes)
	\item $A[4+4] = 5$ (test passes)
\end{itemize}
We can clearly see that the missing number is $6 = A[8]+1 = A[n-1]+1$.

An implementation of this idea is shown in Listing
\ref{list:first_positive_missing_sorting}.

\lstinputlisting[language=c++, caption=Solution to the problem of finding the smallest missing positive integer in an array.,label=list:first_positive_missing_sorting]{sources/first_positive_missing/first_positive_missing_solution1.cpp}

Note that:
\begin{itemize}
	\item the iterator \inline{it} always points at the currently evaluated element. 
	It is initialized to either:
	\begin{itemize}
		\item the \textbf{smallest positive} in the sorted array;
		\item to one element past the end of the array if no positives are present.
	\end{itemize}
	\inline{it} is moved to its initial location by using the \inline{std::find_if} function from the STL which runs in linear time. We might have used binary search to perform this task, but that would not have signficantly helped in  lowering the overall asymptotic time complexity as the sorting operation itself costs $O(nlog(n))$ and the subsequent \inline{while} loop runs in linear time. 
	\item \inline{expected} is a variable holding the value that is expected to be found where \inline{it} is pointing to (the value $i+1$ mentioned above).
	\item if the \inline{while} runs to completion because we have examined every element of $A$ (\inline{it ==std::end(A)}) then \inline{expected} points to \inline{A.front()+1}.
	\item if no positives are present, then the \inline{while} does not even run once and $1$ is returned.
\end{itemize}

This is considered a good solution with an overall time and space complexity of $O(nlog(n))$ and $O(n)$ respectively. It is, however, not optimal, solutions with better time and space complexities exist.  


\subsection{Linear time and space solution}
\label{first_positive_missing:sec:linear_space}

Examining the problem more closely  we immediately notice that the missing number will always be
in the range $[1,n]$, where $n$ is the size of the input array. \textbf{Why is this the case?}
We can draw this conclusion by considering which input can possibly lead to the largest possible output: among
all possible arrays of size $n$, only one configuration leads to the highest missing number: $A =
\{1,2,3,4, \ldots ,n\}$ i.e. the configuration where all numbers from $1$ to $n$ are present. All the other configurations contain duplicates,
negative or numbers higher than $n$ which forces the input to have ``holes'' (i.e. missing numbers in the range $[1,n]$). 

This fact can be exploited to keep track of which positive numbers from
$1$ to $n$ are present in $A$. We can use an array of booleans flags of size $n$ to store this information. Therefore, all we have to do is set the $x^{th}$ flag to true for each number  $x$ in $A$ in the range $[1,n]$.
Eventually, this array of flags contains the answer, which can be found by scanning through it linearly to find the first \textbf{false} element which signals that this is the first missing element.

An implementation of the idea above is found in Listing \ref{list:first_positive_missing_constant_space}.

\lstinputlisting[language=c++, caption=Solution to the problem of finding the smallest missing positive integer in an array.,label=list:first_positive_missing_constant_space]{sources/first_positive_missing/first_positive_missing_solution2.cpp}


The code in Listing \ref{list:first_positive_missing_linear_space} works in two phases:
\begin{enumerate}
	\item For each number \inline{x} of $A$ in the range $[1,n]$, we remember we have found it by marking the cell at index \inline{x-1} in \inline{F}.
	\item We find if exists the first cell in \inline{F} containing false which means the corresponding element was not found in $A$. If a false cell does not exist then we can conclude the missing number is $n+1$.
\end{enumerate}
This approach is considered good and it has time and space complexity both equal to $O(n)$.

As an alternative, instead of an array, we can use a hashmap to keep track of which element has
already been seen as the array is scanned from left to right. This might have advantages in some cases, especially when $n$ is large but there are many duplicates in $A$. 


\subsection{Linear time and constant space solution}
\label{first_positive_missing:sec:constant_space}

As mentioned above, the optimal solution does not use any additional space but shares the idea of keeping track of which
element has been found with the solution described in Section \ref{first_positive_missing:sec:linear_space}. 

In order to avoid using additional space we have to somehow implement the functionality the array \inline{F} (in the code above) provides using the input array itself. Doing so sounds harder than it is, especially before we realize that we can store the information about a positive number being pres

The previous solutions have already demonstrated that we can safely ignore every negative number in $A$, and that the largest output we can ever hope to get is always less than the number of positives in $A$. For instance, if we have an array of size $n$ with $x$ negatives and $y$ positives ($n=x+y$), then the largest possible output we can get is: $y+1$. 

As an example, let's consider the array $A=\{-1, -2, -3, 0, 1, 2, 3\}$ which has $x=4$ negatives and $y=3$ positives ($n=7$). We can see that the missing number is $4=y+1$ and also that, if we substitute any positive (or negative or zeros for that matter) number with a different positive, the output of the function will \textbf{not increase}. 

The idea is to loop through $A$ and, for each number $x>0$ , store the information about $x$ being present in $A$ in some cell of $A$ itself. But which one? What we want is to have a mapping from the positives in $A$ to indices of $A$. We can use this mapping to choose the cell in which we remember whether a number is present or not by changing its sign.

Noe that, if $A$ contains $y$ positive, then we have $y$ cells to which we can change the sign from positive to negative. In our quest to create this 1-to-1 mapping one problem, the problem we face is that $A$ is unsorted and positives and negatives are all shuffled together. Therefore, the first step would be to rearrange the elements so that all the positives appear before all the negatives. In this way creating the mapping becomes much easier. If all $y$ positive are located from index $0$ to $y-1$, then every time we process an element of $x$ of $A$, which value is between $1$ and $y$ ($1 \leq x \leq y$), we can change the value of the cell at index $x$ to remember the fact the value $x$ is present in $A$. The remaining problem at this point is to rearrange $A$ so that all positives appear before all negatives and zeros in an efficient manner.


Given an unsorted array, we can rearrange its elements so that all positives are before all negatives by using a two-pointer technique where we use  - unsurprisingly -  two pointers $s$ and $e$ pointing initially to index $0$ and $n-1$, respectively.
The idea is to keep moving $s$ towards the end of the array and $e$ towards the start, until $s$ points to a positive and $e$ points to a negative. At this point, $s$ points to a value that should appear after the element pointed by $e$ and therefore we can swap those values. 
If we keep repeating this process until $s>=e$, eventually all the pairs of misplaced elements (a negative appearing before a positive) are swapped and the final array is arranged so that all negatives appear in the first $x$ positions of the array.

Consider for instance the array $A=\{-1, 1, 2,-2 0,3,-3\}$. Initially $s=0$ and $e=6$. We first move $s$ to the first positive which happens to appear at index $1$. Similarly, we move $e$ to the left towards the first negative which appears at index $6$. The elements pointed by $s$ and $e$ are swapped, and $A=\{-1, -3, 2,-2 0,3,1\}$.

The same process is repeated and after they are moved, $s=2$ and $e=4$. 
The values they point to are swapped leaving $A=\{-1, -3, 0,-2 2,3,1\}$. When the pointers are next moved, they would cross, and this signals the rearrangement is finally complete. It is important to note that the aim of this process is not to sort the array, but to simply make all the negatives appear before all the positives (therefore still allowing the positives and the negatives to appear in any order).


Once all the numbers in $A$ are processed this way, similar to what we did in step $2$ of the solution running in linear time and space (where we used the array `F`), we can scan the portion of $A$ from index $0$ to $y$ looking for positives. 
If we find one at index $k$, it means that the element $k+1$ is missing from the array as, if it was present,  it would have undergone a sign change. 
If all those cells contain negative, then it means that the missing number is $n+1$.

This idea is implemented in Listing \ref{list:first_positive_missing_linear_space} shown below.

\lstinputlisting[language=c++, caption=Linear time and constantspace solution to the problem of finding the smallest missing positive integer in an array.,label=list:first_positive_missing_linear_space]{sources/first_positive_missing/first_positive_missing_solution3.cpp}

The two phases of this solution are packaged into two functions:
\begin{itemize}
	\item \inline{first_positive_missing_constant_space} builds on it and uses the rearrangement to mark the presence of any element $x$ in the range $[1,y]$ by changing the sign of the cell at index $x-1$. When it is done with it, it proceeds in finding the answer by searching for the smallest index in $i$ containing a positive.
	\item \inline{divide_pos_neg} is responsible for rearranging the input array as discussed above
\end{itemize}

The complexity of this approach is linear in time and constant in space which is optimal.
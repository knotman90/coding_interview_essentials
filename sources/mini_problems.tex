\chapter{Mini Problems}

\section{Greatest Common Divisor}
\begin{exercise}
    Write a function to calculate the GCD of two integers.
    \begin{example}
        \label{ex:gcd:example1}
        \hfill \\
        Given $35$ and $28$ the function returns $7$.
    \end{example}
    
    \begin{example}
        \label{ex:gcd:example1}
        \hfill \\
        Given $15$ and $8$ the function returns $1$.
    \end{example}

    \end{exercise}

\subsection{\CC Brute-force}
The GCD of two numbers $x$ and $y$ is defined as the largest integer that divides both $x$ and $y$. A simple and inefficient solution would simply loop over all numbers from the smallest between $x$ and $y$ and would stop as soon as we find one that divides both. We are guaranteed to find such a nuber as the number $1$ will happily divide any number. This solution is shown in Listing \ref{list:gcd_bruteforce}.

\lstinputlisting[language=c++, caption={Brute-force, linear time solution.},label=list:gcd_bruteforce]{test/mini_problems/gcd/gcd_bruteforce.cpp}

\subsection{Log-time solution. Euclidean Algorithm}
A much faster solution can be achieved by using the Euclidean algorithm for the GCD.
This algorithm is based on the principle that the greatest common divisor of two numbers $x$ and $y$ does not change if the larger number is replaced by the reminder of the integral division between $x$ and $y$.

For example, $21$ is the GCD of $252$ and $105$ (as $252 = 21 \times 12$ and $105 = 21 \times 5$), and the same number $21$ is also the GCD of $105$ and $252 \mod 105 = 42$.
Since this replacement reduces the larger of the two numbers, repeating this process gives successively smaller pairs of numbers until we reach a point where the smallest numbers divides the largest and therefore the reminder is zero (in the worst case the smallest becomes $1$). 

It was proven by Gabriel Lamé in 1844 that this algorithm always terminates in less steps than five times the number of digits of the smaller number (in base 10) making this algorithm extremely efficient as the number of digits grows logaritmically compared the number it represents.

Listing \ref{} shows a recursive implementation of this algorithm while Listing \ref{} shows an iterative one. 

\lstinputlisting[language=c++, caption={Brute-force, linear time solution.},label=list:gcd_euclide]{test/mini_problems/gcd/gcd_euclide.cpp}

\lstinputlisting[language=c++, caption={Brute-force, linear time solution.},label=list:gcd_euclide]{test/mini_problems/gcd/gcd_euclide_iterative.cpp}

\subsection{\CC Compile-time}


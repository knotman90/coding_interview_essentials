\chapter{Mini Problems}

\section{Greatest Common Divisor}
\begin{exercise}
    Write a function to calculate the GCD of two integers.
    \begin{example}
        \label{ex:gcd:example1}
        \hfill \\
        Given $35$ and $28$ the function returns $7$.
    \end{example}
    
    \begin{example}
        \label{ex:gcd:example1}
        \hfill \\
        Given $15$ and $8$ the function returns $1$.
    \end{example}

    \end{exercise}

\subsection{\CC Brute-force}
The GCD of two numbers $x$ and $y$ is defined as the largest integer that divides both $x$ and $y$. A simple and inefficient solution would simply loop over all numbers from the smallest between $x$ and $y$ and would stop as soon as we find one that divides both. We are guaranteed to find such a nuber as the number $1$ will happily divide any number. This solution is shown in Listing \ref{}.

\lstinputlisting[language=c++, caption={Brute-force, linear time solution.},label=list:mirror_binary_tree1]{test/mini_problems/gcd/gcd_bruteforce.cpp}

\subsection{Log-time solution. Euclidean Algorithm}

\subsection{\CC Compile-time}


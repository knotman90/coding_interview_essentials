\chapter{Mini Problems}

\section{Greatest Common Divisor}
\begin{exercise}
    Write a function to calculate the GCD of two integers.
    \begin{example}
        \label{ex:gcd:example1}
        \hfill \\
        Given $35$ and $28$ the function returns $7$.
    \end{example}
    
    \begin{example}
        \label{ex:gcd:example1}
        \hfill \\
        Given $15$ and $8$ the function returns $1$.
    \end{example}

    \end{exercise}

\subsection{\CC Brute-force}
The GCD of two numbers $x$ and $y$ is defined as the largest integer that divides both $x$ and $y$. A simple and inefficient solution would simply loop over all numbers from the smallest between $x$ and $y$ and would stop as soon as we find one that divides both. We are guaranteed to find such a nuber as the number $1$ will happily divide any number. This solution is shown in Listing \ref{list:gcd_bruteforce}.

\lstinputlisting[language=c++, caption={Brute-force, linear time solution.},label=list:gcd_bruteforce]{test/mini_problems/gcd/gcd_bruteforce.cpp}

\subsection{Log-time solution. Euclidean Algorithm}
A much faster solution can be achieved by using the Euclidean algorithm for the GCD.
This algorithm is based on the principle that the greatest common divisor of two numbers $x$ and $y$ does not change if the larger number is replaced by the reminder of the integral division between $x$ and $y$.

For example, $21$ is the GCD of $252$ and $105$ (as $252 = 21 \times 12$ and $105 = 21 \times 5$), and the same number $21$ is also the GCD of $105$ and $252 \mod 105 = 42$.
Since this replacement reduces the larger of the two numbers, repeating this process gives successively smaller pairs of numbers until we reach a point where the smallest numbers divides the largest and therefore the reminder is zero (in the worst case the smallest becomes $1$). 

It was proven by Gabriel Lamé in 1844 that this algorithm always terminates in less steps than five times the number of digits of the smaller number (in base 10) making this algorithm extremely efficient as the number of digits grows logaritmically compared the number it represents.

Listing \ref{list:gcd_euclide} shows a recursive implementation of this algorithm while Listing \ref{list:gcd_euclide_iterative} shows an iterative one. 

\lstinputlisting[language=c++, caption={Euclide algorithm, recursive implementation.},label=list:gcd_euclide]{test/mini_problems/gcd/gcd_euclide.cpp}

\lstinputlisting[language=c++, caption={Euclide algorithm, iterative implementation.},label=list:gcd_euclide_iterative]{test/mini_problems/gcd/gcd_euclide_iterative.cpp}

\subsection{\CC Compile-time}
It is quite common to see specific requirements on the compile implementation of the GCD algorithm. Therefore in this section we will see how we can calculate the GCD in \CC at compile time. 

Before C++-11 the only way to do compile time computation was by using templates. In order to calculate GCD using templates we will use a structure with two integral template parameters that we will manipulate in a \inline{static const} variable. An implementation of this idea is shown in \ref{list:gcd_euclide_precpp11}.

\lstinputlisting[language=c++, caption={Pre C++11, compile-time template based solution.},label=list:gcd_euclide_precpp11]{test/mini_problems/gcd/gcd_euclide_pre_cpp11.cpp}

The code works by have a template class with two integral template parameters and a partial specialization that is used to terminate the recursion which is triggered whenever we request the static field \inline{::gcd}.

C++-11 introduces \inline{constexpr} function that can be used to specify function that can be run at compile-time. In C++-11 there are quite some limitation in what statements and operations we can do in a \inline{constexpr} context: for instance we can only have one return statement. Most of these contraints are related in the subsequent versions of the standard. A \inline{constexpr} recursive solution that works in C++-11 is shown in Listing \ref{list:gcd_euclide_cpp11}.

\lstinputlisting[language=c++, caption={C++11 \inline{constexpr} based solution.},label=list:gcd_euclide_cpp11]{test/mini_problems/gcd/gcd_euclide_cpp11.cpp}

Notice that, from C++-14 we can decorate Listing \ref{list:gcd_euclide_iterative} with \inline{constexpr} so that it can be used in compile-time computation.



%%%%%%%%%%%%%%%%%%%%%%%%%%%%%%%%%%%%%%%%%%%%%%%%%%%%%%%%%%%

\section{Maximum Depth of N-ary Tree}
\begin{exercise}
   Given a N-ary tree, return its depth which is defined as the length of the longest path from the tree's root to any of its leaves.
    \begin{example}
        \label{ex:nary-treedepth:example1}
        \hfill \\
        Given the tree depicted in Figure \ref{} the function returns $xxx$.
    \end{example}
    
    \begin{example}
        \label{ex:nary-treedepth:example1}
        \hfill \\
        Given the tree depicted in Figure \ref{} the function returns $xxx$.
    \end{example}
    \end{exercise}
 
 
    
    \subsection{Discussion}
    This is a classical problem, mostly asked during phone screening due to its simplicity. 
    What the problem, in other words, is asking us to do, is to return the maximum level of any of the tree's nodes.
    The level of a node is the number of its ancestors plus one. 
    Therefore to solve this problem, all we have to do is to visit the tree and keep track of the number of ancestors which is equivalent to the number of steps down the tree we took.
    We know that the root has $0$ ancestors and therefore, we know that each of its children will have $1$ ancestor (the root itself) and that any of its grandchildren will have $2$ ancestors and so on.
    
    Visiting a tree can be equally easily done recursively and iteratively as shown in Sections \ref{} and \ref{}, respectively.
 
    For the reminder of the discussion, we will define the root of a tree to be a pointer to the \inline{Node} structure defined in Listing \ref{list:n-arytreedepth:nodedef}.
 
    \begin{lstlisting}[language=c++, caption={Node definition.},label=list:n-arytreedepth:nodedef]]{
class Node {
public:
    int val;
    vector<Node*> children;

    Node() {}

    Node(int _val) {
        val = _val;
    }

    Node(int _val, vector<Node*> _children) {
        val = _val;
        children = _children;
    }
};            
    \end{lstlisting} 
    

\subsection{Recursive solution}
The recursive solution has the advantage of being shorter in terms of lines of code and in our opinion more elegant. It could, however, underperform the iterative solution if the compiler is not able to optimize the code properly or lead to out of memory errors if the tree depth goes over a certain value because at any given time, we keep in the stack a number of activation records that is equal to the level of the node we are visiting.

Listing \ref{list:n-ary-treedepth_recursive}  shows an implementation of this approach and has a linear time and space (considering the space occupied by the activation records) in the number of nodes of the tree.

\lstinputlisting[language=c++, caption={Recursive solution.},label=list:n-ary-treedepth_recursive]{test/mini_problems/n-ary-treedepth/n-ary-treedepth_recursive.cpp}

\subsection{Iterative Solution}
The same idea can be implemented iteratively as shown in Listing \ref{list:n-ary-treedepth_iterative} where we use a stack to keep track of the nodes \textbf{and their level}. Whenever we visit a node, we compare its level with the maximum found so far, then we remove it from the stack and insert all of its children in the stack with a level value increased by one. 

\lstinputlisting[language=c++, caption={Iterative solution.},label=list:n-ary-treedepth_iterative]{test/mini_problems/n-ary-treedepth/n-ary-treedepth_iterative.cpp}






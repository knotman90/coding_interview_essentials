%!TEX root = ../main.tex
%%%%%%%%%%%%%%%%%%%%%%%%%%%%%%%%%%
% Links:
%
% Difficulty:
% Companies: 
%%%%%%%%%%%%%%%%%%%%%%%%%%%%%%%%%%

\chapter{Find the odd occurring element}
\label{ch:find_odd_occurring_element}
\section*{Introduction}
This chapter describes a problem on array and the XOR\footnote{$\oplus$ is a boolean operator that outputs true only when its inputs differ i.e. one is true, the other is false.} (also called disjuntive-or, $\oplus$) operation that has a very easy yet inefficient  brute-force solution and a faster one that is conceptually very different from the brute-force approach. For this reason it can be challenging to start from the brute-force and then iteratively improve the solution so that it reaches the optimal time and space complexity. It is therefore better to approach this problem by reading the problem statement a carefully and to reason about the property of XOR. 

\section{Problem statement}
\begin{exercise}
Write a function that given an array $A$ of positive integers where all elements except one appear an even number of times, returns the (only) element appearing an odd numbers of time.
\end{exercise}


\begin{example}
	\hfill \\
Given the array $A=\{4,3,6,2,4,2,3,4,3,3,6\}$ the function returns $4$ because it appears $3$ times while all the other elements appear an even number of times.

\end{example}


\section{Clarification Questions}

\begin{QandA}
	\item Is the input array always valid. Does it always contain only one element appearing an odd number of times?
	\begin{answered}
		\textit{Yes the input array can be assumed to be valid.}
	\end{answered}
	\item  Is the range of the input integers known?\footnote{ This is very good question because in case the answer is yes one could use an approach similar to the counting sort to keep track of the number of times an element appears.}
	\begin{answered}
		\textit{No it is not. All integers that are representable can be present in the input array.}
	\end{answered}
	
\end{QandA}

\section{Discussion}

\subsection{Brute-force}
\label{find_odd_occurring_element:sec:bruteforce}
The brute-force solution to this problem is quite easy: provided that a function `count` is available which counts the occurrences of an element in the array, it is only matter of using that function for all the elements in the array and return as soon as it return an odd number. 
 Listing \ref{list:find_odd_occurring_element_bruteforce_rawloop} shows a possible implementation in C++ using the standard library function \texttt{count} for counting the number of occurrences of a given number in the array.

\lstinputlisting[language=c++, caption=Brute force solution to the problem of finding the odd occurring element in an array.,label=list:find_odd_occurring_element_bruteforce_rawloop]{sources/find_odd_occurring_element/find_odd_occurring_element_solution1.cpp}

What the code above is really trying to do is to \textbf{find} the element appearing an odd number of times. Instead of using a raw loop for doing so, the code can be made much more expressive (which is always appreciated and positively evaluated by the interviewer) by using the standard \texttt{find\_if} (or equivalents in other languages) function as shown in the Listing \ref{list:find_odd_occurring_element_bruteforce_standard} 

\lstinputlisting[language=c++, caption=Brute force solution to the problem of finding the odd occurring element in an array using only standard libraries functions.,label=list:find_odd_occurring_element_bruteforce_standard]{sources/find_odd_occurring_element/find_odd_occurring_element_solution2.cpp}


This is however considered a poor solution because the time complexity is $O(n^2)$ which is much higher than the optimal, while space complexity is constant. 
Please also note that the code in Listing \ref{list:find_odd_occurring_element_bruteforce_rawloop}  also handles the case when the input is invalid. Even if the interviewer says is it not necessary to take care of this scenario, it is good to show that the code can handle it without big penalty. By adding a throw statement, both expressiveness and  performances are preserved\footnote{ In-fact throwing an exception is cheap when the exception is not raised (main model used today for exceptions (Itanium ABI, VC++ 64 bits Zero-Cost model exceptions. More info here: \url{http://www.open-std.org/jtc1/sc22/wg21/docs/TR18015.pdf})} , and this add a bonus point to the solution. 

\subsection{Linear time and space solution}
\label{find_odd_occurring_element:sec:map}

In order to speed up the process of keeping count of how many times each element appear in the input array a map-like structure can be adopted, where the keys are the input numbers and the values are integers representing the number of times each element appear in the array. If an hash-map is used then this effectively reduces the time complexity down to O(n) (on average) but the space complexity increases to linear as well. However keeping track of the actual number of times an element appears is also unnecessary, mainly because all it is needed is the information whether or not the number of times it appears is even or odd. A single bit is sufficient to store this information, and thus, a bool can be used as a value type for the hash-map.
The in Listing \ref{list:find_odd_occurring_element_bruteforce_linearspace} shows a possible implementation of the idea above.


\lstinputlisting[language=c++, caption=Linear time and space solution to the problem of finding the odd occurring element in an array.,label=list:find_odd_occurring_element_bruteforce_linearspace]{sources/find_odd_occurring_element/find_odd_occurring_element_solution3.cpp}

This solution is considered fairly good and has a time and complexity of $O(n)$.

\subsection{Linear time and constant space solution}
\label{find_odd_occurring_element:sec:constant_space}

There is however a way to solve this problem in constant space and linear time. This solution makes itself appearent once one analyses the property of the $\oplus$ fuunction. 
The  function can be thought as the equivalent of the sum for bits and has two interesting properties that can be used to derive a solution for this problem:
\begin{itemize}
\item The neutral element for the $\oplus$ is the number $0$. It means that xor-ing a number $x$ with $0$ always results in $x$ i.e.  $x \oplus 0 = x$
\item xor-ing an element with itself always results in 0 i.e. $x \oplus x = 0$
\end{itemize}


This means that when xor-ing an element $x$ with itself an odd number of times the result is $x$: $(x \oplus x) \oplus x  = (0 \oplus x) = x)$ but, xor-ing  an even number of times results in $0$:  $(x \oplus x) \oplus (x \oplus x) = 0 \oplus 0  = 0$

Why is this useful? It is very useful because it is known that the all input integers are occurring an even number of times except of one. If all numbers are xor-ed together then all it is left at the end is the number appearing an odd number of times. 
W.r.t. to the example above $A=\{4,3,6,2,4,2,3,4,3,3,6\}$ we obtain: $4 \oplus 3 \oplus  6 \oplus 2 \oplus 4 \oplus 2 \oplus 3 \oplus 4 \oplus 3 \oplus 3 \oplus 6 = 4$. the $\oplus$ operation is commutative, distributive and associative, and thus, the operation above can be rearranged as follows (sorting  the input array): $\underbrace{(2 \oplus 2)}_{0} \oplus \underbrace{(3 \oplus 3 \oplus 3 \oplus 3)}_{0} \oplus \underbrace{(4 \oplus 4 \oplus 4)}_{4} \oplus \underbrace{(6 \oplus 6)}_{0} = 4$.

An implementation of the idea above can be found in Listings \ref{list:find_odd_occurring_element_bruteforce_final1} and \ref{list:find_odd_occurring_element_bruteforce_final2} (implemented using C++ standard library functions).

\lstinputlisting[language=c++, caption=Linear time and constnat space solution to the problem of finding the odd occurring element in an array using XOR $\oplus$.,label=list:find_odd_occurring_element_bruteforce_final1]{sources/find_odd_occurring_element/find_odd_occurring_element_solution4.cpp}


\lstinputlisting[language=c++, caption=Linear time and constnat space solution to the problem of finding the odd occurring element in an array using XOR $\oplus$ and C++ standard library accumulate function.,label=list:find_odd_occurring_element_bruteforce_final2]{sources/find_odd_occurring_element/find_odd_occurring_element_solution5.cpp}
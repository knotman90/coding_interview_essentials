%!TEX root = ../main.tex
%%%%%%%%%%%%%%%%%%%%%%%%%%%%%%%%%%
% Links:
%
% Difficulty:
% Companies: 
%%%%%%%%%%%%%%%%%%%%%%%%%%%%%%%%%%

\chapter{Search in sorted and rotated array}
\label{ch:search_sorted_rotated_array}
\section*{Introduction}
The problem presented in this chapter is considered one of the classic among interview questions as has been asked in countless interviews. It can be considered an evolution of the problem of finding the minimum element in a sorted and rotated array which was covered in chapter \ref{ch:min_rotated_array} at page \pageref{ch:min_rotated_array}. The two problems are so linked together that in-fact, it is possible to solve this problem by using the other. 

\section{Problem statement}
\begin{exercise}
Write a function that given an ascending sorted array $A$ of lenght $n$ with no duplicates and rotated around a pivot $p$ (meaning that the array has been rotated such that the smallest element of $A$ ends up at index $p$), and an integer $t$, returns:
\begin{itemize}
	\item if $t$ does not exists in $A$ it returns $-1$ 
	\item otherwise the index of $A$ where $t$ appears.
\end{itemize}


	\begin{example}
		\hfill \\
		Given $A=\{3,4,5,6,1,2\}$ and $t=5$ the function returns $2$.
		
	\end{example}

	\begin{example}
		\hfill \\
		Given $A=\{3,4,5,6,1,2\}$ and $t=7$ the function returns $-1$.
		
	\end{example}
\end{exercise}

\section{Clarification Questions}

\begin{QandA}
	\item Are all the elements unique? 
	\begin{answered}
		\textit{Yes, you can assume all the elements are unique}
	\end{answered}
	\item Can the input array be empty?
	\begin{answered}
		\textit{No, you might assume the array contains at least one element.}
	\end{answered}
\end{QandA}


\section{Discussion}
\label{search_sorted_rotated_array:sec:discussion}


\subsection{Brute-force}
\label{search_sorted_rotated_array:sec:bruteforce}
As for the problem of finding the minimum in a sorted and rotated array (Chapter \ref{ch:min_rotated_array}) the brute-force solution is trivial and consist of simply running a linear search in the entire array as shown in Listing \ref{list:search_sorted_rotated_array:bruteforce}.
Not surprisingly, the complexity of this implementation if linear in time and constant in space.

\lstinputlisting[language=c++, caption={Brute force solution (linear search) to the problem of finding an element in a sorted and potentially rotated array.},label=list:search_sorted_rotated_array:bruteforce]{sources/search_sorted_rotated_array/search_sorted_rotated_array_solution1.cpp}

\subsection{Logarithmic time solution}
\label{search_sorted_rotated_array:sec:log}
The solution presented in Section \ref{search_sorted_rotated_array:sec:bruteforce} is far from being optimal because we can solve this problem in logarithmic time and constant space (as we did for the problem in Chapter \ref{ch:min_rotated_array}).

The main idea is that we need to take advantage of the fact that the array is sorted and that if we know the pivot location, $p$ then we can logically divide the array into two starting and ending at indices:
\begin{enumerate}
	\item $0$ and $p-1$
	\item $p$ and $n-1$ ($n$ is the size of the array)
\end{enumerate}
Both arrays are sorted and thus binary search can be applied in each of the subarrays separately. This is why we can reuse the solution to the problem of finding the minimum element in a sorted and rotated array to solve this one.

To summarize the algorithm proceed as follows:
\begin{itemize}
	\item search for the pivot index $p$
	\item search for $t$ in $A[0,p-1]$. If the search is successful return the found index for $t$.
	\item search for $t$ in $A[p,n-1]$.  If the search is successful return the found index for $t$.
	\item None of the searches had success. $t$ is not in the array. Return $-1$.
\end{itemize}

This algorithm can be implemented as shown in Listing \ref{list:search_sorted_rotated_array:log}. Note that the function \inline{find_idx_min} is almost identical to the one in Listing \ref{list:min_rotated_array_log} and has been modified so to return the index instead of the value for the smallest element in the array.

Also note that the function \inline{midpoint} is not implemented as \inline{(l+r)/2} because that might cause overflow during the computation of \inline{(l+r)} even if the final result fits in a \inline{int}. Specifically, it fails if the sum of low and high is greater than the maximum positive \inline{int} value ($2^{31} - 1$ in most C++ implementation). The sum overflows to a negative value, and the value stays negative when divided by two. In C++ this causes an array index out of bounds with unpredictable results. In Java, it throws \inline{ArrayIndexOutOfBoundsException}.

\lstinputlisting[language=c++, caption={Log time solution (using binary search) to the problem of finding an element in a sorted and rotated array.},label=list:search_sorted_rotated_array:log]{sources/search_sorted_rotated_array/search_sorted_rotated_array_solution2.cpp}


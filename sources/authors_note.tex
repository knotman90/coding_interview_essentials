\chapter*{A note from the author}
\addcontentsline{toc}{chapter}{A note from the author}

\section*{How I began writing this book}
I started writing what eventually become this manuscipt in 2018. 
Until then, I followed a quite strict daily study routine that consisted of solving at least one coding interview/competitive programming challenge and, only for those worth the effort, to put together a markdown document with a summary of the problem and all the solution approaches plus the thought process that lead me to them that proved effectively in solving the challenge. 
Therefore, right from the beginning, solving a coding interview question meant for me more than just writing solutions and associated unit-tests. My main goal was writing code and associated notes that I could use to boostrap my understanding of the problem months or years later.
The idea I had in mind was that, when was time to throw myself into a real interview I could use this material as a reference and sharpen my preparation with material that I was sure it was correct I could absorb and understand.

Over the years I accumulated a lot of material that I eventually started sharing with colleagues at the university and at work. 
Many of them found the content and the format of my notes useful and convinced me to polish, add illustrations and organize them into a proper collection.

\section*{C++ as language for all the solutions}
Almost all the solution code in this book is written in C++ (C++14 and newer version). 
I have been using C++ academically and professionally since I basically started programming and the original material I have collected over the years was almost entirely C++ code. 
The reason I did not decide to rewrite it in another language is that C++ is still nowadays an extremely popular language and engineers skilled in this language are in big demand which, in turn means that C++ is very much used in interviews!
C++ is also one of the most adopted languages in the competitive programming community and this makes it easier to find resources on problems.

I tried my best not to use any C++ specific features or hacks so that it would be relatively easy to port every piece of code in this book to another mainstream imperative language like Python or Java. 
Eventually during an interview you need to code (hopefully elegant, fast and readable) solutions, but, without the right algorithm, these would be vain efforts. Therefore the focus of this book is more on trying to convey the insights and core ideas of the best algorithms and not on a particular feature of a particular programming languages.
This does not mean the code you will find in this book is sloppy or C++ unidiomatic. Quite the opposite, I tried to make full use of the its potential.


\section*{Rest}
Write as if you spoke directly to your readers about your written work.
preparing 
Short summary of the book. 
How did you decide to write this book? You started off with a set of notes and written code. you noticed that were useful to others and slowly decided to organize them and add illustrations to it. this is how it started. 

Why did you decide to go for c++ and not another language? Consider that the ideas and code in the text can be easily ported to other mainstream languages. I tried my best not to introduce any language/compiler/system specific instructions of hacks.

It is important to remember this is a book about coding interview and not on algorithm. 
identify significant omissions, abridgments, simplifications, or inventions; 


what was the writing process?
creation process, including composition, revision, research, challenges, and impactful events along the way. personal growth and learning through the writing process; 
Recount challenges while writing => create illustrations that were able to integrate the text and the code was not easy. 


The author’s note can be used to give credit where credit is due by acknowledging individuals’ contributions, references, and other’s materials included in the core text.

How is the book organised? Why this specific organization?


\section*{Audience}




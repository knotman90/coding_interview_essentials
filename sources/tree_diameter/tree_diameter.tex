%!TEX root = ../main.tex
%%%%%%%%%%%%%%%%%%%%%%%%%%%%%%%%%%
% Links:
%
% Difficulty:
% Companies: 
%%%%%%%%%%%%%%%%%%%%%%%%%%%%%%%%%%

\chapter{Tree Diameter}
\label{ch:tree_diameter}
\section*{Introduction}

\section{Problem statement}
\begin{exercise}
 Given a binary tree, you need to compute the length of the diameter of the tree. The diameter of a binary tree is the length of the longest path between any two nodes in a tree.

 The length of path between two nodes is represented by the number of edges between them.

 The definition of the tree is shown in Listing \ref{list:verify_BST:tree_structure}.


	\begin{example}
		\hfill \\
	Given the binary tree shown in Figure \ref{fig:tree_diameter:example1} the function return $3$. One path of such length is from node $4$ (or $5$) to node $3$.
	\end{example}

	\begin{example}
		\hfill \\
		
	\end{example}
\end{exercise}

\section{Clarification Questions}

\begin{QandA}
	\item 
	\begin{answered}
		\textit{}
	\end{answered}
	
\end{QandA}

\section{Discussion}
\label{tree_diameter:sec:discussion}


\subsection{Brute-force}
\label{tree_diameter:sec:bruteforce}

\lstinputlisting[language=c++, caption={Sample Caption},label=list:tree_diameter]{sources/tree_diameter/tree_diameter_solution1.cpp}


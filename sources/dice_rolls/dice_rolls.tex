%!TEX root = ../main.tex
%%%%%%%%%%%%%%%%%%%%%%%%%%%%%%%%%%
% Links:
%
% Difficulty:
% Companies: 
%%%%%%%%%%%%%%%%%%%%%%%%%%%%%%%%%%

\chapter{Number of Dice Rolls With Target Sum}
\label{ch:dice_rolls}
\section*{Introduction}

\section{Problem statement}
\begin{exercise}
You have $d$ dice, and each die has $f$ faces numbered $1, 2, \ldots..., f$.

Return the number of possible ways (out of $f^d$ total ways) modulo $10^9 + 7$ to roll the dice so the sum of the face up numbers equals a given number $t$.

	\begin{example}
		\hfill \\
		Given $d=1$, $f=6$ and $t=6$ the function should return $1$. There is only one way of obtaining $6$ with $1$ $6$-faced dice.
	\end{example}

	\begin{example}
		\hfill \\
		Given $d=2$, $f=6$ and $t=7$ the function should return $6$. The following are the possible way of obtaining $7$ from $2$ $6$-faced dices.
		\begin{enumerate}
			\item $1+6$
			\item $2+5$
			\item $3+4$
			\item $4+3$
			\item $5+2$
			\item $6+1$
		\end{enumerate}
		
	\end{example}

	\begin{example}
		\hfill \\
		Given $d=2$, $f=3$ and $t=7$ the function should return $0$ because the highest number obtainable by rolling two dices with $3$ faces is $6$.
	\end{example}
\end{exercise}

\section{Clarification Questions}

\begin{QandA}
	\item 
	\begin{answered}
		\textit{}
	\end{answered}
	
\end{QandA}

\section{Discussion}
\label{dice_rolls:sec:discussion}


\subsection{Brute-force}
\label{dice_rolls:sec:bruteforce}

\lstinputlisting[language=c++, caption={Sample Caption},label=list:dice_rolls]{sources/dice_rolls/dice_rolls_solution1.cpp}


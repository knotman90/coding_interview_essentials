%!TEX root = ../main.tex
%%%%%%%%%%%%%%%%%%%%%%%%%%%%%%%%%%
% Links:
%
% Difficulty:
% Companies: 
%%%%%%%%%%%%%%%%%%%%%%%%%%%%%%%%%%

\chapter{Delete duplicates from Linked List}
\label{ch:delete_duplicates_list}
\section*{Introduction}
The problem presented in this chapter is an easy one on Linked list that has been asked to preliminary interview steps many times already. It is therefore very important that this problem is understood well and you can implement it quickly and most importantly flawlessly during an interview. 
\section{Problem statement}
\begin{exercise}
Given a singly linked link $L$ (see definition at Listing \ref{list:delete_duplicates_list:linked_list}), return a linked list with no duplicated.
\end{exercise}
\begin{lstlisting}[language=c++, caption=Singly Linked list definition,label=list:delete_duplicates_list:linked_list]
template<typename T>
struct Node {
    T val;
    Node *next;
    Node(T x) : val(x), next(nullptr) {}
};
\end{lstlisting}

\begin{example}
	\hfill \\
	\begin{itemize}
		\item[-] Input: $1 \rightarrow 1 \rightarrow 2$
		\item[-] Output: $1 \rightarrow 2$
	\end{itemize}
\end{example}

\begin{example}
	\hfill \\
	\begin{itemize}
		\item[-] Input: $1 \rightarrow  2 \rightarrow  2 \rightarrow  2 \rightarrow  2 \rightarrow  3 \rightarrow  4 \rightarrow  4 $
		\item[-] Output: $1 \rightarrow  2 \rightarrow 3 \rightarrow  4$
	\end{itemize}
\end{example}

\section{Clarification Questions}

\begin{QandA}
	\item \begin{questionitem} \begin{question} Can the input list be modified?  \end{question} 	 
    \begin{answered}
		\textit{Yes.}
	\end{answered} \end{questionitem}
	
\end{QandA}

\section{Discussion}
\label{delete_duplicates_list:sec:discussion}
As you can imagine there are many approaches that can be adopted in order to solve this problem, but we will focus on basically two, with the second being a refinement of the first which will lead to an elegant yet efficient solution.

\subsection{Brute-force}
\label{delete_duplicates_list:sec:bruteforce}
The easiest solution possible consist in:
\begin{enumerate}
	\item Create a vector that basically contains a copy of the list
	\item Remove duplicates from the vector
	\item Create a new List with the content of the duplicate free vector
\end{enumerate}
This solution is straightforward but as you can imagine not optimal. A possible implementation is shown in Listing \ref{list:delete_duplicates_list_brite_force1} where for the step $2$ the  remove-erase idiom\cite{cit::wiki::remove-erase} is used to remove duplicates from the vector(the erase part is actually not necessary in this case).

\lstinputlisting[language=c++, caption={C++ solution $O(n)$ time and $O(n)$ space solution to the problem of removing duplicates from a Linked List using std::remove.},label=list:delete_duplicates_list_brite_force1]{sources/delete_duplicates_list/delete_duplicates_list_solution1.cpp}

This solution is optimal in time but we are not taking advantage of the fact that the we can modify the list in place and it is not necessary to create a brand new List to be returned. 

\subsection{In-place $O(1)$ space solution}
\label{delete_duplicates_list:sec:linear_space}
In the solution describe in Section \ref{delete_duplicates_list:sec:discussion} we did use of additional space to both remove the duplicates and also to avoid the pain of rearranging the input list by creating a brand new list containing no duplicates. It is possible though with a bit of thinking to write a in-place solution that uses no additional space. 

The main idea is that since the list is \textbf{sorted}, duplicate element will be one after the other, and we can take advantage of this fact by simply ignoring pair of consecutive nodes that have the same payload. Ignored nodes can therefore be deleted. The only complications is that we need to make sure to connect the first occurrence of every Node in the list with each other For this reason we need to therefore remember the first Node of a stride of the same value.

This idea is implemented in Listing \ref{list:delete_duplicates_list_lineartime}. Note that:

\begin{itemize}
	\item[-] The base case \lstinline[columns=fixed]{if(!head || !head->next) return head;} is making sure that if we are examining the last element of the list (or an empty list) then there is no duplicate to ignore and thus we can return this element.
	\item[-] otherwise, we are looking at a list with \textbf{at least} two elements. These two elements can be potentially the start of a stride of equal element that we want to ignore. We keep a pointer,\lstinline[columns=fixed]{Node<T>* head;} (which is never modified) to the first element  of the stride and advance a second pointer, \lstinline[columns=fixed]{Node<T>* head_n;}, until we either reach the end of the list or we find an element that is different from the first one, \lstinline[columns=fixed]{while ( head_n && head -> val == head_n -> val)}. All the advanced elements are deleted. At the end of the loop the second pointer  is pointing to either:
	\begin{itemize}
		\item[-] An element different than the element pointed by the first one. The stride of equal elements is processed and  \lstinline[columns=fixed]{head->next} can now point to the second pointer.
		\item[-] \lstinline[columns=fixed]{nullptr}. We have reached the end of the input list. We are done.
	\end{itemize}
\end{itemize}


The time and space complexity for this approach are $O(n)$ and $O(1)$, respectively. All the nodes of the list are  visited at most once.

\lstinputlisting[language=c++, caption={C++ solution $O(n)$ time and $O(1)$ space solution to the problem of removing duplicates from a Linked.},label=list:delete_duplicates_list_lineartime]{sources/delete_duplicates_list/delete_duplicates_list_solution2.cpp}


\section{Common Variations and follow-up questions}
One follow-up question that the interviewer might be keen to ask is related to the deletion of the duplicate nodes. In the Listing \ref{list:delete_duplicates_list_lineartime} we see that the nodes are deleted using the \lstinline[columns=fixed]{operator delete}, but what if the list was not allocated using \lstinline[columns=fixed]{operator new}? The question is left for the reader.

\begin{exercise}
How would the solution change if the nodes were allocated using a custom allocator? (spoiler, a custom deleter is also needed.)
\end{exercise}

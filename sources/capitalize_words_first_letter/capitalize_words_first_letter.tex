%!TEX root = ../main.tex
%%%%%%%%%%%%%%%%%%%%%%%%%%%%%%%%%%
% Links:
%
% Difficulty:
% Companies: 
%%%%%%%%%%%%%%%%%%%%%%%%%%%%%%%%%%

\chapter{Capitalize the first letters of every words}
\label{ch:capitalize_words_first_letter}
\section*{Introduction}
Editing text is probably one of the most basic and common operations computers are still used for nowadays. There are a huge number of editors out there, some of them are specialized for a particular kind of users (think of the zoo of editors a programmer can choose from) while others are intended to be for a broader audience like Word or LibreOffice writer. 

Imagine for a second to be working on a feature for the new version of Word that is supposed to make the tedious tasks of converting a particular piece of text to a sort of title case\footnote{All words are capitalized, except non-initial articles like “a, the, and”, etc.}. The idea is that the used would highlight a portion of text and then have the text modified in place by simply pressing a button instead of manually changing every single letter. 

This simple task if often part of preliminary interview stages and it is used as a warm-up problem mostly due to its simplicity. 
In this chapter we will discuss how the core such feature can be implemented and we will have a look at a number of possible implementations and solutions approaches. This problem is not hard and therefore the focus of this chapter is more on making sure the final solution is readable and easy to understand.

\section{Problem statement}
\begin{exercise}
Write a function that given a string $s$, modifies so that every first letter of every word in $s$ is in capital case while leaving the rest of the characters untouched.

	\begin{example}
		\hfill \\
		Given the string \verb\"  arturo benedetti michelangeli is the best pianist ever     \"
		The function should turn it into: \verb\"  Arturo Benedetti Michelangeli Is The Best Pianist Ever  \"
	\end{example}

	\begin{example}
		\hfill \\
		Given the string:
		\begin{verbatim}
			"Truth May Seem BUt Cannot be;
			Beauty brag but ’tis not she;
			TruTh and beauty buried be."
		\end{verbatim}
		The function should turn it into: 
		\begin{verbatim}
			"Truth May Seem BUt Cannot Be;
			Beauty Brag But ’tis Not She;
			TruTh And Beauty Buried Be."
		\end{verbatim}
	\end{example}
	
\end{exercise}
	
\section{Discussion}
\label{capitalize_words_first_letter:sec:discussion}
This problem does not require to come up with a smart algorithm in order to get the job done. The idea behind this problem is more that you should be able to put a working implementation on the table in a reasonable short amount of time and spend the rest of the time polishing it so that it is clean and easy to understand. 


What are the implications of having to capitalize only the first letter of every word? Let's start by first looking at what makes a letter the first letter of a word. 
A character is the beginning of a word if it is any of the following is true:
\begin{itemize}
	\item is not a space and it is preceded by a space,
	\item is not a space and it is the first character of the string.
\end{itemize}
Any other character is either a space (for which the notion of lower/upper case is not defined)
or is in the middle of a word. 

Given this definition, all is necessary to do to solve this problem is to search for any character in the input string satisfying any of the criteria above as shown in Listing \ref{list:capitalize_words_first_letter_simple}

\lstinputlisting[language=c++, caption=Linear time constant space solution.,label=list:capitalize_words_first_letter_simple]{sources/capitalize_words_first_letter/capitalize_words_first_letter_solution4.cpp}

The code works in two phases:
\begin{enumerate}
	\item make sure that the first character of $s$ is handled properly,
	\item takes care of the rest of the characters form the position $1$ (skipping the very first one) onward.
\end{enumerate}



%%%%%%%%% ricomincia da qui


\begin{enumerate}
	\item  all spaces need to be ignored
    \item  the first non space character needs to be modified
    \item  all the rest of non-space characters can be ignored
\end{enumerate}

This three rules  can be easily converted into code as shown in Listing \ref{list:capitalize_words_first_letter_iterator}. The code is clearly divided into three distint blocks each doing one of the item listed above i.e. processing the text from left to right and, in this order, ignoring all the spaces, until a letter is found, that is capitalized, and finally all the rest of the word is also ignored. The process repeats until the text is fully processed. 

\lstinputlisting[language=c++, caption=Solution to the problem of turning all first letter of every word in a text into capital case. Works by ignoring all the spaces until a letter is found (marking the beginning of a word) and turning it into capital case and then ignoring the rest of the letters of that word. The process repeats until all the text is processed.,label=list:capitalize_words_first_letter_iterator]{sources/capitalize_words_first_letter/capitalize_words_first_letter_solution1.cpp}

Note that the check \lstinline[columns=fixed]{it != end(s)} appears first in the \lstinline[columns=fixed]{&&} expression. If it points to the end of the word then \lstinline[columns=fixed]{*it} is not evaluated (this behavior of \lstinline[columns=fixed]{&&} is known as short circuit\footnote{In a boolean expression the second argument is executed or evaluated only if the first argument does not suffice to determine the value of the expression.}\cite{cit:wiki:shortcircuit}).

The complexity of this solution is linear in space and constant in time, as the every letter is not read of modified more than once and no additional space is allocated.

\section{Solution using \texttt{std::adjacent\_find}}

An alternative way of solving this problem is by noticing that all the characters that need satisfy both the followings:
\begin{itemize}
    \item [-] are letters (not spaces)
    \item [-] are preceded by a bspace
\end{itemize}
The core idea is that when a pair made of a space followed by a  non-space is  found in the text, then the second element of this pair can be made uppercase. This works because whenever such a pair is found, then its second element represent the starting point of a new word. This methods do not work when the are not spaces at the beginning of the text and the first word starts with a lower case letter. Thus, a special treatement for this case is necessary. The first \lstinline[columns=fixed]{if} handles this special case.

One way to implement this idea in C++ is by using the standard algorithm \lstinline[columns=fixed]{std::adjacent_find} as shown in Listing \ref{list:capitalize_words_first_letter_adj_find}.

\lstinputlisting[language=c++, caption=Solution to the problem of turning all first letter of every word in a text into capital case. uses the function \href{https://en.cppreference.com/w/cpp/algorithm/adjacent_find}{\texttt{std::adjacent\_find}}\cite{cit::std::adjancefind} to find subsequent occurrence of a space and a letter in the text. This sequence marks the beginning of a word. by,label=list:capitalize_words_first_letter_adj_find]{sources/capitalize_words_first_letter/capitalize_words_first_letter_solution2.cpp}


The complexity of the solution in Listing is linear in time and constant in space and it is considered a very good solution. This version is asymptotically as fast  as the one presented in Listing \ref{list:capitalize_words_first_letter_adj_iterator} but much more expressive, thus is considered better in a interview.

\section{Common Variations}
\label{capitalize_words_first_letter:sec:variation}

\subsection{User provided function}
It is not unlikely that this problem can be posed such that the operation to be applied on it is different than the one which capitalize the letter. A way to solve this is to make the solution generic such that it accept the operation from the user.  Listing \ref{list:capitalize_words_first_letter_userdefined} shows a possible implementation of the idea above. The code takes as a input a string and a function which takes as a input a character and return a character and applies it to every first letter of all words in the text.

\lstinputlisting[language=c++, caption=Solution to the problem of modifying all first letter of every word in a text using a user provided function.,label=list:capitalize_words_first_letter_userdefined]{sources/capitalize_words_first_letter/capitalize_words_first_letter_solution3.cpp}


\subsection{Modify the every $k^{th}$ character of every word}
Another common variation asks to modify (to make upper, or lower case for instance) every $k^{th}$ character of a word (if exists). For instance you might be asked to:

\begin{exercise}
Given a string s, modify s such that every $3^{rd}$ letter of every word in $s$ is modifyed accordin to a function provided by the user. The rest of the string should not be touched.
\end{exercise}
The implementation of this exercice is left to the reader.


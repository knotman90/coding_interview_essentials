%!TEX root = ../main.tex
%%%%%%%%%%%%%%%%%%%%%%%%%%%%%%%%%%
% Links:
%
% Difficulty:
% Companies: 
%%%%%%%%%%%%%%%%%%%%%%%%%%%%%%%%%%

\chapter{Capitalize the first letters of every words}
\label{ch:capitalize_words_first_letter}
\section*{Introduction}
Text editing is one of the most basic and common operations computers are used for. There are vast numbers of text editors out there, some of them specialized for a particular type of users (e.g. specific editors for programmers such as \begin{enumerate*}
	\item vi,
	\item GNU Emacs,
	\item gedit,
	\item TextPad,
	\item Visual Studio Code,
	\item Eclipse,
	\item Sublime Text,
	\item Qt Creator,
	\item etc.
\end{enumerate*}
), while others are intended for a broader audience and use case such as  MS Word or LibreOffice writer. 


A problem that is often posed during coding interviews asks us to put ourselves in the place of a software engineer working on a feature for the newest version of Word that is supposed to make the tedious tasks of converting a particular piece of text into a variant of the title case:\footnote{All words are capitalized, except non-initial articles like \quotes{a}, \quotes{the, }and", etc.}.
The idea is that the user would highlight a portion of text and then have the text modified in place by simply pressing a button instead of manually changing every single letter. 


These types of questions often appear as a warm up during the preliminary stages as it isn't inherently complex.  As such, the main focus of this chapter is to ensure that the final solution is readable and easy to understand, rather then creating a smarter algorithm. 

We will first discuss how the core feature can be implemented and then examine a number of possible implementations and solutions. 


\section{Problem statement}
\begin{exercise}
Write a function that given a string $s$, modifies it so that every first letter of every word in $s$ is in upper case while leaving the rest of the characters untouched.

	\begin{example}
		\hfill \\
		Given the string \verb\"arturo benedetti michelangeli is the best pianist ever\". The function should turn it into: \verb\"Arturo Benedetti Michelangeli Is The Best Pianist Ever\"\footnote{A.B.M (5 January 1920 - 12 June 1995) was an Italian classical pianist considered one of the greatest pianists of all time. He was perhaps the most reclusive, enigmatic and obsessive among the handful of the world's legendary pianists.}
	\end{example}

	\begin{example}
		\hfill \\
		Given the string:
		\begin{verbatim}
			"Truth May Seem BUt Cannot be;
			Beauty brag but ’tis not she;
			TruTh and beauty buried be."
		\end{verbatim}
		The function should turn it into: 
		\begin{verbatim}
			"Truth May Seem BUt Cannot Be;
			Beauty Brag But ’tis Not She;
			TruTh And Beauty Buried Be."
		\end{verbatim}
	\end{example}
	
\end{exercise}
	
\section{Discussion}
\label{capitalize_words_first_letter:sec:discussion}
This problem does not require coming up with a smart algorithm in order to get the job done. Our goal is to be able to put a working implementation on the table in a reasonably short amount of time and spend the rest of the time polishing it so that it is clean and easy to understand. 

What are the practical implications of having to capitalize only the first letter of every word? Let's start by first looking at what makes a letter the first letter of a word. 
A character is the beginning of a word if any of the following is true:
\begin{itemize}
	\item is not space and it is preceded by a space;
	\item is not space and it is the first character of the string.
\end{itemize}

Any other character is either space (for which the notion of lower/upper case is not defined) or is in the middle of a word. 
Given this definition, all we need to do to solve this problem is to search for any character in the input string satisfying any of the criteria above as shown in Listing \ref{list:capitalize_words_first_letter_simple}.

\lstinputlisting[language=c++, caption=Linear time constant space solution.,label=list:capitalize_words_first_letter_simple]{sources/capitalize_words_first_letter/capitalize_words_first_letter_solution4.cpp}

Listing \ref{list:capitalize_words_first_letter_simple} works in two phases:
\begin{enumerate}
	\item makes sure that the first character of $s$ is handled properly (depending on whether it is a space or not);
	\item takes care of the rest of the characters from the position $1$ (skipping the very first one) onward.
\end{enumerate}

\subsubsection{\texttt{std::adjacent\_find}}
The same idea discussed above and shown in Listing \ref{list:capitalize_words_first_letter_simple} can be implemented using the function \inline{std::adjacent_find}\cite{cit::std::adjancefind} from the STL which can be used to search, in a range, for a pair of subsequent elements satisfying user-provided criteria. In the context of this solution we can use it to find all pairs composed by a space followed by a letter; which we know is the letter that has to be capitalized as it marks the beginning of a word.
Listing \ref{list:capitalize_words_first_letter_adj_find} implements this idea. 

\lstinputlisting[language=c++, caption=Linear time constant space solution using \href{https://en.cppreference.com/w/cpp/algorithm/adjacent_find}{\texttt{std::adjacent\_find}}\cite{cit::std::adjancefind}. by,label=list:capitalize_words_first_letter_adj_find]{sources/capitalize_words_first_letter/capitalize_words_first_letter_solution2.cpp}

The complexity of the Listing \ref{list:capitalize_words_first_letter_adj_find}  is linear in time and constant in space and it has the same asymptotical complexity profile as the one presented in Listing \ref{list:capitalize_words_first_letter_simple} with the added benefit of being more expressive.

\subsubsection{Recursive solution}

Another way to solve this problem is to adopt a recursive approach as follows:
\begin{enumerate}
	\item find the first character in the string; 
	\item transform it in uppercase;
	\item ignore all the subsequent non-space characters until a space or the end of the string is reached.
\end{enumerate}
When we reach a space we repeat the process from the beginning; otherwise we stop.  At that point the whole string is modified so that only the first character of every word is in uppercase and the rest of the string is untouched.
These rules can also be easily turned into code as shown in Listing \ref{list:capitalize_words_first_letter_iterator}.

\lstinputlisting[language=c++, caption=Linear time constant space solution. ,label=list:capitalize_words_first_letter_iterator]{sources/capitalize_words_first_letter/capitalize_words_first_letter_solution1.cpp}

The code is clearly divided into three distinct blocks; each performing one of the tasks listed above (see code comments).
The variable \inline{it} is an iterator pointing to the element currently under examination and it is used by the outer loop to determine whether the string has been completely processed.
\inline{it} is moved inside the body of the loop which, by processing the text from left to right, ignores all spaces until a letter is found (first inner loop).
This letter is then capitalized and  \inline{it} is moved forward so that all the non-space intra-word characters are ignored (second inner loop).
This process repeats until the text is fully processed. 


Note how we use short-circuit evaluation\footnote{Also known as \textit{minimal evaluation} or \textit{McCarthy evaluation} refers to the semantic of certain boolean operators in which the second argument is executed or evaluated only if the first argument does not suffice to determine the value of the expression.}\cite{cit:wiki:shortcircuit}
in the \inline{while (it != end(s) && *it == ' ')} expression so as to always be sure \inline{it} is pointing to a valid element when we dereference it.

The complexity of this solution is linear in time as every letter is read or modified at most once. The space complexity is constant.



\section{Common Variations}
\label{capitalize_words_first_letter:sec:variation}

\subsection{Apply an user provided function}
Sometimes this problem can be posed such that the operation to be applied to the letter is different than capitalization; or we can be even asked to write a high order function that takes the operation to be performed as an additional parameter (as a lambda for instance).

We can use the same core ideas discussed above even if we decide to go for a generic solution where we accept a function from the user. The complications are only syntactical as shown in the Listing \ref{list:capitalize_words_first_letter_userdefined}, where we can see how a generic solution can be implemented.
The code takes as an input a string and, in addition to the other solutions discussed above, a function \inline{char f(char)} which takes as an input a character and returns a character. This function is used in place of the \inline{std::to_upper}.

\lstinputlisting[language=c++, caption=Generic version of Listing \ref{list:capitalize_words_first_letter_adj_find},label=list:capitalize_words_first_letter_userdefined]{sources/capitalize_words_first_letter/capitalize_words_first_letter_solution3.cpp}

\subsection{Modify the every $k^{th}$ character of every word}
Another common variation is where we are asked to modify every $k^{th}$ character of a word if it exists. For example, you might be asked to solve the exercise below:
\begin{exercise}
Given a string $s$, modify $s$ such that every $3^{rd}$ letter of every word in $s$ is modified according to a function provided by the user. The rest of the string should remain untouched.
\end{exercise}

This variation can be solved by using any of the codes shown above as a starting point and you have the chance of solving it yourself in the next exercise.


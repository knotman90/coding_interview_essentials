%!TEX root = ../main.tex
%%%%%%%%%%%%%%%%%%%%%%%%%%%%%%%%%%
% Links:
%
% Difficulty:
% Companies: 
%%%%%%%%%%%%%%%%%%%%%%%%%%%%%%%%%%

\chapter{Capitalize the first letters of every words}
\label{ch:capitalize_words_first_letter}
\section*{Introduction}
This chapter describes a common problem on strings often asked during the preliminary whiteboard sessions, mostly because it is easy to understand and explain and can be implemented quickly. It is a good warm-up problem.
The idea is that we are given a long text and we need to capitalize every first letter of the words in it leaving the rest of the text intact.

\section{Problem statement}
\begin{exercise}
Given a string s, modify s such that every first letter of every word in s is capital. The rest of the string should not be touched.

\end{exercise}

\begin{example}
	\hfill \\
	Given the string \verb\"  arturo benedetti michelangeli is the best pianist ever     \"
	The function should turn it into: \verb\"  Arturo Benedetti Michelangeli Is The Best Pianist Ever  \"
\end{example}

\begin{example}
	\hfill \\
	Given the string:
	\begin{verbatim}
	"Truth May Seem BUt Cannot be;
	Beauty brag but ’tis not she;
	TruTh and beauty buried be."
	\end{verbatim}
	The function should turn it into: 
	\begin{verbatim}
	"Truth May Seem BUt Cannot Be;
	Beauty Brag But ’tis Not She;
	TruTh And Beauty Buried Be."
	\end{verbatim}
\end{example}


\section{Discussion}
\label{capitalize_words_first_letter:sec:discussion}
This problem does not require to come up with a smart algorithm in order to do what it should. The idea behind this problem is more that you should be able to put an implementation on the table in a reasonable amount of time which is correct from the first draft. The interviewer wants to know that you are comfortable deriving a non-trivial algorithm on the spot and that you can write correct code under pressure.
One way to solve this problem in a blink of an eye is to notice some properties of this problem:

\begin{enumerate}
	\item  all spaces need to be ignored
    \item  the first non space character needs to be modified
    \item  all the rest of non-space characters can be ignored
\end{enumerate}

This three rules  can be easily converted into code as shown in Listing \ref{list:capitalize_words_first_letter_iterator}. The code is clearly divided into three distint blocks each doing one of the item listed above i.e. processing the text from left to right and, in this order, ignoring all the spaces, until a letter is found, that is capitalized, and finally all the rest of the word is also ignored. The process repeats until the text is fully processed. 

\lstinputlisting[language=c++, caption=Solution to the problem of turning all first letter of every word in a text into capital case. Works by ignoring all the spaces until a letter is found (marking the beginning of a word) and turning it into capital case and then ignoring the rest of the letters of that word. The process repeats until all the text is processed.,label=list:capitalize_words_first_letter_iterator]{/home/dspataro/git/algorithm_articles/sources/capitalize_words_first_letter/capitalize_words_first_letter_solution1.cpp}

Note that the check \lstinline[columns=fixed]{it != end(s)} appears first in the \lstinline[columns=fixed]{&&} expression. If it points to the end of the word then \lstinline[columns=fixed]{*it} is not evaluated (this behavior of \lstinline[columns=fixed]{&&} is known as short circuit\footnote{In a boolean expression the second argument is executed or evaluated only if the first argument does not suffice to determine the value of the expression.}\cite{cit:wiki:shortcircuit}).

The complexity of this solution is linear in space and constant in time, as the every letter is not read of modified more than once and no additional space is allocated.

\section{Solution using \texttt{std::adjacent\_find}}

An alternative way of solving this problem is by noticing that all the characters that need satisfy both the followings:
\begin{itemize}
    \item [-] are letters (not spaces)
    \item [-] are preceded by a bspace
\end{itemize}
The core idea is that when a pair made of a space followed by a  non-space is  found in the text, then the second element of this pair can be made uppercase. This works because whenever such a pair is found, then its second element represent the starting point of a new word. This methods do not work when the are not spaces at the beginning of the text and the first word starts with a lower case letter. Thus, a special treatement for this case is necessary. The first \lstinline[columns=fixed]{if} handles this special case.

One way to implement this idea in C++ is by using the standard algorithm \lstinline[columns=fixed]{std::adjacent_find} as shown in Listing \ref{list:capitalize_words_first_letter_adj_find}.

\lstinputlisting[language=c++, caption=Solution to the problem of turning all first letter of every word in a text into capital case. uses the function \href{https://en.cppreference.com/w/cpp/algorithm/adjacent_find}{\texttt{std::adjacent\_find}}\cite{cit::std::adjancefind} to find subsequent occurrence of a space and a letter in the text. This sequence marks the beginning of a word. by,label=list:capitalize_words_first_letter_adj_find]{/home/dspataro/git/algorithm_articles/sources/capitalize_words_first_letter/capitalize_words_first_letter_solution2.cpp}


The complexity of the solution in Listing is linear in time and constant in space and it is considered a very good solution. This version is asymptotically as fast  as the one presented in Listing \ref{list:capitalize_words_first_letter_adj_iterator} but much more expressive, thus is considered better in a interview.

\section{Common Variations}
\label{capitalize_words_first_letter:sec:variation}

\subsection{User provided function}
It is not unlikely that this problem can be posed such that the operation to be applied on it is different than the one which capitalize the letter. A way to solve this is to make the solution generic such that it accept the operation from the user.  Listing \ref{list:capitalize_words_first_letter_userdefined} shows a possible implementation of the idea above. The code takes as a input a string and a function which takes as a input a character and return a character and applies it to every first letter of all words in the text.

\lstinputlisting[language=c++, caption=Solution to the problem of modifying all first letter of every word in a text using a user provided function.,label=list:capitalize_words_first_letter_userdefined]{/home/dspataro/git/algorithm_articles/sources/capitalize_words_first_letter/capitalize_words_first_letter_solution3.cpp}


\subsection{Modify the every $k^{th}$ character of every word}
Another common variation asks to modify (to make upper, or lower case for instance) every $k^{th}$ character of a word (if exists). For instance you might be asked to:

\begin{exercise}
Given a string s, modify s such that every $3^{rd}$ letter of every word in $s$ is modifyed accordin to a function provided by the user. The rest of the string should not be touched.
\end{exercise}
The implementation of this exercice is left to the reader.


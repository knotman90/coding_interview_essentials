%!TEX root = ../main.tex
%%%%%%%%%%%%%%%%%%%%%%%%%%%%%%%%%%
% Links:
%
% Difficulty:
% Companies: 
%%%%%%%%%%%%%%%%%%%%%%%%%%%%%%%%%%

\chapter{Exponentiation}
\label{ch:exponentiation}
\section*{Introduction}
This chatper addresses a common problem for which it is fairly easy to come up with a good solution that works in linear time and that if implemented properly can already be good enough to get the green light from the interviewer. However, in order to be sure to ace this problem a time of complexity of  $O(log_2)$ should be achieved.

\section{Problem statement}
\begin{exercise}
Implement a function that given two positive integers $n$ and $k$ calculates $n^k$.
\end{exercise}


\begin{example}
	\hfill \\
	Given $n=2$ and $k=3$ the function returns $8$.
\end{example}

\begin{example}
	\hfill \\
	Given $n=5$ and $k=2$ the function returns $25$.
\end{example}

\section{Clarification Questions}

\begin{QandA}
	\item Should the function handles the case where $k=0$?
	\begin{answered}
		\textit{Yes $k=0$ is a valid input.}
	\end{answered}
	
	\item Should the function handles integer overflow?
	\begin{answered}
		\textit{No overflow should not be accounted for. }
	\end{answered}
		
\end{QandA}

\section{Discussion}
\label{exponentiation:sec:discussion}
After just reading the problem discussion the obvious solution should immediately be discussed. This approach simply consist in performing $k$ consecutive multiplications as per the definition of exponentiation i.e. $n^k = \underbrace{n \times n \times n \times \ldots \times n}_\text{k times}$. This idea can be implemented in just a few lines as shown in Listing \ref{list:exponentiation_linear}

\lstinputlisting[language=c++, caption=Sample Caption,label=list:exponentiation_linear]{sources/exponentiation/exponentiation_solution1.cpp}

\subsection{Binary fast exponentiation}
\label{exponentiation:sec:fast_exponentiation}
The problem can be solved much faster by expoiting the fact that:
\begin{itemize}
	\item[-] if $x+y=k$ then, $n^k = n^x * n^y = n^{x+y}$ 
	\item[-] if $x \times y=k$ then, $ n^k = (n^x)^y$
\end{itemize}
The idea is that if depending on the parity of $k$ (whether $k$ is even or odd):
  \[
    n^k = \begin{cases}
        		n^{\frac{k}{2}} \times n^{\frac{k}{2}}, & \text{if  k even}\\
        		n \times n^{k-1}, & \text{if k odd}\\
	        \end{cases}
  \]
This allows for the number of multilication to be reduced by a half all the times that $k$ is even. Also note that all the times that $k$ is odd then $n^{k-1}$ can be calculated again reducing the number of multiplication by half becuase $k-1$ is even. This approach is inherintly recursive and can be coded as shown in Listing \ref{list:exponentiation_fast} but an interative solution is also quite easy to write as can be seen in Listing \ref{list:exponentiation_fast_iterative}.

\lstinputlisting[language=c++, caption=Sample Caption,label=list:exponentiation_fast]{sources/exponentiation/exponentiation_solution2.cpp}

\lstinputlisting[language=c++, caption=Sample Caption,label=list:exponentiation_fast_iterative]{sources/exponentiation/exponentiation_solution3.cpp}

%!TEX root = ../main.tex
%%%%%%%%%%%%%%%%%%%%%%%%%%%%%%%%%%
% Links: https://www.toptal.com/algorithms/interview-questions
%
% Difficulty:
% Companies: 
%%%%%%%%%%%%%%%%%%%%%%%%%%%%%%%%%%

\chapter{Exponentiation}
\label{ch:exponentiation}
\section*{Introduction}
This chatper addresses a common problem for which it is fairly easy to come up with a good solution that works in linear time and that if implemented properly can already be good enough to get the green light from the interviewer. However, in order to be sure to ace this problem a time of complexity of  $O(log_2)$ should be achieved.
The idea behind exponentiation by squaring can be applied not only to integer numbers but to many other mathematical objects such as polynomials or matrices. For instance, it can be used to calculate the $k^{th}$ fibonacci number in $log_2(k)$ steps. 

\section{Problem statement}
\begin{exercise}
Implement a function that given two positive integers $n$ and $k$ calculates $n^k$.
\end{exercise}
f

\begin{example}
	\hfill \\
	Given $n=2$ and $k=3$ the function returns $8$.
\end{example}

\begin{example}
	\hfill \\
	Given $n=5$ and $k=2$ the function returns $25$.
\end{example}

\section{Clarification Questions}

\begin{QandA}
	\item Should the function handles the case where $k=0$?
	\begin{answered}
		\textit{Yes $k=0$ is a valid input.}
	\end{answered}
	
	\item Should the function handles integer overflow?
	\begin{answered}
		\textit{No overflow should not be accounted for. }
	\end{answered}
		
\end{QandA}

\section{Discussion}
\label{exponentiation:sec:discussion}
After just reading the problem discussion the obvious solution should immediately be discussed. This approach simply consist in performing $k$ consecutive multiplications as per the definition of exponentiation i.e. $n^k = \underbrace{n \times n \times n \times \ldots \times n}\text{k times}$. This idea can be implemented in just a few lines as shown in Listing \ref{list:exponentiation_linear}

\lstinputlisting[language=c++, caption=Trivial and linear time complexity solution to the exponentiation problem.,label=list:exponentiation_linear]{sources/exponentiation/exponentiation_solution1.cpp}

\subsection{Binary fast exponentiation}
\label{exponentiation:sec:fast_exponentiation}
The problem can be solved much faster by expoiting the fact that:
\begin{itemize}
	\item[-] if $x+y=k$ then, $n^k = n^x * n^y = n^{x+y}$ 
	\item[-] if $x \times y=k$ then, $n^k = (n^x)^y$
\end{itemize}
The idea is that if depending on the parity of $k$ (whether $k$ is even or odd):
  \[
    n^k = \begin{cases}
        		n^{\frac{k}{2}} \times n^{\frac{k}{2}}, & \text{if  k even}\\
        		n \times n^{k-1}, & \text{if k odd}\\
	        \end{cases}
  \]
This allows for the number of multilication to be reduced by a half all the times that $k$ is even. Also note that all the times that $k$ is odd then $n^{k-1}$ can be calculated again reducing the number of multiplication by half becuase $k-1$ is even. This approach is inherintly recursive and can be coded as shown in Listing \ref{list:exponentiation_fast}.

\lstinputlisting[language=c++, caption=Recursive $O(log_2())$ solution to the exponentiation problem.,label=list:exponentiation_fast]{sources/exponentiation/exponentiation_solution2.cpp}

\subsection{Iterative implementation}

Another way to tackle this problem is by looking at the binary representation of the exponent $k = b_0 \times 2^0 + b_1 \times 2^1 + \ldots + b_l \times 2^l$ where $b_i$ is a binary digit. 
When plugging $k$ into $n^k$ the following is obtained:
\[
	\begin{array}{lcl}
		n^k & = &  x^{b_0 \times 2^0 + b_1 \times 2^1 + \ldots + b_l \times 2^l} \\
		& = & x^{b_0 \times 2^0} \times x^{b_1 \times 2^1} \times \ldots \times x^{b_l \times 2^l} \\
		& = & (x^{2^0})^{b_0} \times  (x^{2^1})^{b_1} \times \ldots \times (x^{2^l})^{b_l} \\
		& = & (x^{2^0})^{b_0} \times  (x^{2^1})^{b_1} \times \ldots \times (x^{2^l})^{b_l} 
	\end{array}
\]

Immediately, it can be noticed that a term contribute the the final result only when the corrensponding $b_i$ is set to $1$, because if it is $0$ then the term contribution is $1$. Additionally as $i$ increases $n$ get squared at each step.
The idea above can be used to implement a fast exponentiation iterative solution as shown in Listing \ref{list:exponentiation_fast_iterative}.
\lstinputlisting[language=c++, caption=Implementation of the iterative $log_2$ solution for the fast exponentiation problem.,label=list:exponentiation_fast_iterative]{sources/exponentiation/exponentiation_solution3.cpp}

Note how at each iteration the value of $x$ is squared and a bit of $k$ is discaterd and that only when the current bit is set to $1$ the term is calculated and included in the result. 

The complexity of this method is $O(log_2(n))$ because as can be clearly seen in the iterative solution in Listing \ref{list:exponentiation_fast_iterative}, the algorithm does not perform more iteration than the number of bits of the exponent $k$.
Thus at most $\floor{log(k)}$ squaring and multiplication are performed.

An alternative iterative implementation solution is shown in Listing \ref{list:exponentiation_fast_iterative}
\lstinputlisting[language=c++, caption=Alternative implementation of the iterative $O(log_2())$  solution for the fast exponentiation problem.,label=list:exponentiation_fast_iterative_alternative]{sources/exponentiation/exponentiation_solution4.cpp}

Note that in practice the $O(log_2())$  solution might not provide big speed advantage versus the linear one because of the data type (\texttt{unsigned}) used which limits the maximum value the exponent $k$ can take. That is not a problem because the above approach can be quickly extended or templatized for any type supponting the \texttt{operator*()},\texttt{operator>>()} and the \texttt{operator\&()}.

%\subsection{Fast $k^{th}$ Fibonacci calculation}
\chapter{\CC questionnaire}
   
    \begin{cppquestion}
        \label{cppquestion:q1}
        (Solution \ref{cppquestion:s1} at page \pageref{cppquestion:s1})
        \question \textbf{Which of the following statements is true for the code below:} 
\begin{lstlisting}[language=c++,numbers=none, caption={}]
struct X
{
  A a;
  B b;
  X() : a{}, b{}  {}
};

struct Y : X
{
  C c;
  D d;
  Y() : d{}, c{}  {}
  ~Y() {  }
};                    
\end{lstlisting} 
        \begin{choices}
            \choice Destruction of type \inline{Y} will call member destructors in the following order \inline{A::~A(), B::~B(), D::~D(), C::~C()}
            \choice Destruction of type \inline{Y} will call member destructors in the following order \inline{A::~A(), B::~B(), C::~C(), D::~D()}
            \choice Destruction of type \inline{Y} will call member destructors in the following order \inline{C::~C(), D::~D(), B::~B(), A::~A()}
            \choice Destruction of type \inline{Y} will call member destructors in the following order \inline{D::~D(), C::~C(), B::~B(), A::~A()}
            \choice Destruction of type \inline{Y} will only call destructors of classes \inline{C} and \inline{D} as the destructor of class \inline{X} is not called from \inline{Y::~Y()}
        \end{choices}
\end{cppquestion}
        



\begin{cppquestion}
    \label{cppquestion:q2}
    (Solution \ref{cppquestion:s2} at page \pageref{cppquestion:s2}) 
    \question \textbf{What will be the result of the code below?} 
    \begin{lstlisting}[language=c++,numbers=none, caption={}]
const char* ptr1 = "123456";
const char* ptr2 = "1234567";

if(ptr1 == ptr2)
    printf("same address");
else
    printf("different address");
        \end{lstlisting} 
    \begin{choices}
    \choice \quotes{different address}
    \choice \quotes{same address}
    \choice compilation error
    \choice runtime error
    \end{choices}
\end{cppquestion}


\begin{cppquestion}
    \label{cppquestion:q3}
    (Solution \ref{cppquestion:s3} at page \pageref{cppquestion:s3})
    \question \textbf{Which of the following pointer declarations will allow you to modify the value the pointer points to?} 
    
    \begin{choices}
     \choice \inline{int* ptr;}
     \choice \inline{const int* ptr;}
     \choice \inline{const int* const ptr;}
     \choice \inline{int const* ptr;}
     \choice \inline{int const* const ptr;}
     \choice \inline{int* const ptr;}
    \end{choices}
\end{cppquestion}

\begin{cppquestion}
    \label{cppquestion:q4}
    (Solution \ref{cppquestion:s4} at page \pageref{cppquestion:s4})
    \question \textbf{Which of the following statements are true for the code below?} 
    \begin{lstlisting}[language=c++,numbers=none, caption={}]
class Account
{
  mutable std::mutex m_;
  unsigned balance_;

 public:
  friend void transfer(Account& src, Account& dst, unsigned amount)
  {
    std::lock_guard<std::mutex> lck_src(src.m_);
    std::lock_guard<std::mutex> lck_dst(dst.m_);
    src.balance_ -= amount;
    dst.balance_ += amount;
  }
};
    \end{lstlisting} 

    \begin{choices}
     \choice Code is thread safe thanks to usage of \inline{lock_guards} that will prevent races and deadlocks.
     \choice Mutable \inline{std::mutex} makes this code not thread safe.
     \choice Code is not exception safe.
     \choice Code is not deadlock-free.
     \choice Using \inline{std::lock()} would be better to lock the mutexes.
    \end{choices}
\end{cppquestion}
 

\begin{cppquestion}
    \label{cppquestion:q5}
    (Solution \ref{cppquestion:s5} at page \pageref{cppquestion:s5})
    \question \textbf{What will be the address the \inline{ptr} points to after code execution? Assume that \inline{CHAR_BITS} equals 8.} 
\begin{lstlisting}[language=c++,numbers=none, caption={}]
int32_t* ptr = (int32_t*) 0x20000004;
ptr += 2;
\end{lstlisting} 
    \begin{choices}
     \choice \inline{0x20000000}
     \choice \inline{0x20000002}
     \choice \inline{0x20000004}
     \choice \inline{0x20000006}
     \choice \inline{0x20000008}
     \choice \inline{0x20000010}
     \choice \inline{0x2000000C}
    \end{choices}
\end{cppquestion}


\begin{cppquestion}
    \label{cppquestion:q6}
    (Solution \ref{cppquestion:s6} at page \pageref{cppquestion:s6})
    \question \textbf{What is the value of \inline{x} after the call to \inline{foo}?} 
\begin{lstlisting}[language=c++,numbers=none, caption={}]
uint8_t foo(uint8_t a)
{
    return ++a;
}

int main()
{
    uint8_t x = foo(std::numeric_limits<uint8_t>::max(););
    return 0;
}
\end{lstlisting}
    \begin{choices}
     \choice \inline{4294967295}
     \choice \inline{255}
     \choice \inline{0}
     \choice \inline{-1}
     \choice \inline{-2147483647 - 1}
     \choice \inline{129}
     \choice \inline{-128}
     \choice \inline{Compilation error}
     \choice \inline{Undefined behavior}
    \end{choices}
\end{cppquestion}



\begin{cppquestion}
    \label{cppquestion:q6}
    (Solution \ref{cppquestion:s7} at page \pageref{cppquestion:s7})
    \question \textbf{In which of the following statement is Return Value Optimization (RVO) guaranteed to happen? Assume C++-17 standard.} 
    \begin{choices}
     \choice \inline{FooClass a; FooClass b = a;}
     \choice \inline{auto a = CreateMyClass();}
     \choice \inline{FooClass a\{ CreateFooClass() \};}
     \choice \inline{FooClass a; a = CreateFooClass();}
    \end{choices}
\end{cppquestion}



\begin{cppquestion}
    \label{cppquestion:q8}
    (Solution \ref{cppquestion:s8} at page \pageref{cppquestion:s8})
    \question \textbf{What is the order in which destructors are called when \inline{d} gets out of scope?} 
\begin{lstlisting}[language=c++,numbers=none, caption={}]
    struct base0 { ~base0(); };
    struct base1 { ~base1(); };
    struct member0 { ~member0(); };
    struct member1 { ~member1(); };
    struct local0 { ~local0(); };
    struct local1 { ~local1(); };
    struct derived: base0, base1
    {
      member0 m0_;
      member1 m1_;
      ~derived()
      {
        local0 l0;
        local1 l1;
      }
    }
    void userCode()
    {
      derived d;
    }                 
\end{lstlisting} 
\end{cppquestion}


















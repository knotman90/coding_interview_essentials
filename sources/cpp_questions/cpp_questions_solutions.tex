\chapter{\CC questionnaire solutions}

\begin{cppanswer}
    \label{cppquestion:s1}
    (Question \ref{cppquestion:q1} at page \pageref{cppquestion:q1}) \hfill \\

\end{cppanswer}

\begin{cppanswer}
    \label{cppquestion:s2}
    (Question \ref{cppquestion:q2} at page \pageref{cppquestion:q2}) \hfill \\
    
\end{cppanswer}

\begin{cppanswer}
    \label{cppquestion:s3}
    (Question \ref{cppquestion:q3} at page \pageref{cppquestion:q3}) \hfill \\

\end{cppanswer}

\begin{cppanswer}
    \label{cppquestion:s4}
    (Question \ref{cppquestion:q4} at page \pageref{cppquestion:q4}) \hfill \\

\end{cppanswer}

\begin{cppanswer}
    \label{cppquestion:s5}
    (Question \ref{cppquestion:q5} at page \pageref{cppquestion:q5}) \hfill \\
    Correct answer is \textbf{G}. \\
    Each element pointed by \inline{ptr} is $4$ bytes and therefore we are going to add $8=2\times 4$ to the start address.

    On pointers we can perform the following operations:
    \begin{itemize}
        \item Increment \inline{++},\inline{+},\inline{+=}
        \item Decrement \inline{--},\inline{-},\inline{-=}
        \item Comparison \inline{==}
        \item Assignment \inline{=}
    \end{itemize}

    When performing an increment operation on a pointer \inline{ptr} or type T like: \inline{ptr += 2} the value of the address pointed by \inline{ptr} will be increased by \inline{2*sizeof(T)}. The decrement operation works in a similar fashion.
    This means that if we have a  pointer \inline{ptr} pointing to the first element in an array \inline{A} i.e. \inline{A[0]}, then, the following causes \inline{ptr} to point to the fifth element in \inline{A} i.e. \inline{A[4]}:   \inline{ptr = ptr + 4};

    It is important to notice that you can only use integer as right parameters for the increase and decrease operations on pointers and that you cannot add a pointer to another pointer.
\end{cppanswer}
%!TEX root = ../main.tex
%%%%%%%%%%%%%%%%%%%%%%%%%%%%%%%%%%
% Links:
%
% Difficulty:
% Companies: 
%%%%%%%%%%%%%%%%%%%%%%%%%%%%%%%%%%

\chapter{Best time to buy and sell stock}
\label{ch:buy_sell_stocks}
\section*{Introduction}
The problem discussed in this chapter is not particularly difficult as it is easily solvable in quadratic time using a brute-force algorithm. 
However, a more efficient solution is possible and, given that this is exactly the type of question for which interviewers expect fast and elegant solutions, it's worth taking the time to become familiar with the problem structure and the best approaches to solving it.  

\section{Problem statement}
\label{sec:buy_sell_stocks:statement1}
\begin{exercise}
You are given prices for a stock for a number $n$ of days. The prices are stored in an array $P$ of length $n$ where each cell $i$ of the array contains the price for the stock on the $i^{th}$ day. You are only permitted to perform \textbf{one} buy and \textbf{one} sell operations. What is the maximum profit you can achieve given the prices for the stock in $P$?

You have to perform the buy operation \textbf{before} the sell operation. You cannot buy the stock on the \nth{10} day and sell on the \nth{9}.

\begin{example}
	\hfill \\
	Given the array of prices for the stock is: $[7,1,5,3,6,4]$, the answer is $5$. You can buy on the \nth{2} day and sell on the \nth{5}.
\end{example}

\begin{example}
	\hfill \\
	Given the array of prices for the stock is: $[6,5,4,3,2,1]$, the answer is $0$. There is no way you can make a profit higher than $0$ i.e. not buying and not selling. 
\end{example}

\end{exercise}

\section{Clarification Questions}

\begin{QandA}
	\begin{questionitem} \begin{question} Can you perform the buy and sell operation on the same day?  \end{question}      
    \begin{answered}
		\textit{Yes, that is possible.}
	\end{answered} \end{questionitem}
\end{QandA}

\section{Discussion}
\label{buy_sell_stocks:sec:discussion}
A profit is achieved when a buy and sell transaction are performed with prices $p_b$ and $p_s$ respectively and $p_b \leq p_s$. In other words, our goal is to buy at a lower price than we sell. The maximum profit is obtained whenever the spread between those two prices is maximum i.e. $\max_{}{(p_s - p_b)}$

\subsection{Brute-force}
\label{buy_sell_stocks:sec:bruteforce}
The brute force approach is very straightforward as the only thing we need to do is apply the definition of maximum profit we discussed earlier. For all pairs of ordered index $i \leq j$ we can calculate $P_i - P_j$ and return the maximum among all those profit values. Listing \ref{list:buy_sell_stocks:bruteforce} shows an implementation of this approach. Note that a profit of $0$ is always possible by either not performing any transaction or simply performing the buy and sell on the same day. Thus $j = i+1$, because it is pointless to calculate the profit for the same day as we know already it will always be $0$. For this reason we also limit the buy operation to the day before ($i< n-1$) the last, because if we want to have any chance of making a profit we need to at least have one day left after the buy to perform the sell operation. 

\lstinputlisting[language=c++, caption={Brute force $O(n^2)$ solution to the problem of buying and selling stock.},label=list:buy_sell_stocks:bruteforce]{sources/buy_sell_stocks/buy_sell_stocks_solution1.cpp}

\subsection{Linear time solution}
\label{buy_sell_stocks:sec:linear}
The solution above can be improved if we look at the problem from slightly different angle. The idea is that we can process the array from the last day to the first and, for each of the days, calculate the \textbf{best} profit to be made by selling on any of the days already processed (which occurs later in time).

We keep a variable $b$ with the maximum price seen so far which is initially $-\infty$. The algorithm starts from day $n$ and for each day checks whether buying that day and selling at the price $b$ (the highest price seen so far) would improve the profit found thus far. This approach is correct because the maximum profit happens when the spread between sell and buy price is maximum.
The implementation of the idea above is shown in Listing \ref{list:buy_sell_stocks:linear}.

\lstinputlisting[language=c++, caption={Dynamic programming linear time, constant space solution to the problem of buying and selling stock.},label=list:buy_sell_stocks:linear]{sources/buy_sell_stocks/buy_sell_stocks_solution2.cpp}


% Variation 2
\section{Common Variations - Multiple Transactions}
\label{sec:buy_sell_stocks:multiple_transaction}

\subsection{Problem statement}
\begin{exercise}
	You are given an integer array $P$ where $P[i]$ contains the price of a given stock on the $i^{th}$ day.

	On each day, you may decide to buy and/or sell the stock. 
	You can only hold at most one share of the stock at any given time.
	However, you might engage in multiple transaction over the course of time i.e. you repeat the process of buying a share then sell it after a while (also the next day) multiple times.
	
	Write a function that given $P$ returns the maximum profit achievable.

	Notice that you may not engage in multiple transactions at the same time i.e., you must sell the stock before you buy it again.
	\begin{example}
	\label{ex:buy_sell_stocks_2:exmaple1}
		\hfill \\
		Given the array of prices for the stock is: $[7,1,5,3,6,4]$, the answer is $7$. 
		You can buy on the \nth{2} day and sell on the \nth{3} and then engage on a second transaction where you buy on the \nth{4} day and sell on the \nth{5}.
	\end{example}

\end{exercise}


\section{Discussion}
\label{buy_sell_stocks_2:sec:discussion}
This might seems like an harder problem at first than the version presented in Section \ref{sec:buy_sell_stocks:statement1} but in reality as we will see in Section \ref{buy_sell_stocks_2:sec:linear} its solution is actually easier.

\section{Brute force solution}
\label{buy_sell_stocks_2:sec:bruteforce}
As usual we start our discussion by quickly presenting the brute force solution. In this case this means trying all possible sets of transactions (a valid pair of buy and sell operation not overlapping with any other transaction). We can try all possible sets by using recursion cleverly. However this approach will not take us far because the number of possible sets of transaction grows exponentially.
We are showing this approach in Listing \ref{list:buy_sell_stocks_2:bruteforce}  only because we think its implementation can be somehow instructive.

\lstinputlisting[language=c++, caption={Bruteforce exponential solution to the problem of buying and selling stock with no limits on the number of transactions.},label=list:buy_sell_stocks_2:bruteforce]{sources/buy_sell_stocks/buy_sell_stocks_2/buy_sell_stocks_2_solution1.cpp}

\section{Linear time solution}
\label{buy_sell_stocks_2:sec:linear}


The idea is simple and it is clearer once we look at prices plotted on a graph. As you can see in Figure \ref{}, the data is made of peaks and valleys (unless the data is fully increasing or decreasing). Those are the point of interests because if we buy at valleys and sell at peaks we are able to obtains the maximum profit. 
One can simply loop thought the array and identify those peaks and valleys and calculate the total profit as the sum of the profits along those point of interests. 
For instance w.r.t. the example \ref{ex:buy_sell_stocks_2:exmaple1} there are two pairs  valley-peak happening at days $2$ and $3$ and days $4$ and $5$, respectively. 
But, what is a valley and/or a peak exactly?
A day $i$ is a valley if $P_i < P_{i-1}$ and $P_i > P_{i+1}$
while is a peak if $P_i > P_{i-1}$ and $P_i < P_{i+1}$.
So all it is needed is to identify those pairs of valleys and peaks and we are done. 

But do we really need to find peaks and valleys? The answer is not as all it is necessary is to make sure we cash at \textbf{all} opportunities we have i.e. in all those cases where we can buy at a lower price we sell. Thus we can process days two at the time and, since there is no limit on the number of transactions, simply buy and sell whenever the spread between buy and sell price is convenient. 

The idea above can be implemented as shown in Listing \ref{list:buy_sell_stocks_2:linear}. 


\lstinputlisting[language=c++, caption={$O(n)$ time and $O(1)$ space solution to the problem of buying and selling stock with no limits on the number of transactions.},label=list:buy_sell_stocks_2:linear]{sources/buy_sell_stocks/buy_sell_stocks_2/buy_sell_stocks_2_solution2.cpp}


% Variation 3
\section{Common Variations - Best profit with exactly two transactions}
\label{sec:buy_sell_stocks3:intro}

\subsection{Problem statement}
\begin{exercise}
    You are given an array $P$ where $P[i]$ is the price of a given stock on the $i^th$ day.
    Write a function that given $P$ finds the max profit you can achieve by performing at most two transactions.
        
    Notice that you may not engage in multiple transactions simultaneously, meaning that you must sell the stock before you buy it again.
    
     
    \begin{example}
    \label{ex:buy_sell_stocks_3:exmaple1}
        \hfill \\
        Given $P=\{3,3,5,0,0,3,1,4\}$ the function outputs $6$. 
        You can buy on the \nth{4} day and sell on the \nth{6}. This transaction yelds a profit of $3$.
        You can then perform another transaction with buy and sell dates being the \nth{7} and \nth{8} days, respectively, for a total profit of $6$.
    \end{example}

    \begin{example}
        \label{ex:buy_sell_stocks_3:exmaple2}
            \hfill \\
            Given $P=\{1,2,3,4,5\}$ the function outputs $4$. 
            The best strategy here is to perform a single transaction where you buy the first and sell the last day. Notice that you can achieve the same total profit by also performing two transactions.
    \end{example}

    \begin{example}
        \label{ex:buy_sell_stocks_3:exmaple3}
            \hfill \\
            Given $P=\{7,6,4,3,1\}$ the function outputs $0$. 
            It is best in this case not to trade this stock at all, as all possible transaction leads to a loss.
    \end{example}
\end{exercise}


\section{Discussion}
\label{buy_sell_stocks_3:sec:discussion}
This variation might seems at first easier than the one presented in Section \ref{sec:buy_sell_stocks:multiple_transaction}. 
However, not having a limit on the number of transactions you can make allows us to adopt the strategy in which we make 
all perform all possible transactions that result in a profit. 
When we have a constraint on the maximum number of transactions we can make, we are suddenly forced to discard some and keep only the best (two in this specific case, but the same reasoning will apply to the variation in Section \ref{buy_sell_stocks4:sec:discussion}).
This makes solving this problem significantly harder.


\section{DP - Linear time solution}
\label{buy_sell_stocks_3:sec:linear}
A possible way of solving this problem is by noticing that if you complete the first transaction at day $i$ (day of the sale) then
you must have made the buying part of the transaction when the price was at its minimum between day $0$ and $i-1$. Say we made a profit of $t^i_1$ ($t^x_y$ represent the profit of the $y^{th}$ transaction completed on day $x$).
At this point, we still have one more transaction we can make from day $i+1$ to $n-1$. Say that the profit of the best second transaction is $t^i_2$ then you end up with a total profit of $t^i_1+t^i_2$. If we have a way of quickly determining $t_2$ for each $i$ then this problem can be solved relatively easy as we need to return the maximum among $t^i_1+t^i_2$ for all days.

What is exactly $t^i_2$? It is the maximum profit we can make by making a \textbf{single} between days $i+1$ and $n-1$. 
Luckly we can calculate $t^i_2$ for all $0 \leq i < n$ using DP. We know that the value of $t^{n-1}_2$ is zero. For every other day $i$ 
we can calculate the answer to $t^i_2$ if we know the answer for $t^{i+1}_2$ because  the value of $t^i_2$ can either be:
\begin{itemize}
    \item $t^{i+1}_2$ 
    \item $M(j > i)-P[i]$ where $M(j)$ is the highest price of the stock for some day after $i$.
\end{itemize}
The reasoning behind this is that the best single transaction you can make with the prices from day $i$ to $n-1$ is either the one you make by buying exactly at day $i$ (and therefore selling at the highest price later) or a transaction you make with the prices for the days ahead i.e. from $i+1$ to $n-1$ for which, crucially, we already have an answer. 


Equation \ref{eq:buy_sell_stocks_3:Bdpformula} formalizes this idea where the final answer is the maximum among all values of $T(i)$.
$m(i)$ and $M(i)$ contain the information about the smallest and largest element to the left and to the right of index $i$, respectively.
$B(i)$ instead carries the value of the best single transaction that you can make with any values to the left of index $i$ (inclusive).
In other words, what Equation \ref{eq:buy_sell_stocks_3:Bdpformula} states is that the value of the final answer is the value of the best transaction you can make by selling at exactly index $i$\footnote{For which you must have bought at the minimum between $0$ and $i-1$ to make the best profit.} 
plus the value of the best transaction you can make with any of the prices to the right of $i$.
If you take the maximum among all indices, then it is clear that this quantity is indeed the value you can achieve by performing two transactions at most (when $i=0$ or $i=n-1$ we are in practice making only a single transaction.).

\begin{equation}
    \begin{cases}
        B(|I|-1) = 0 \\
        B(i) = max(B(i+1), M(i)-P(i)) \\
        M(i) = max(P(i), P(i+2), \ldots, P(n-1)) \\
        m(i) = min(P(0), P(1), \ldots, P(i)) \\
        T(i) = (P[i]-m(i)) + B(i)
     \end{cases}
    \label{eq:buy_sell_stocks_3:Bdpformula}
\end{equation}

Listing \ref{list:buy_sell_stocks_3:linear} shows an implementation of this idea.

\lstinputlisting[language=c++, caption={$O(n)$ time and space DP solution.},label=list:buy_sell_stocks_3:linear]{sources/buy_sell_stocks/buy_sell_stocks_3/buy_sell_stocks_3_solution1.cpp}

The code works by calculating the values of the best transaction we can make with the values to the right of each index $i$ and stores this info in an array of size $n$ (this is $B$ in Equation \ref{eq:buy_sell_stocks_3:Bdpformula}).
The code then proceeds in calculating the answer by looping over all days and maintaining a variable \inline{min_left} which contains the minimum element seen so far: this value is useful in calculating the profit for the first transaction we can make by selling at index $i$.
The loop goal is to calculate $T(i)$ of Equation \ref{eq:buy_sell_stocks_3:Bdpformula} and remember the maximum value ever calculated (in the variable \inline{ans}which is eventually returned.

Listing \ref{list:buy_sell_stocks_3:linear} has linear time and space complexity.


\subsection{Linear time and constant space (draft)}
Suppose you make some profit p1 by doing your first transaction in the stock market. Now you are excited to purchase another stock to earn more profit. Suppose the price of the second stock you aim to buy is x. Now, for you, the net effective price that you are spending from your pocket for this stock will be x-p1, because you already have p1 bucks in your hand. Now, if you sell the second stock at price $y$ your net profit p2 will be p2 = y - (x-p1). You have to do nothing but maximize this profit p2. Here's the code:
\begin{lstlisting}[language=c++,numbers=none, caption={}]
          
    int maxProfit(vector<int>& prices) {
        if(!prices.size())
        return 0;
        int buy1    = INT_MAX;
        int profit1 = INT_MIN;
        int buy2    = INT_MAX;
        int profit2 = INT_MIN;
        for(int i = 0; i < prices.size(); i++){
            buy1    = min(buy1, prices[i]);
            profit1 = max(profit1, prices[i] - buy1);
            buy2    = min(buy2, prices[i] - profit1);
            profit2 = max(profit2, prices[i] - buy2);
            }
            return profit2;
            }
        \end{lstlisting} 

% Variation 4
\section{Variation - Best profit with at most $k$ transactions}
\label{sec:buy_sell_stocks4:intro}

\subsection{Problem statement}
\begin{exercise}
    You are given an integer $K$ and an array $P$ where $P[i]$ is the price of a given stock on the $i^th$ day.
    Write a function that given $P$ and $K$. finds the maximum profit you can achieve by performing at most $K$ transactions.
    
    Notice that you may not engage in multiple transactions simultaneously, meaning that you must sell the stock before you buy it again.
         

    \begin{example}
    \label{ex:buy_sell_stocks4:exmaple1}
        \hfill \\
        Given $K=2$ and $P=\{2,4,1\}$ the function outputs $2$. 
        You can buy on the \nth{1} day and sell on the \nth{2}. This transaction yelds a profit of $2$. Notice that you only use one of the two transactions allowed.
    \end{example}

    \begin{example}
        \label{ex:buy_sell_stocks4:exmaple2}
            \hfill \\
            Given $K=2$ and $P=\{3,2,6,5,0,3\}$ the function outputs $7$. 
            You can buy on the \nth{2} day and sell on the \nth{3}. You can then make another transaction where you buy on the \nth{5} and sell on the \nth{6} day for a total profit of $4+3=7$.
    \end{example}

    \begin{example}
        \label{ex:buy_sell_stocks4:exmaple3}
            \hfill \\
            Given $K=4$ and $P=\{4
            [3,2,6,5,0,3,3,8,2,3,5,5,9]\}$ the function outputs $19$. 
            Notice that the function would output $19$ even when $K=3$.
    \end{example}
\end{exercise}


\section{Discussion}
\label{buy_sell_stocks4:sec:discussion}
The variation discussed is a generalization of the one discussed in Section \ref{sec:buy_sell_stocks3:intro} and now we are allowed to make up to $K$ transaction where $K$ is given to us as a parameter. Clearly when $K=2$ this variation is equivalent to the one in Section \ref{sec:buy_sell_stocks3:intro}. However not knowing precisely how many transactions at most we can make complicates things a bit (but not too much).

Let's start with a simple observation: if $K > \frac{|P|}{2}$ then there is basically no limit on the number of transactions we can make and this problem can be solved with the same approach shown in Listing \ref{list:buy_sell_stocks_2:linear}.
Despite being interesting is not really key to solving this problem in its generality albeit it might in practice speed up the actual runtime for these specific cases.


\section{$O(n^2K)$ time and space}
\label{buy_sell_stocks4:sec:linerartime1}
We are going to make an attempt to write a recursive formula similar to Equation \ref{eq:buy_sell_stocks_3:Bdpformula} for the variation number $3$ of this problem, that describes the answer in terms of subproblems.
Let $DP(i,j)$ be the maximum profit possible by only considering prices up to index $i$ and by using at most $j \leq K$ transactions.
Clearly $DP(0,j)=DP(i,0)=0$, as it is impossible to make a transaction only on the first day as well as when you can make at most $0$ transactions.
When calculating a $DP(i,j)$ for the general case we should consider that it can be equivalent to either:
\begin{itemize}
    \item $DP(i-1,k)$ the value of the maximum profit you can make up to index $i-1$ with at most $k$ transaction (we are basically saying we ignore the price at index $i$);
    \item or the maximum profit we can make by performing a transaction that ends in a sell exactly at index $i$. To calculate this value we need to find the best place to perform the buy-side of the transaction which can take place at every index $l < i$. This can be calculate with the following formula: $max DP(l,j-1) + (P[i]-P[j])$.
\end{itemize}
We can rewrite the general case $DP(i,j)$ in Equation \ref{eq:buy_sell_stocks_4:Bdpformula} as 
$$DP(i,j) = max(DP(i-1,k), P[i] + max(DP(l,j-1)+-P[j])) \: \: \forall 0 \leq l < i$$ as $P[i]$ is constant in the inner max expression ($l$ is the only variable there).

Equation \ref{eq:buy_sell_stocks_4:Bdpformula} summarises what we have discussed so far for $DP$.
\begin{equation}
    \begin{cases}
        DP(0,j) = 0 \\
        DP(i,0) = 0 \\
        DP(i,j) = max\Big\{DP(i-1,k), \; P[i] + \underbrace{max\big\{DP(l,j-1)-P[l]\big\}}_{\forall \: \: 0 \leq l < i}\Big\}
     \end{cases}
    \label{eq:buy_sell_stocks_4:Bdpformula}
\end{equation}
We could already proceed in turning Equation \ref{eq:buy_sell_stocks_4:Bdpformula} naively into code and we would obtain a working solution with $O(nK)$ space and $O(n^2K)$ time complexities, respectively as shown in Listing \ref{list:buy_sell_stocks_4:linear0}.


\lstinputlisting[language=c++, caption={$O(n^2K)$ time and $O(nK)$ space DP solution.},label=list:buy_sell_stocks_4:linear0]{sources/buy_sell_stocks/buy_sell_stocks_4/buy_sell_stocks_4_solution0.cpp}

\section{$O(nK)$ time and space}
\label{buy_sell_stocks4:sec:linerartime1}
However, the solution discussed above can be improved dramatically by noticing that we do not need to calculate the value of the innermost max in the third case of Equation \ref{eq:buy_sell_stocks_4:Bdpformula} as shown in Listing \ref{list:buy_sell_stocks_4:linear2}.

\lstinputlisting[language=c++, caption={$O(nK)$ time and $O(nK)$ space DP solution.},label=list:buy_sell_stocks_4:linear2]{sources/buy_sell_stocks/buy_sell_stocks_4/buy_sell_stocks_4_solution2.cpp}

The important and most challenging part of this solution is to make sure that the quantity $L=\underbrace{max\big\{DP(l,j-1)-P[j]\big\}}_{\forall \: \: 0 \leq l < i}$ is calculated as we iterate incrementally over all values of $i$. 
To understand why let's look at the particular values of $L$ for some incremental values of $i$:
\begin{itemize}
    \item if $i=1$ then $L_1=max\big\{DP(0,j-1)-P[0]\big\}$
    \item if $i=2$ then $L_2=max\big\{DP(0,j-1)-P[0],DP(1,j-1)-P[1]\big\}$; but crucially $DP(0,j-1)-P[0] = L_1$
    \item if $i=3$ then $L_3=max\big\{DP(0,j-1)-P[0],DP(1,j-1)-P[1],DP(1,j-1)-P[1]\big\}$; thanks to the fact that $l_2 = max \{DP(0,j-1)-P[0],DP(1,j-1)-P[1]\}$ we can simply this expression as  $L_3=max\big\{L_2,DP(1,j-1)-P[1]\big\}$.
     crucially $DP(0,j-1)-P[0] = L_1$
     \item \ldots
     \item In general $L_l = max\big\{L_{l-1}, DP(l,j-1)-P[l]\big\}$
\end{itemize}

This approach allows us to avoid the loop over $l$ each time we calculate an entry in $DP(i,j)$ and to lower the time complexity $O(nK)$: a good improvement w.r.t. the previous solutions!


\section{$O(|P|K)$ time  and $O(|P|)$ and space}
If we pay attention to either the main loop in Listing \ref{list:buy_sell_stocks_4:linear0} or Equation \ref{eq:buy_sell_stocks_4:Bdpformula} we notice that in the innermost loop for $i$ we never ever reference in \inline{DP} any value of $k$ that is lower than $k-1$. This observation opens for the an space optimization opportunity where the size of DP goes down from  $n\times K$ to $n\times 2$. 
We can use one column of \inline{DP} to refer to the current value of $k$ and the other to $k-1$ and after the innermost loop is completed we can swap these two columns. For example at the first iteration of the outermost loop ($k=1$), \inline{DP[_][1]} refers to the values for $k=1$ and \inline{DP[_][0]} to the values for $k-1=0$. 
When the innermost loop ends  and the outermost starts again, the two columns are swapped and therefore \inline{DP[_][0]} refers to $k=2$ while  \inline{DP[_][1]} to $k-1=1$. This process continues until both loops end.

Listing \ref{list:buy_sell_stocks_4:linear3} shows an implementation of this idea.

\lstinputlisting[language=c++, caption={$O(nK)$ time and $O(n)$ space DP solution.},label=list:buy_sell_stocks_4:linear3]{sources/buy_sell_stocks/buy_sell_stocks_4/buy_sell_stocks_4_solution1.cpp}

The code is extremely similar to Listing \ref{list:buy_sell_stocks_4:linear2}, with the only difference being the size of \inline{DP} is now $O(n)$ and we use two variables \inline{curr_k} and \inline{prec_k} to keep track of the column assigned to the \quotes{current} and \quotes{previous} values of $k$. 
Notice how at the end of each innermost loop, the two columns are swapped by simply swapping around the values of \inline{curr_k} and \inline{prec_k}.

%!TEX root = ../main.tex
%%%%%%%%%%%%%%%%%%%%%%%%%%%%%%%%%%
% Links:
%
% Difficulty: Easy Companies: 
%%%%%%%%%%%%%%%%%%%%%%%%%%%%%%%%%%

\chapterimage{header}

\chapter{Unique Elements in a collection}
\label{ch:unique_elements}
\section*{Introduction}
I have to admit I was not really sure it was the case to write about a problem as easy as this one,
but I have seen this problem asked enough that I think it is valuable to discuss how to attack and
code it well during an interview. The problem is about checking whether a string does not contain
duplicate characters and as we shall see it can be coded in a handful of lines. It is quite
important that a candidate asks the right questions, as the interview might be really trying to ask
you an harder problem instead with some hidden constraints (for instance what if the charset is not
ASCII?), and that the implementation is impeccable and efficient. As with most of the problems in
this book, there is usually a number of solutions ranging from easy and straightforward (and usually
suboptimal) to more complex and faster. Looking at the various solution proposed here 
it seems like this problem is no different, but in reality it does not really matter which solution among the ones presented here
you actually use during a real interview because all of them have constant time and space complexity.


\section{Problem statement}
Given a string $s$ of length $n$, return \textit{true} if it does \textbf{not} contains duplicate characters, \textit{false} otherwise. 

\begin{example}
\hfill
	\begin{itemize}
		\item [-] $s=$"graph" $ \longrightarrow$ \textit{true}
		\item [-] $s=$"tree" $ \longrightarrow$ \textit{false}
		\item [-] $s=$"Einstein" $ \longrightarrow$ \textit{false}
	\end{itemize}
	
\end{example}

\section{Clarification Questions}

\begin{QandA}
	\item What is the maximum size of the input?
	\begin{answered}
		\textit{The maximum length for the input string is $10^6$.}
	\end{answered}
	
	\item Are all character upper or lower case?
	\begin{answered}
		\textit{No, both upper and lower case might be present.}
	\end{answered}

	\item Is the function case sensitive?
	\begin{answered}
		\textit{Yes.}
	\end{answered}

	\item Can I assume characters only alphanumeric characters are present in the input?
	\begin{answered}
		\textit{Yes. Upper and lower case latin letters and numbers only.}
	\end{answered}
\end{QandA}

\section{Discussion}
Being this problem so popular, when asked, the interviewer is expecting you to come up
with a good solution in a relatively short time window. 
For this reason the obvious solution (shown in Section \ref{sec:unique_character:bruteforce})
should be immediately put on the table despite the fact it not worth it to spend time to actually code and test it.

\subsection{Brute Force}
\ref{sec:unique_character:bruteforce}
One of the possible brute-force solutions to this problem works by looping over each character of the input string $s_i$ once,
and for each of them checking if it is present in any of the subsequent position of $s$. 
In other word we check whether the following is true: $\exists \; j $ s.t.  $s_j=s_i \; j>i$.
A C++ implementation of this idea is shown in Listing \ref{list:unique_elements_brute_force1}. 
The code uses  two simple nested loops to perform the checks where the internal loop has a loop variable $j$ starting one
position after $i$ as there is no need to check the all the values from the index $0$ as that work has been already done during the previous iterations. 

\begin{minipage}{\linewidth}
	\lstinputlisting[language=c++, caption="C++ solution for determining all characters in a string are unique.",label=list:unique_elements_brute_force1]{sources/unique_elements/unique_elements_brute_force.cpp}
\end{minipage}

As a stylistic improvements to the code in Listing \ref{list:unique_elements_brute_force1}, Listing
\ref{list:unique_elements_brute_force2}  uses the C++ standard library function \texttt{find} to
substitute the internal loop to search for a duplicate of the character $s_i$. 
This not only makes the code shorter and cleaner, but also shows to the interviewer that you are
able to use the standard library and do not reinvent the wheel everytime an ubiquitous operation, like \texttt{find}, happends to be needed.

\begin{minipage}{\linewidth}
	\lstinputlisting[language=c++, caption="C++ solution for determining all characters in a string are unique using \texttt{std::find}",label=list:unique_elements_brute_force2]{sources/unique_elements/unique_elements_brute_force_std.cpp}
\end{minipage}

\subsection{Trading space for time}
In Listing \ref{list:unique_elements_brute_force1} the internal loop is doing the hard work of
searching for a duplicate of the character at index $i$. 
We can trade space for time and reduce the complexity of the search of a duplicate for a single character down to $O(1)$.
The idea is that when the search for a duplicate is over for character $s_i$, then $s_i$ is inserted into a set.
The set keeps track of all the characters seen so far. The internal loop can be modified to simply be a lookup into the
map as shown in Listing \ref{list:unique_elements_brute_force_map}. This way for each character of the string only constant work is done making this approach having (apparently) linear time complexity.

\begin{minipage}{\linewidth}
	\lstinputlisting[language=c++, caption="C++ solution for determining all characters in a string are unique in $O(n)$ using an hashset.",label=list:unique_elements_brute_force_map]{sources/unique_elements/unique_elements_brute_force_map.cpp}
\end{minipage}

This approach seems to effectively lower the time complexity down to linear, but at the cost of some
space. But how much space exactly? The intuition would say $n$ because that is the size of the
input string. But the string is made of characters from an alphabet $\Sigma$ which has a very
limited size, at most $128$ (size of the ASCII table) elements. The insert instruction will not be
executed more than $|\Sigma|$ times. Because of this the space complexity of this solution if
$O(1)$. 

The previous argument can be expanded further more with the following idea: \textbf{Every string
with more than $|\Sigma|$ character contains at least one duplicate}(follows from the pigeon
principle\footnote{The pigeonhole principle states that if n items are put into m containers, with
$n > m$, then at least one container must contain more than one item.}). The longest string with
only unique characters is one of the permutation of "abcde\ldots zABCD \ldots Z123 \ldots 9". Thus
the solution using the hashset has complexity of $O(1)$ because in the worst case after $|\Sigma|$
lookups the next one will be positive. For this reason the checks can be limited to the first
$|\Sigma|$ character. Note that this observation suddenly makes the brute force approach also $O(1)$
if $i$ and $j$ in Listing \ref{list:unique_elements_brute_force1} are forced to stay below
$|\Sigma|$.

Armed with these new arguments, the solution we suggest to present during the interview only uses a
stack allocated array of bool of size $|\Sigma|$ storing the information regarding the presence of a
character in the characters examined so far. If at any time the currenclty examined character has
been already seen, then there is a duplicate. See Listing
\ref{list:unique_elements_brute_force_final} for a possible implementation of this idea.

\begin{minipage}{\linewidth}
	\lstinputlisting[language=c++, caption="C++ solution for determining all characters in a string are unique in $O(n)$ using an hashset.",label=list:unique_elements_brute_force_final]{sources/unique_elements/unique_elements_brute_force_final.cpp}
\end{minipage}

\section{Common variations}

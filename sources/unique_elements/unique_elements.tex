%!TEX root = ../main.tex
%%%%%%%%%%%%%%%%%%%%%%%%%%%%%%%%%%
% Links:
%
% Difficulty: Easy Companies: 
%%%%%%%%%%%%%%%%%%%%%%%%%%%%%%%%%%

\chapterimage{header}

\chapter{Unique Elements in a collection}
\label{ch:unique_elements}
\section*{Introduction}
I have to admit I was not really sure it was the case to write about a problem as easy as this one,
but I have seen it asked enough to think it is valuable to discuss how to attack and
code it well during an interview. 
The problem is about checking whether a string does not contain
duplicate characters and as we shall see it can be coded in a handful of lines. 
It is quite important that a candidate asks the right questions, 
as the interview might be really trying to ask
you an harder problem instead with some hidden constraints (for instance what if the charset is not
ASCII or groups of characters of the input string map to a certain element of a given alphabet),
and that the implementation is impeccable and efficient. 
As with most of the problems in
this book, there is usually a number of solutions ranging from easy and straightforward (and usually
suboptimal) to more complex and faster. This problem is no exeption.  We are going to have a look at 
a bruteforce approach in Section \ref{} while Section \ref{} will discuss a much faster solution using an approach 
that we believe is the one that should be used during a real interview.


\section{Problem statement}
Given a string $s$ of length $n$, return \textit{true} if it does \textbf{not} contains duplicate characters, \textit{false} otherwise. 

\begin{example}
\hfill
	\begin{itemize}
		\item [-] $s=$"graph" $ \longrightarrow$ \textit{true}
		\item [-] $s=$"tree" $ \longrightarrow$ \textit{false}
		\item [-] $s=$"Einstein" $ \longrightarrow$ \textit{false}
	\end{itemize}
	
\end{example}

\section{Clarification Questions}

\begin{QandA}
	\item What is the maximum size of the input?
	\begin{answered}
		\textit{The maximum length for the input string is $10^6$.}
	\end{answered}
	
	\item Are all character upper or lower case?
	\begin{answered}
		\textit{No, both upper and lower case might be present.}
	\end{answered}

	\item Is the function case sensitive?
	\begin{answered}
		\textit{Yes.}
	\end{answered}

	\item Can I assume characters only alphanumeric characters are present in the input?
	\begin{answered}
		\textit{Yes. ASCII upper and lower case latin letters and numbers only.}
	\end{answered}
\end{QandA}

\section{Discussion}
Being this problem so popular and simple, the interviewer is expecting you to come up
with a good solution in a relatively short time window. 
For this reason at least the obvious solution (shown in Section \ref{sec:unique_character:bruteforce})
should be immediately put on the table despite the fact it is probably not the most elegant and concise.

\subsection{Brute Force}
\ref{sec:unique_character:bruteforce}
One of the possible brute-force solutions to this problem works by looping
over each character of the input string $s_i$ once,
and for each of them checking if it is present in any of the subsequent position of $s$. 
In other word we check whether the following is true: $\exists \; j $ s.t.  $s_j=s_i \; j>i$.
A C++ implementation of this idea is shown in Listing \ref{list:unique_elements_brute_force1}. 
The code uses two simple nested loops to perform the checks where the internal loop has a
loop variable $j$ starting one position after $i$ as there is no need to check
all the values in the range $[0,i]$ as that work has been already performed during previous iterations. 

This solution works in $O(|s|^2)$ time and $O(1)$ space. 
It can be however, turned into a much faster solution (spoiler $O(1)$ time) using an idea that is at the core of the solution presented index
Section \ref{sec:unique_elements:constanttime}.

\begin{minipage}{\linewidth}
	\lstinputlisting[language=c++, caption="C++ solution for determining all characters in a string are unique.",label=list:unique_elements_brute_force1]{sources/unique_elements/unique_elements_brute_force.cpp}
\end{minipage}

Because what the inner loop is really doing is searching for the char $s[i]$ in a substring or $s$ string,  Listing
code shown in Listing \ref{list:unique_elements_brute_force1} can be made more expressive 
by substituting that loop with a call to the standard library \inline{find} function. Listing 
\ref{list:unique_elements_brute_force2} shows such version of the bruteforce solution which contains 
shorter and cleaner code, but also shows to the interviewer that you are
able to use the standard library and do not need to 
reinvent the wheel everytime an ubiquitous operation, like \texttt{find}, happends to be needed.
The complexity however remains unchanged.
\begin{minipage}{\linewidth}
	\lstinputlisting[language=c++, caption="C++ solution for determining all characters in a string are unique using \texttt{std::find}",label=list:unique_elements_brute_force2]{sources/unique_elements/unique_elements_brute_force_std.cpp}
\end{minipage}

\subsection{Linear Time}
\ref{sec:unique_elements:constanttime}
In Listing \ref{list:unique_elements_brute_force1} the internal loop is doing the hard work of
searching for a duplicate of the character at index $i$. By trading space for time we can improve the cost of that loop so that it runs in constant time.
The idea is that when we process the char $s[i]i$ all we need to do is check whether we have seen this character before.
However in order to be able to do that, everytime a character $i$ is processed we need to remember about it by for instance inserting it 
into an hashset.
If we do that then the search for the duplicate of $s[i]$ simply becomes a cheap hashset find query.
This idea is shown in Listing \ref{list:unique_elements_brute_force_map} which uses \inline{std::unordere_map} as hashset.
\begin{minipage}{\linewidth}
	\lstinputlisting[language=c++, caption="C++ solution for determining all characters in a string are unique in $O(n)$ using an hashset.",label=list:unique_elements_brute_force_map]{sources/unique_elements/unique_elements_brute_force_map.cpp}
\end{minipage}

This approach seems to effectively lower the time complexity down to linear (the outer loop apparently runs $O(|s|)$ times and the search query in $L$ costs $O(1)$\footnote{On average}),
but at the cost of some space. But how much space exactly? The intuition would say $n$ because that is the size of the
input string. But the string is made of characters from an alphabet $\Sigma$ which has a fixed (and rather small) size i.e. $128$ (the size of the ASCII table)
elements. 
The insert instruction will therefore not be executed more than $|\Sigma|$ times. Because of this the space complexity of this solution if
$O(1)$. 

\subsection{Constant time}
The previous argument can be expanded further more with the following idea: \textbf{every string
with more than $|\Sigma|$ character contains at least one duplicate}.
The key idea supporting this argument is that there is only a finite and fixed
number of unique characters in the alphabet of $|s|$ and therefore strings longer than the size of such charset must have duplicates. 
This is a consequence of the piegeonhole principle \footnote{\url{https://en.wikipedia.org/wiki/Pigeonhole_principle}}.
The longest string with only unique characters is one of the permutations of "abcde\ldots zABCD \ldots Z123 \ldots 9". 
Therefore, if $|s| > |\Sigma| = 128$ ($128$ is the size of the ASCII character set), we can immediately say that there 
is at least a duplicate in $s$ and return \inline{false}.
Thus the solution using the hashset has complexity of $O(1)$ because in the worst case after $|\Sigma|$
lookups the next one will be positive.
For this reason the checks can be limited to the first $|\Sigma|$ character. 
Note that this observation suddenly makes the brute force approach also $O(1)$
if $i$ and $j$ in Listing \ref{list:unique_elements_brute_force1} are forced to stay below
$|\Sigma|$.

Armed with these new arguments, the solution we suggest to present during the interview only uses a
stack allocated array of bool of size $|\Sigma|$ storing the information regarding the presence of a
character in the characters examined so far. If at any time the currenclty examined character has
been already seen, then there is a duplicate. See Listing
\ref{list:unique_elements_brute_force_final} for a possible implementation of this idea.

\begin{minipage}{\linewidth}
	\lstinputlisting[language=c++, caption="C++ solution for determining all characters in a string are unique in $O(n)$ using an hashset.",label=list:unique_elements_brute_force_final]{sources/unique_elements/unique_elements_brute_force_final.cpp}
\end{minipage}

\section{Conclusion}
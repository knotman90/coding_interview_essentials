%!TEX root = ../main.tex
%%%%%%%%%%%%%%%%%%%%%%%%%%%%%%%%%%
% Links:
%
% Difficulty: Easy Companies: 
%%%%%%%%%%%%%%%%%%%%%%%%%%%%%%%%%%

\chapterimage{header}

\chapter{Unique Elements in a collection}
\label{ch:unique_elements}
\section*{Introduction}
The problem presented in this section is probably one of the most popular
possibly because it features an incredibly simple statement and it is and easy to understand.
This problem has an intuitive brute-force solution that can be coded in a few minutes and that
can be refined and optimized into a short and efficient one. 

\section{Problem statement}
\begin{exercise}
Given a string $s$ of length $n$, determine whether it does \textbf{not} contain duplicate characters. 

\begin{example}
\hfill
	\begin{itemize}
		\item Given \textit{s="graph"} the function returns \inline{true}. There are no duplicates  in $s$.
		\item Given \textit{s="tree"} the function returns \inline{false}. Characters at indices $2$ and $3$  are the same.
	\end{itemize}
\end{example}

\end{exercise}

\section{Clarification Questions}

\begin{QandA}
	\item \begin{questionitem} \begin{question} What is the maximum size of the input?  \end{question} 	 
    \begin{answered}
		\textit{The maximum length for the input string is $10^6$.}
	\end{answered} \end{questionitem}
	
	\item \begin{questionitem} \begin{question} Are all characters upper or lower case?  \end{question} 	 
    \begin{answered}
		\textit{No, both upper and lower case might be present.}
	\end{answered} \end{questionitem}

	\item \begin{questionitem} \begin{question} Is the function case-sensitive?  \end{question} 	 
    \begin{answered}
		\textit{Yes.}
	\end{answered} \end{questionitem}

	\item \begin{questionitem} \begin{question} Can I assume characters only alphanumeric characters are present in the input?  \end{question} 	 
    \begin{answered}
		\textit{Yes. Upper and lower case Latin letters and numbers only.}
	\end{answered} \end{questionitem}
\end{QandA}

\section{Discussion}
Being this problem so popular, the interviewer is expecting you to come up
with a good solution in a relatively small time-window. 
For this reason the obvious $O(n^2)$ solution should be
immediately put on the whiteboard or at least spoken out-loud.

\subsection{Brute Force}
The trivial and easy approach in solving this problem works by looping over each character at index $i$,
and checking if $s_i$ is present in any of the elements  of $s$ appearing at positions higher than $i$. 
In other words we want to check whether the following is true: $\exists \; j $ s.t.  $s_j=s_i$ and $j>i$.
This idea can be implemented as shown in
Listing \ref{list:unique_elements_brute_force1} using two simple nested loops.

\lstinputlisting[language=c++, caption="C++ solution for determining all characters in a string are unique.",label=list:unique_elements_brute_force1]{sources/unique_elements/unique_elements_brute_force.cpp}


As a stylistic improvements to the code in Listing \ref{list:unique_elements_brute_force1}, Listing
\ref{list:unique_elements_brute_force2}  uses the C++ standard library function \inline{std::find} to
search for a duplicate of the character $s_i$. 
This not only makes the code shorter and cleaner,
but also shows to the interviewer that you are able to use the standard library and do not reinvent the wheel.

\lstinputlisting[language=c++, caption="C++ solution for determining if all characters in a string are unique using \inline{std::find}",label=list:unique_elements_brute_force2]{sources/unique_elements/unique_elements_brute_force_std.cpp}

\subsection{Linear time - Hashset}
In Listing \ref{list:unique_elements_brute_force1} the internal loop is doing the hard work of
searching for a duplicate of the character at index $i$. We can trade space for time and reduce the
complexity of the  search for a duplicate of $s_i$ down to $O(1)$.
The idea is that wen can use a set to keep track, as we loop over the characters of $s$, of all the distinct characters seen so far.
A search for a duplicate of $s_i$ becomes a query into this set. If the query is positive then we know we have seen this character before, otherwise 
we inser $s_i$ into the set and can continue  processing the rest of $s$.
Listing \ref{list:unique_elements_brute_force_map} shows how this idea can be implemented.


\lstinputlisting[language=c++, caption="C++ solution for determining all characters in a string are unique in $O(n)$ using an hashset.",label=list:unique_elements_brute_force_map]{sources/unique_elements/unique_elements_brute_force_map.cpp}


This approach effectively lowers the time complexity down to linear, but at the cost of some
space. But how much space exactly? 
The intuition would suggest $O(n)$ because that is the size of the
input string, and afterall we might be inserting into the hashset all of the characters of $s$.
But the intuition is wrong as the string is made of characters from an alphabet $\Sigma$ 
which has a (very) limited size, at most $128$ (which is the size of the ASCII set) elements.
The insert instruction will not be
executed more than $|\Sigma|$ times.
Because of this the space complexity of this solution is $O(1)$. 

The previous argument can be expanded furthermore with the following idea: \textbf{Every string
with more than $|\Sigma|$ character contains at least one duplicate}(follows from the pigeon
principle\footnote{The pigeonhole principle (\url{https://en.wikipedia.org/wiki/Pigeonhole_principle}) states that if $n$ items are put into $m$ containers, with
$n > m$, then at least one container must contain more than one item.}). 
The longest string with only unique characters is one of the permutations of \textit{"abcde\ldots zABCD \ldots Z123 \ldots 9"}.
Thus the solution using the hashset has complexity of $O(1)$ because in the worst case 
we can have $|\Sigma|$ negative lookups.
For this reason, we can limit our investigation to only strings that have size smaller or equal to
$|\Sigma|$ character. For all other strings we can immediately return false.
Note that under the light of these new facts the brute-force approach also has a complexity of $O(1)$
if $i$ and $j$ in Listing \ref{list:unique_elements_brute_force1} are forced to stay below
$|\Sigma|$.

Armed with these new arguments, the solution we suggest to present during an actual interview only uses a
array of booleans of size $|\Sigma|$ storing the information regarding the presence of a
character in the portion of $s$ considered so far.
If at any time the currently examined character has
been already seen, then it is a duplicate.
Listing \ref{list:unique_elements_brute_force_final} shows an implementation of this idea.

\lstinputlisting[language=c++, caption="C++ solution for determining all characters in a string are unique in $O(n)$ using an hashset.",label=list:unique_elements_brute_force_final]{sources/unique_elements/unique_elements_brute_force_final.cpp}

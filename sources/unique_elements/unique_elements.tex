%!TEX root = ../main.tex
%%%%%%%%%%%%%%%%%%%%%%%%%%%%%%%%%%
% Links:
%
% Difficulty: Easy Companies: 
%%%%%%%%%%%%%%%%%%%%%%%%%%%%%%%%%%

\chapterimage{header}

\chapter{Unique Elements in a collection}
\label{ch:unique_elements}
\section*{Introduction}
The problem presented in this Chapter is very popular in coding interviews, 
possibly because it features an incredibly simple statement and is therefore easily understood and used as interview opener. 

The task we are asked to perform is one that programmers encounter almost daily: we are given a list and we need to determine whether it contains duplicates.
Pretty much every modern programmming languages has a library function dedicated to answer this question and in this chapter we will study two possibly algorithmic strategies by which these function can be implemented.
We will  first at an intuitive but slow brute-force solution that can be coded in a handul of minutes and then we will refine and rework it to the point it becomes concise and fast.

\section{Problem statement}
\begin{exercise}
Given a string $s$ of length $n$, determine whether it does \textbf{not} contain duplicate characters. 

\begin{example}
Given \textit{s="graph"} the function returns \inline{true}.
\end{example}

\begin{example}
	Given \textit{s="tree"} the function returns \inline{false} as characters at indices $2$ and $3$ are the same.
	\end{example}
\end{exercise}

\section{Clarification Questions}

\begin{QandA}
	\item \begin{questionitem} \begin{question} What is the maximum length of the input?  \end{question} 	 
    \begin{answered}
		\textit{The maximum length for the input string is $10^6$.}
	\end{answered} \end{questionitem}
	
	\item \begin{questionitem} \begin{question} Are all characters upper or lower case?  \end{question} 	 
    \begin{answered}
		\textit{No, both upper and lower case characters are allowed.}
	\end{answered} \end{questionitem}

	\item \begin{questionitem} \begin{question} Is the function case-sensitive?  \end{question} 	 
    \begin{answered}
		\textit{Yes.}
	\end{answered} \end{questionitem}

	\item \begin{questionitem} \begin{question} Can I assume only alphanumeric characters to be present in $s$?  \end{question} 	 
    \begin{answered}
		\textit{Yes. You can assume that no character other than 
		\begin{itemize*}
			\item upper and  lower case Latin letters and
			\item digits
		\end{itemize*} are present in $s$. }
	\end{answered} \end{questionitem}
\end{QandA}

\section{Discussion}
As this problem so popular, the interviewer will expect a good solution in a short time frame. 

For this reason the obvious $O(n^2)$ solution should be
immediately put on the whiteboard or verbally explained.

\subsection{Brute Force}
The easy approach to solving this problem works by looping over each character at index $i$,
and checking if $s_i$ is present in any of the elements  of $s$ appearing at positions higher than $i$. 
In other words we want to check whether the following is true: $\exists \; j $ s.t.  $s_j=s_i$ and $j>i$.
This idea can be implemented as shown in
Listing \ref{list:unique_elements_brute_force1} using two simple nested loops.

\lstinputlisting[language=c++, caption="C++ solution for determining all characters in a string are unique.",label=list:unique_elements_brute_force1]{sources/unique_elements/unique_elements_brute_force.cpp}


As a stylistic improvement to the code in Listing \ref{list:unique_elements_brute_force1}, Listing
\ref{list:unique_elements_brute_force2}  uses the \CC standard library function \inline{std::find} to
search for a duplicate of the character $s_i$. 
This not only makes the code shorter and cleaner
but also shows to the interviewer that you are able to use the standard library and do not try to reinvent the wheel.

\lstinputlisting[language=c++, caption="\CC solution for determining if all characters in a string are unique using \inline{std::find}",label=list:unique_elements_brute_force2]{sources/unique_elements/unique_elements_brute_force_std.cpp}

\subsection{Linear time - Hashset}
In Listing \ref{list:unique_elements_brute_force1} the internal loop is doing the hard work of
searching for a duplicate of the character at index $i$. We can trade space for time and reduce the
complexity of the  search for a duplicate of $s_i$ down to $O(1)$.
The idea is that we can use a set to keep track, as we loop over the characters of $s$, of all the distinct characters seen so far.
A search for a duplicate of $s_i$ becomes a query into this set. If the query is positive then we know we have seen this character before, otherwise 
we insert $s_i$ into the set and can continue  processing the rest of $s$.
Listing \ref{list:unique_elements_brute_force_map} shows how this idea can be implemented.


\lstinputlisting[language=c++, caption="\CC solution for determining all characters in a string are unique in $O(n)$ using an hashset.",label=list:unique_elements_brute_force_map]{sources/unique_elements/unique_elements_brute_force_map.cpp}


This approach effectively lowers the time complexity down to linear, but at the cost of some
space. But how much space exactly? 
Intuition would suggest $O(n)$ as that is the size of the
input string and, after all,  we might be inserting into the hashset all of the characters of $s$.
But intuition is wrong in this case as the string is made of characters from an alphabet $\Sigma$ 
which has a (very) limited size, at most $128$ (which is the size of the ASCII set) elements.
The insert instruction will not be
executed more than $|\Sigma|$ times.
Because of this the space complexity of this solution is $O(1)$. 

We can expand further on this as follows: \textbf{Every string
with more than $|\Sigma|$ character contains at least one duplicate}(follows from the pigeon
principle\footnote{The pigeonhole principle (\url{https://en.wikipedia.org/wiki/Pigeonhole_principle}) states that if $n$ items are put into $m$ containers, with
$n > m$, then at least one container must contain more than one item.}). 
The longest string with only unique characters is one of the permutations of \textit{"abcde\ldots zABCD \ldots Z123 \ldots 9"}.
Thus the solution using the hashset has complexity of $O(1)$ because in the worst case 
we can have $|\Sigma|$ negative lookups.
For this reason we can limit our investigation to only strings that have size smaller or equal to
$|\Sigma|$ character. For all other strings we can immediately return false.
Note that, in light of these new facts, the brute-force approach also has a complexity of $O(1)$
if $i$ and $j$ in Listing \ref{list:unique_elements_brute_force1} are forced to stay below
$|\Sigma|$.

It therefore follows that the most efficient solution to present during an interview need only use an
array of booleans of size $|\Sigma|$ storing the information regarding the presence of a
character in the portion of $s$ considered so far.
If at any time the currently examined character has
been already seen, then it is a duplicate.
Listing \ref{list:unique_elements_brute_force_final} shows an implementation of this idea.

\lstinputlisting[language=c++, caption="\CC solution for determining all characters in a string are unique in $O(n)$ using an hashset.",label=list:unique_elements_brute_force_final]{sources/unique_elements/unique_elements_brute_force_final.cpp}


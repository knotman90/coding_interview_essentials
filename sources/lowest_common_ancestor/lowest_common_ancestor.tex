%!TEX root = ../main.tex
%%%%%%%%%%%%%%%%%%%%%%%%%%%%%%%%%%
% Links:
%
% Difficulty:
% Companies: 
%%%%%%%%%%%%%%%%%%%%%%%%%%%%%%%%%%

\chapter{Lowest Common Ancestor for BST}
\label{ch:lowest_common_ancestor}
\section*{Introduction}
The lowest common ancestor is an important concept in graph theory and it is
quite often a topic or a fundamental building block of coding interview questions.
Given a tree and two nodes $p$ and $q$, the lowest common ancestor  of $p$ and $q$, ($LCA(p,q)$) is defined as the lowest or deepest node that has both $p$ and $q$ as descendant.
In other words the LCA is the shared ancestor or $p$ and $q$ that is the farthest from the root of the tree.

In \section\ref{} we will investigate an easy approach that can be used to the LCA of two nodes and we will provide an 
\section{Problem statement}
\begin{exercise}
	Given a binary search tree (BST), find the lowest common ancestor (LCA) of two given nodes in the BST.

	According to the definition of LCA on Wikipedia: “The lowest common ancestor is defined between two nodes p and q as the lowest node in T that has both p and q as descendants (where we allow a node to be a descendant of itself).”
	\begin{example}
		\hfill \
		
	\end{example}

	\begin{example}
		\hfill \
		
	\end{example}
\end{exercise}

\section{Clarification Questions}

\begin{QandA}
	\item 
	\begin{answered}
		\textit{}
	\end{answered}
	
\end{QandA}

\section{Discussion}
\label{lowest_common_ancestor:sec:discussion}


\subsection{Brute-force}
\label{lowest_common_ancestor:sec:bruteforce}

\lstinputlisting[language=c++, caption={Sample Caption},label=list:lowest_common_ancestor]{sources/lowest_common_ancestor/lowest_common_ancestor_solution1.cpp}


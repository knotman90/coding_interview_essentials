%!TEX root = ../main.tex
%%%%%%%%%%%%%%%%%%%%%%%%%%%%%%%%%%
% Links:
%
% Difficulty: Easy/Medium Companies: 
%%%%%%%%%%%%%%%%%%%%%%%%%%%%%%%%%%

\chapterimage{header}

\chapter{Two string anagram}
\label{ch:two_string_anagram}
\section*{Introduction}
Anagrams are words that share the same character set. This makes it possible to create multiple words by rearranging the letters in a single source word. For example, the letters in the word \textit{\quotes{alerting}} can be reordered to create 4 new words:
\begin{itemize}
	\item  \textit{\quotes{altering}}
	\item  \textit{\quotes{integral}}
	\item  \textit{\quotes{relating}}
	\item  \textit{\quotes{triangle}}.
\end{itemize}
 
The creation of anagrams, especially ones that reflect or comment on the source words (for instance turning
\textit{\quotes{Madam Curie}} into \textit{\quotes{Radium came}})is difficult. As such, 
computers are often used to find anagrams in longer texts, as well as to generate the so-called anagram dictionaries: a specific kind of dictionary, where all the letters in a word and
all their transposition are arranged in alphabetical order. Such alphabet dictionaries  are often used in games like
Scrabble\footnote{https://en.wikipedia.org/wiki/Scrabble}. Often, at the core of such applications lies an efficient algorithm for determining if a word is an anagram of another word.

In this chapter, we are going to consider anagrams and, more specifically, how to determine the number of
modifications needed to make a word into a valid anagram of another word.
Although this type of question is considered straightforward in the context of coding interviews as all it really requires is a basic understanding of the concept of an anagram; nevertheless it is worth studying as it is often asked during the preliminary interview stages and provides an opportunity to demonstrate more than one neat and elegant approach leading to an efficient solution to the problem.

We will examine three possible
solutions, beginning with the slow but easy to understand brute-force in Section \ref{sec:anagrams:bruteforce}, moving on to a faster approach using sorting in Section \ref{sec:anagrams:sorting},and finally addressing the optimal solution running in linear time in Section
\ref{sec:anagrams:histograms}. 

\section{Problem statement}
	\begin{exercise}
		Write a function that given two string, $a$ and $b$ of length $n$ and $m$, respectively, determines the minimum number of character substitution, $C(s, i, c)$, necessary to make the string $a$ an anagram of the string $b$.

		Two strings are said to be anagrams of one another if you can turn the first string into the second by rearranging its letters. 

		A substitution operation $C(s,i,c)$ modifies the string $s$, by changing its $i^{th}$ character into $c$. Notice that deletions or additions of characters are not allowed.
		The only operation you can do is change a character of the first string into another one. 

		In other words, what is the minimum number of characters of the input strings that need to be modified (no addition or deletion)  so that $a$ becomes an anagram of $b$?
		
	\begin{example}
		\label{ex:two_string_anagram:example1}
		\hfill \\
		\begin{itemize}
			\item 	a = \quotes{aaa}
			\item 	b = \quotes{bbb}
		\end{itemize}
		The function returns $3$. 
		All the characters of \textit{a} need to be changed into \textit{'b'}.
		\label{ex:anagrams:example1}
	\end{example}

	\begin{example}
		\hfill \\
		\begin{itemize}
			\item 	a = \quotes{tear}
			\item	b = \quotes{fear}
		\end{itemize}
		The function returns $1$. 
		All that is necessary is turning the first letter \textit{'t'} into a \textit{'f'}.
	\end{example}

	\begin{example}
		\hfill \\
		\begin{itemize}
			\item[] 	a = \quotes{Protectional}
			\item[] 	b = \quotes{Lactoprotein}
		\end{itemize}
		The answer for this case is $0$ because \emph{Protectional} is already an anagram of \emph{Lactoprotein}.
	\end{example}
\end{exercise}

\section{Clarification Questions}

\begin{QandA}
	\begin{questionitem} \begin{question} Are the letters of the string always only letters from the English alphabet?    \end{question}      
    \begin{answered}
		\textit{Yes, letters are always from the English alphabet.}
	\end{answered} \end{questionitem}
	
	\begin{questionitem} \begin{question} Should the function be case sensitive?   \end{question}      
    \begin{answered}
		\textit{ No. You can assume the input letters are always lower case.}
	\end{answered} \end{questionitem}
	\begin{questionitem} \begin{question} Can the input string be modified? No, the input is immutable.  \end{question}      
    \begin{answered}
		\textit{No, the input strings are immutable.}
	\end{answered} \end{questionitem}

	\begin{questionitem} \begin{question} What value should be returned when there is no solution?  \end{question}      
    \begin{answered}
		\textit{In such case you can return $-1$.}
	\end{answered} \end{questionitem}
\end{QandA}

\section{Discussion}

Let's start by reviewing what the word anagram means in the context of this problem. First note that both $a$ and $b$ contain a single word (which can be fairly long).
Moreover, for $a$ to be an anagram
of $b$, it has to be the case that there exists an arrangement of characters in $a$ that is equal to $b$.
In other words, the question we need to answer is: is it possible to shuffle the character of $a$ so that we obtain $b$?
For this to be the case, it must be that $a$ and $b$ contain the same set of characters meaning that sorting both $a$ and $b$ would make them equal.
In addition, as a consequence of the fact that no addition or deletion
is allowed, \textbf{$a$ and $b$ must have the same length}. 
On the other hand, if they have the same length then it is always
possible to solve this problem because in the worst case, we can modify every letter of $a$ (see Example \ref{ex:two_string_anagram:example1}).
Thus, the only case when the problem has no solution has been isolated: when $n \neq m$ we must return $-1$ otherwise we can proceed with our calculation knowing that a solution exists.

\subsection{Brute-Force}
\label{sec:anagrams:bruteforce}
One of the first options to consider is a solution where we generate all possible arrangements of the letters in $a$, and for each of these arrangements, calculate the number of modifications necessary to convert it into $b$. 
The key idea is that the cost of transforming a string into another is equal to the number positions having different letters. 
For instance, the cost of transforming \textit{\quotes{abcb}} into \textit{\quotes{bbbb}} is $2$ because the two strings differ in the first and third letters. 

Although it is simple to explain, this approach must be considered sub-optimal because the number of arrangements of a set of $n$ letters grows as fast as $n!$.
Moreover, enumerating all the arrangements is no trivial task, unless we use a  library function capable of doing it (for instance, the C++ standard library provides the function \inline{std::next_permutation}\footnote{\href{https://en.cppreference.com/w/cpp/algorithm/next_permutation}{https://en.cppreference.com/w/cpp/algorithm/next\_permutation}} devoted to this purpose).


Listings \ref{list:two_string_anagram_bruteforce} shows a \CC implementation of the idea above.

\lstinputlisting[language=c++, caption=\quotes{Brute force.},label=list:two_string_anagram_bruteforce]{sources/two_string_anagram/two_string_anagram_brute_force.cpp}


\subsection{Sorting}
\label{sec:anagrams:sorting}
This brute-force solution does a lot of superfluous work, because it tries to find a permutation of the string $a$ requiring minimal modifications to be morphed into $b$.
But is it really necessary to turn $a$ into \textbf{exactly} $b$, or is it sufficient to modify $a$ so that it is equal to a particular permutation of $b$? 
After all, being an anagram is a transitive property: if $a$ is a permutation of $b$ and $b$ is a permutation of $c$, then $a$ must also be a permutation of $c$. 

By definition, an anagram of $b$ is any permutation of its characters, and therefore, the particular permutation in which the characters of $b$ are sorted is a valid anagram on its own. 
It is much easier than checking all possible permutations, to modify $a$ into the \quotes{sorted} anagram of $b$ (where all of its characters are sorted), rather than to exactly $b$ because all we need to do is to create a copy of both $a$ and $b$, sort both of them and then calculate the character-by-character difference.
\textbf{This approach works because if $x$ is an anagram of $b$ then $x$ is also an
anagram of `sort(b)`}.
In other words, it does not matter how the characters are arranged in $a$ and $b$, as the only thing that matters is the set of the characters
appearing in $a$ and $b$: the order in which characters in both $a$ and $b$ appear does not matter. 

Listings \ref{list:two_string_anagram_sorting}  shows how we can take advantage of this fact and write a fast solution for this problem.

\lstinputlisting[language=c++, caption="Solution based on sorting.",label=list:two_string_anagram_sorting]{sources/two_string_anagram/two_string_anagram_sorting.cpp}

Note that, if the input was mutable, then, the additional space occupied by the copies of the string \inline{aa} and \inline{bb} could have been avoided. 

The time complexity of Listing \ref{list:two_string_anagram_sorting}  is $O(n log(n))$ (because of sorting). The space complexity is $O(n)$ (we create copies of the input strings).



\subsection{Histograms}
\label{sec:anagrams:histograms}


There is another piece of information that we have not used yet: \textbf{the alphabet} from which the
letters of $a$ and $ab$ are taken from \textbf{is small}. 
If the only thing that matters is the set of characters appearing in $a$ and $b$ (and not their order, as discussed above),
then we can use the same idea at the core of the \href{https://en.wikipedia.org/wiki/Bucket_sort}{bucket sort} algorithm to achieve a linear time complexity solution.


The key idea here is to pre-process $a$ and $b$ so as to calculate their per-character frequencies, denoted here as $F_a$ and $F_b$, respectively.
An entry of $F_a[\mathrm{c}]$ and $F_b[\mathrm{c}]$, where $\mathrm{c} \in \{\mathrm{a},\mathrm{b},\ldots,\mathrm{z}\}$ (a letter of the alphabet), contains the frequency of character $\mathrm{c}$, in $a$ and $b$, respectively.


If $F_a$ and $F_b$ are the same, then $a$ and $b$ have exactly the same character set and $a$ is an anagram of $b$.
Otherwise, it must be the case that some characters of $a$ appear in $b$ a different number of times.
In this case, we can fix $a$ in such a way as to make sure that its frequencies $F_a$ ey match the ones in $F_b$. 
But the main question remains unanswered: how many operations are necessary to do so?  In order to get this answer, it is useful to look at
the difference ($D$) of $F_a$ and $F_b$.

$D = F_a - F_b = \{D[\mathrm{a}] = (F_a[\mathrm{a}] - F_b[\mathrm{a}]), D[\mathrm{b}] = (F_a[\mathrm{b}] - F_b[\mathrm{b}]), D[\mathrm{c}] = (F_a[\mathrm{c}] - F_b[\mathrm{c}]), \ldots, D[\mathrm{z}] = (F_a[\mathrm{z}] - F_b[\mathrm{z}])\}
$

$D[\mathrm{c}]$ (where $\mathrm{c} \in \{\mathrm{a},\mathrm{b},\ldots,\mathrm{z}\}$) contains the difference between the number of occurrences of the character $\mathrm{c}$ in the string $a$ and $b$. Depending on whether the value of $D[\mathrm{c}]$  is greater or smaller than $0$, $a$ has an excess or a deficit of the letter c, respectively.

Firstly, note that $\sum_{c=\mathrm{a}}^{\mathrm{z}} D[\mathrm{c}] = 0$. This observation stems from the fact that $|a|=n=m=|b|$ ($a$ and $b$ must have equal length for this problem to have a solution as noted above) and that if $a$ has an excess of a certain character $\mathrm{c}$ then there must exist another character $\mathrm{d} \neq \mathrm{c}$ that the string $a$ has a shortage of. If that is not the case, it is impossible for $a$ and $b$ to have equal length.

We can use this fact to modify the excesses of the letters of $a$, the ones having a positive value of $D$ into some of the letters there is a shortage of so that eventually, every single value of $D$ is zero.
If $D[\mathrm{c}] = x$ is going to take $x$ modifications to transform the excess of characters $\mathrm{c}$.
The answer to this problem is, therefore, the sum of all the positive numbers of $D$. 

Listings \ref{list:two_string_anagram_histogram} shows a possible implementation of the solution above.

\lstinputlisting[language=c++, caption="\CC solution to the two string anagram problem using the histogram approach.",label=list:two_string_anagram_histogram]{sources/two_string_anagram/two_string_anagram_histogram.cpp}

Note how the array of differences of frequencies $D$ can be calculated easily without explicitly
computing the frequencies for the characters of $a$ and $b$ but simply by adding $1$ to $D[\mathrm{c}]$ when the letter $\mathrm{c}$ appears in $a$
and subtracting $1$ when it does in $b$. 

The time and space complexity of the code above is $O(n)$ and $O(1)$ in space (we are using an array of $128$ integers regardless of the size of the input). We cannot do any better than this, as all characters in the input strings must be read at least once.



%\section{Conclusion}

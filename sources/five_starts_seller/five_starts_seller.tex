%!TEX root = ../main.tex
%%%%%%%%%%%%%%%%%%%%%%%%%%%%%%%%%%
% Links:
%
% Difficulty:
% Companies: 
%%%%%%%%%%%%%%%%%%%%%%%%%%%%%%%%%%

\chapter{Five Starts Sellers}
\label{ch:five_starts_seller}
\section*{Introduction}
Given the number of five-star and total reviews for each product a company sells,
as well as the threshold percentage, 
what is the minimum number of additional five-star reviews the company needs to become five star seller.
For ex, there are 3 products (n=3) with productRatings =[[4,4],[1,2],[3,6]], percentage rating threshold = 77.
[1,2] indicates => [1 (five star reviews) ,2 (total reviews)].
We need to get the seller reach the threshold with minimum number of additional five star reviews.

Before we add more five star reviews, the percentage for this seller is ((4/4) + (1/2) + (3/6))/3 = 66.66%
If we add a five star review to 2nd product, ((4/4) + (2/3) + (3/6))/3 = 72.22%
If we add another five star review to 2nd product, ((4/4) + (3/4) + (3/6))/3 = 75%
If we add a five star review to 3rd product, ((4/4) + (3/4) + (4/7))/3 = 77.38%
At this point, 77% (threshold) is met. Therefore, answer is 3 (because that is the minimum five star reviews we need to add, to get the seller reach the threshold).

\begin{lstlisting}[language=c++, caption={Definition of a linked list with a random pointer.},label=list:delete_duplicates_list:linked_list]
	public static int fiveStarReviews(List<List<Integer>> productRatings, int ratingsThreshold){
	}




\end{lstlisting}
\section{Problem statement}
\begin{exercise}

\end{exercise}


\begin{example}
	\hfill \
	
\end{example}

\begin{example}
	\hfill \
	
\end{example}

\section{Clarification Questions}




\begin{QandA}
	\item 
	\begin{answered}
		\textit{}
	\end{answered}
	
\end{QandA}

\section{Discussion}
\label{five_starts_seller:sec:discussion}

Please point out if my solution is wrong.

Upvote if you like it.

Part 0. Intuition
Heap question. To improve the rating at best, we need to figure out what's the difference between current rating and if 1 more five star is added to this product
thus, create heap = [[-diff(p)]+p for p in productRatings], where diff is the difference between current state and if one more 5 star is added to it
there is no max-heap in Python, so use negative (-) instead.
until our rating overpasses threshold, pop heap and update the diff then push back to heap
Return how many times we did this (ans)
Part 1. Implementation
class Solution:
    def fiveStartReviews(self, productRatings, ratingsThreshold):
        n = len(productRatings)
        ans = 0
        cur_rating = sum(p[0]/p[1] for p in productRatings) / n * 100
        def diff(p): return (p[0]+1)/(p[1]+1) - p[0]/p[1]
        heap = [[-diff(p)]+p for p in productRatings]                     # create diff and store in heap & heapify
        heapq.heapify(heap)
        while cur_rating < ratingsThreshold:
            p_diff, p0, p1 = heapq.heappop(heap)
            cur_rating = (cur_rating * n + (-p_diff) * 100) / n           # update current rating by adding the difference
            heapq.heappush(heap, [-diff([p0+1, p1+1]), p0+1, p1+1])       # maintain diff order in heap
            ans += 1
        return ans
Part 2. Input & Test
productRatings = [[[4, 4], [1, 2], [3, 6]]]
ratingsThreshold = [77]
output = [3]

s = Solution()
for i in range(len(productRatings)):
    assert s.fiveStartReviews(
		productRatings[i], ratingsThreshold[i]) == output[i]
		




		Java PriorityQueue solution, the idea is same as other guys' solutions. For each product, we want to calculate the biggest percentage jump if we add one more five-start to it. So we can use a max-heap so the head of the heap is current biggest jump product. Then we create a new rating based on this head, add one more five-start to this new rating and push to the heap again. Reminder to keep track the updated rating until we are above the threshold. I don't have enough test cases, so please point me out any problems. Thanks!

public class Solution {
    public int fiveStarReviews(List<List<Integer>> productRatings, int ratingsThreshold){
        int num = 0;
        PriorityQueue<List<Integer>> pq = new PriorityQueue<>((l1, l2) -> diff(l2) - diff(l1)); // max-heap.
        for(List<Integer> rating : productRatings) pq.offer(rating); // initialize PriorityQueue.
        double curRating = 0;
        for(List<Integer> rating : productRatings) curRating += 100.0 * rating.get(0) / rating.get(1); // initialize curRating.
        while(curRating < ratingsThreshold * productRatings.size()) {
            num++;
            List<Integer> rating = pq.poll();
            List<Integer> newRating = Arrays.asList(rating.get(0)+1, rating.get(1)+1);
            curRating = curRating - 100.0 * rating.get(0) / rating.get(1) + 100.0 * newRating.get(0) / newRating.get(1); // keep updating the rating.
            pq.offer(newRating);
        }
        return num;
    }

    // the diff between the current product rating and the product added one more five-star rating.
    private int diff(List<Integer> p) {
        double currRating = 100.0 * p.get(0) / p.get(1);
        double newRating = 100.0 * (p.get(0)+1) / (p.get(1)+1);
        return (int)(newRating - currRating);
    }

    public static void main(String[] args) {
        List<List<Integer>> ratings = new ArrayList(){
            {
                add(Arrays.asList(4,4));
                add(Arrays.asList(1,2));
                add(Arrays.asList(3,6));
            }
        };
        int threshold = 77;
        Solution s = new Solution();
        System.out.println(s.fiveStarReviews(ratings, threshold));
    }
}



I think the underlying intuition here is that the further that a product rating is away from 100%
, the quicker its rating improves. 0/1 -> 1/2 is a 50%
 jump. 1/2 -> 2/3 is a 17%
  jump. 2/3 -> 3/4 is a 9%
   jump and so on. So make a priority queue with a comparator for the product ratings, update the one with the lowest and add it back to the PQ
\subsection{Brute-force}
\label{five_starts_seller:sec:bruteforce}

\lstinputlisting[language=c++, caption={Sample Caption},label=list:five_starts_seller]{sources/five_starts_seller/five_starts_seller_solution1.cpp}


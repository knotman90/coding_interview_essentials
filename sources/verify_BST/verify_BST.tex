%!TEX root = ../main.tex
%%%%%%%%%%%%%%%%%%%%%%%%%%%%%%%%%%
% Links:
%
% Difficulty:
% Companies: 
%%%%%%%%%%%%%%%%%%%%%%%%%%%%%%%%%%

\chapter{Verify BST property}
\label{ch:verify_BST}
\section*{Introduction}

\section{Problem statement}
\begin{exercise}
Given a binary tree \cite{cit:wiki:BST}, determine if it is a valid binary search tree.

Assume a BST is defined as follows:
\begin{itemize}
    \item The left subtree of a node contains only nodes with keys less than the node's key.
    \item The right subtree of a node contains only nodes with keys greater than the node's key.
\end{itemize}
You can assume that the tree structure is defined as shown in Listing \ref{list:verify_BST:tree_structure}: 

\end{exercise}

\begin{lstlisting}[language=c++, caption=Binary tree definition used in this exercice.,label=ist:verify_BST:tree_structure]

 struct TreeNode {
     int val;
     TreeNode *left;
     TreeNode *right;
     TreeNode(int x) : val(x), left(nullptr), right(nullptr) {}
 };
 \end{lstlisting}


\begin{example}
	\hfill \\
	For the following tree the function should return \textbf{false}.
	\begin{verbatim}
	    5
	   / \
	  1   4
	     / \
	    3   6
	\end{verbatim}
\end{example}

\begin{example}
	\hfill \\
	For the following tree the function should return \textbf{true}.
	\begin{verbatim}
	    2
	   / \
	  1   3
	\end{verbatim}
	
\end{example}

\begin{example}
\label{example:verify_BST_:one}
	\hfill \\
	For the following tree the function should return \textbf{true}.
	\begin{verbatim}
	    10
	   / \
	  1   14
	 / \    \
    0  9     16
            /  \
           15   19
	\end{verbatim}
	
\end{example}


\section{Clarification Questions}

\begin{QandA}
	\item Are all elements in the tree distinct?
	\begin{answered}
		\textit{Yes, you can assume all elemets are distinct.}
	\end{answered}
	\item How many nodes does the tree contain?
	\begin{answered}
		\textit{Up to $10^6$ nodes.}
	\end{answered}
\end{QandA}

\section{Discussion}
\label{verify_BST:sec:discussion}
The problem is asking for a function that verifies whether a givan tree is a binary search tree. In other words does the given tree satisfy the BST property?  But what does it mean exactly? The BST property says that:
A tree T is a binary search tree if:
\begin{enumerate}
	\item Every node has two subtree (named left and right, respectively)
	\item given a node $n$ in the tree all the nodes in its left subtree are smaller than the value in $n$.
	\item additionally, all nodes in the right subtree are larger.
\end{enumerate}


\subsection{Top Down approach}
\label{verify_BST:sec:topdown}
First of all let's notice that there we need to be able to visit all nodes in order to verify whether the BST property holds. A top down approach can be used to solve this problem. A top down approach on a tree should immediatly ring a bell on recursion. Infact this problem becomes almost trivial once two key considerations are made:
\begin{enumerate}
	\item every node can be thought as a separate tree for which the BST property needs to hold. if the property is verified for all nodes
	\item empty trees satisfy the BST property
	\item every nodes must be within a certain range that is determined by its parent nodes. For instance given the node $15$ in the example \ref{example:verify_BST_:one}. In order to be a valid tree this node must be within the range $(14,16)$ because.  The node $9$ must be within the range $(1,10)$.
\end{enumerate}


\lstinputlisting[language=c++, caption=Sample Caption,label=list:verify_BST]{sources/verify_BST/verify_BST_solution1.cpp}


%!TEX root = ../main.tex
%%%%%%%%%%%%%%%%%%%%%%%%%%%%%%%%%%
% Links:
%
% Difficulty:
% Companies: 
%%%%%%%%%%%%%%%%%%%%%%%%%%%%%%%%%%

\chapter{TITLE OF THE CHAPTER}
\label{ch:kth_largest_in_stream}
\section*{Introduction}

\section{Problem statement}
\begin{exercise}

	Design a class to find the kth largest element in a stream. Note that it is the kth largest element in the sorted order, not the kth distinct element.

	Implement KthLargest class:
	
	KthLargest(int k, int[] nums) Initializes the object with the integer k and the stream of integers nums.
	int add(int val) Returns the element representing the kth largest element in the stream.
	 
	
	Example 1:
	
	Input
	["KthLargest", "add", "add", "add", "add", "add"]
	[[3, [4, 5, 8, 2]], [3], [5], [10], [9], [4]]
	Output
	[null, 4, 5, 5, 8, 8]
	
	Explanation
	KthLargest kthLargest = new KthLargest(3, [4, 5, 8, 2]);
	kthLargest.add(3);   // return 4
	kthLargest.add(5);   // return 5
	kthLargest.add(10);  // return 5
	kthLargest.add(9);   // return 8
	kthLargest.add(4);   // return 8
	 
	
	Constraints:
	
	1 <= k <= 104
	0 <= nums.length <= 104
	-104 <= nums[i] <= 104
	-104 <= val <= 104
	At most 104 calls will be made to add.
	It is guaranteed that there will be at least k elements in the array when you search for the kth element.


\begin{example}
	\hfill \
	
\end{example}

\begin{example}
	\hfill \
	
\end{example}

\end{exercise}

\section{Clarification Questions}

\begin{QandA}
	\item 
	\begin{answered}
		\textit{}
	\end{answered}
	
\end{QandA}

\section{Discussion}
\label{kth_largest_in_stream:sec:discussion}


\subsection{Brute-force}
\label{kth_largest_in_stream:sec:bruteforce}

\lstinputlisting[language=c++, caption={Sample Caption},label=list:kth_largest_in_stream]{sources/kth_largest_in_stream/kth_largest_in_stream_solution1.cpp}


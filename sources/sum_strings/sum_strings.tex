%!TEX root = ../main.tex
%%%%%%%%%%%%%%%%%%%%%%%%%%%%%%%%%%
% Links:
%
% Difficulty: Easy/Medium
% Companies: 
%%%%%%%%%%%%%%%%%%%%%%%%%%%%%%%%%%

\chapterimage{header}

\chapter{Sum of two strings}
\label{ch:sum_two_strings}
\section*{Introduction}

In this chapter, we will have a look at another commonly asked question on a commonly asked topic, namely string processing.

\section{Problem statement}

Write a function (with signature: \lstinline[columns=fixed]{long my_sum_strings(std::string str1, std::string str2, int base)} ) which takes in two strings which represent numbers in a certain base and returns the sum of these strings.


\section{Clarification Questions}
\begin{QandA}

\item What is the maximum size of the string?
	\begin{answered}
		\textit{You can be sure that the sum of two numbers will not exceed the range of long. Also, you can assume that the input string is always a valid input.}
	\end{answered}

\item What is the maximum value that base can take?
	\begin{answered}
		\textit{The value of the base can vary between 1 and 36.}
	\end{answered}

\end{QandA}

\section{Discussion}

This is a pretty straight-forward question and the only problem that you would encounter is the sum operation in a different base other than 10. You could take two approaches to this, you could define a generalized sum operation for different bases. However, the implementation in this case is not going to be straight-forward. The second approach would be to convert the inputs into a base of 10 and use the in-built sum operator to do the sum for you. This is a more easier approach in-terms of implementation. Now, let us analyse the complexity of this approach. In this case, you need to do three operations to find the end result. The first would be to convert the inputs to a base 10. As you might know, to convert a number from a different base to 10, you have to iterate through the whole number to obtain the base in 10. This would put the complexity of this operation in ${O(n)}. The sum operation could be done in constant time ${O(1)}. Also, you would have to convert the result back to the original base. This could be done by ${O_{b}log(n)} using prime factorization of the result in a certain base. The entire approach could be generalized to a ${O(n)} solution. There is another case to keep in mind, which is the base 1. As you should know, when you represent a number in base 1, the number is basically the integer 1 being repeated upto the count of the number(for instance 11 represents 2 and 111 represents 3 and so on). In this case, the sum of two numbers in base is the length of the size of the inputs. 

This idea can be implemented as shown in Listing \ref{list:sum_strings_approach_1}.

\begin{minipage}{\linewidth}
	\lstinputlisting[language=c++, caption="C++ solution for determining sum of two string numbers in any base.", label=list:sum_strings_approach_1]{sources/sum_strings/sum_strings_approach_1.cpp}
\end{minipage}
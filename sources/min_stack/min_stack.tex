%!TEX root = ../main.tex
%%%%%%%%%%%%%%%%%%%%%%%%%%%%%%%%%%
% Links:
%
% Difficulty:
% Companies: 
%%%%%%%%%%%%%%%%%%%%%%%%%%%%%%%%%%

\chapter{Min stack}
\label{ch:min_stack}
\section*{Introduction}
This chapter introduces a very popoular question among among companies like Amazon and Google. The question is very simple and is about designing a data structure for performing stack operations and that is also able to keep track of the minimum element that is currently present in the stack. This problem has a simple, short and elegant solution. It is important to lean how to solve this problem as it is not unlikely we will be asked a very similar problem during the online screening steps of the interview process.

\section{Problem statement}
\begin{exercise}
Design  a stack that supports:

\begin{itemize}
	\item \lstinline[columns=fixed]{push(x)}: the element $x$ is pushed onto the stack
	\item \lstinline[columns=fixed]{pop()}: removes the top of the stack
	\item \lstinline[columns=fixed]{top()}: retrieve the top of the stack
	\item \lstinline[columns=fixed]{min()}: retrieve the minimum among all elements present in the stack.
\end{itemize}


	\begin{example}
		\hfill \\
		Suppose the following set of operation on the stack are performed on a newly constructed and empty stack $S$:
		\begin{itemize}
			\item[-] \lstinline[columns=fixed]{push(1)}: $S=[1]$
			\item[-] \lstinline[columns=fixed]{push(5)}: $S=[5,1]$
			\item[-] \lstinline[columns=fixed]{push(3)}: $S=[3,5,1]$
			\item[-] \lstinline[columns=fixed]{top()}: $S=[3,5,1]$, returns $3$
			\item[-] \lstinline[columns=fixed]{pop()}: $S=[5,1]$
			\item[-] \lstinline[columns=fixed]{min()}: $S=[5,1]$, return $1$
			\item[-] \lstinline[columns=fixed]{push(0)}: $S=[0,5,1]$
			\item[-] \lstinline[columns=fixed]{min()}: $S=[0,5,1]$, returns $0$
		\end{itemize}

	\end{example}

	\begin{example}
		\hfill \\
		Suppose the following set of operation on the stack are performed on a newly constructed and empty stack $S$:
		\begin{itemize}
			\item[-] \lstinline[columns=fixed]{push(3)}: $S=[3]$
			\item[-] \lstinline[columns=fixed]{push(5)}: $S=[5,3]$
			\item[-] \lstinline[columns=fixed]{push(1)}: $S=[1,5,3]$
			\item[-] \lstinline[columns=fixed]{min}: $S=[1,5,3]$, return $1$
			\item[-] \lstinline[columns=fixed]{pop()}: $S=[5,3]$, returns $3$
			\item[-] \lstinline[columns=fixed]{min()}: $S=[5,3]$, return $3$
			\item[-] \lstinline[columns=fixed]{pop()}: $S=[1]$, return $1$
			\item[-] \lstinline[columns=fixed]{pop()}: $S=[]$
			\item[-] \lstinline[columns=fixed]{pop()}: raise \lstinline[columns=fixed]{std::logic_error}
		\end{itemize}
		
	\end{example}
\end{exercise}

\section{Clarification Questions}

\begin{QandA}
	\item What should be done when min() or top() or pop() are performed on an empty stack?
	\begin{answered}
		\textit{You should throw a \lstinline[columns=fixed]{std::logic_error} exception with a sensible and short description.}
	\end{answered}
	
\end{QandA}

\section{Discussion}
\label{min_stack:sec:discussion}


\subsection{Brute-force}
\label{min_stack:sec:bruteforce}

\lstinputlisting[language=c++, caption={Sample Caption},label=list:min_stack]{sources/min_stack/min_stack_solution1.cpp}


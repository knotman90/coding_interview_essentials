%!TEX root = ../main.tex
%%%%%%%%%%%%%%%%%%%%%%%%%%%%%%%%%%
% Links:https://leetcode.com/problems/minimum-difficulty-of-a-job-schedule/
%
% Difficulty: Companies: 
%%%%%%%%%%%%%%%%%%%%%%%%%%%%%%%%%%

\chapter{Minimum difficulty job schedule}
\label{ch:min_difficulty_job_scheduler}
\section*{Introduction}
Imagine you are part of a team currently busy doing beta testing on your new cool feature. The
testing consists of executing a number of tasks. Each task has dependencies on other tasks and is
assigned a certain  amount of complexity points (a measure of how difficult a task is to be
performed; it is not a measure of time). The dependencies between the tasks have been already worked
out i.e. the order in which the task are going to be executed is decided. The problem in this
chapter is about coming up with a schedule plan for the execution of these tasks spanning across a
given number of days. Among all possible schedules we need to make an effort to calculate the
minimum possible complexity achievable for a schedule that will eventually make sure all tasks are
executed and also that there is at least one task executed every day.


\section{Problem statement}
\begin{exercise}
	Write a function that takes as an input a list of tasks $I$ and an integer $d$. The elements in
	$I$ are dependent on each other and in order to schedule a certain task $I_i$ all the tasks $I_j
	\: : j < i$ have to be completed. The function should return the minimum complexity among all
	possible schedules of lenght exactly $d$ days. The complexity of a job is calculated as the sum
	of the complexity of each single day of the schedule. The complexity of a day of the schedule is
	defined as the maximum complexity of the task planned for that day.

	As an additional constraint you have to make sure that there is at least a task scheduled for
	each day.
		
	\begin{example}
		\hfill \\
		Given:
		\begin{itemize}
			\item $I = \{6,5,4,3,2,1\}$
			\item $d = 2$
		\end{itemize}
		the function returns $7$. You can schedule tasks $0$ to $4$ during the first day and the
		last task during the second day. You cannot just schedule all tasks during the first day
		because then you would have a day in the schedule without planned tasks which is not
		allowed.
	\end{example}
	
	\begin{example}
		\hfill \\
		Given:
		\begin{itemize}
			\item $I = \{10,10,10\}$
			\item $d = 4$
		\end{itemize}
		the function returns $-1$. There is no way to schedule tasks for $4$ days when there are
		only $3$ tasks available for scheduling.
	\end{example}

		
	\begin{example}
		\hfill \\
		Given:
		\begin{itemize}
			\item $I = \{7,1,7,1,7,1\}$
			\item $d = 3$
		\end{itemize}
		the function returns $15$. You can schedule the first $4$ tasks the first day for a total
		complexity of $7$. Task at index $4$ and $5$ can be scheduled for days $2$ and $3$
		respectively. 

		Notice that in this case if $d = 2$ then the function would return $8$.
	\end{example}

	\begin{example}
		\hfill \\
		Given:
		\begin{itemize}
			\item $I = \{11,111,22,222,33,333,44,444\}$
			\item $d = 6$
		\end{itemize}
		the function returns $843$. You can schedule tasks $0,1,2,3,4$ in the first $5$ days and the
		rest during the \nth{6}.
		
	\end{example}
\end{exercise}





	\section{Clarification Questions}
	
	\begin{QandA}
		\item What should the function return in case where it is not possible to make a valid
		schedule? For instance when $|I| < d$?
	\begin{answered}
		\textit{You can return $-1$ in that case.}
	\end{answered}

	\item It is guaranteed for the complexity values to be positive ($\gew 0$)?
	\begin{answered}
		\textit{Yes you can assume complexities are always positive.}
	\end{answered}

	
	
\end{QandA}

\section{Discussion}
\label{min_difficulty_job_scheduler:sec:discussion}
Use DP. Try to cut the array into d non-empty sub-arrays. Try all possible cuts for the array.
Use dp[i][j] where DP states are i the index of the last cut and j the number of remaining cuts. Complexity is O(n * n * d).

\subsection{Brute-force}
\label{min_difficulty_job_scheduler:sec:bruteforce}

\lstinputlisting[language=c++, caption={Sample Caption},label=list:min_difficulty_job_scheduler]{sources/min_difficulty_job_scheduler/min_difficulty_job_scheduler_solution1.cpp}


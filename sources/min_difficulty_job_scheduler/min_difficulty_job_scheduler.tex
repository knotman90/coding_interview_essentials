%!TEX root = ../main.tex
%%%%%%%%%%%%%%%%%%%%%%%%%%%%%%%%%%
% Links:https://leetcode.com/problems/minimum-difficulty-of-a-job-schedule/
%
% Difficulty: Companies: 
%%%%%%%%%%%%%%%%%%%%%%%%%%%%%%%%%%

\chapter{Minimum difficulty job schedule}
\label{ch:min_difficulty_job_scheduler}
\section*{Introduction}
Imagine you are part of a team currently busy doing beta testing on your new cool feature. The
testing consists of executing a number of tasks. Each task has dependencies on other tasks and is
assigned a certain  amount of complexity points (a measure of how difficult a task is to be
performed; it is not a measure of time). The dependencies between the tasks have been already worked
out i.e. the order in which the task are going to be executed is decided. The problem in this
chapter is about coming up with a schedule plan for the execution of these tasks spanning across a
given number of days. Among all possible schedules we need to make an effort to calculate the
minimum possible complexity achievable for a schedule that will eventually make sure all tasks are
executed and also that there is at least one task executed every day.


\section{Problem statement}
\begin{exercise}
	Write a function that takes as an input a list of tasks $I$ and an integer $d$. The elements in
	$I$ are dependent on each other and in order to schedule a certain task $I_i$ all the tasks $I_j
	\: : j < i$ have to be completed. The function should return the minimum complexity among all
	possible schedules of length exactly $d$ days. The complexity of a job is calculated as the sum
	of the complexity of every single day of the schedule. The complexity of a day of the schedule is
	defined as the maximum complexity of the task planned for that day.

	As an additional constraint you have to make sure that there is at least a task scheduled for
	each day.
		
	\begin{example}
		\hfill \\
		Given:
		\begin{itemize}
			\item $I = \{6,5,4,3,2,1\}$
			\item $d = 2$
		\end{itemize}
		the function returns $7$. You can schedule tasks $0$ to $4$ during the first day and the
		last task during the second day. You cannot just schedule all tasks during the first day
		because then you would have a day in the schedule without planned tasks which is not
		allowed.
	\end{example}
	
	\begin{example}
		\hfill \\
		Given:
		\begin{itemize}
			\item $I = \{10,10,10\}$
			\item $d = 4$
		\end{itemize}
		the function returns $-1$. There is no way to schedule tasks for $4$ days when there are
		only $3$ tasks available for scheduling.
	\end{example}

		
	\begin{example}
		\hfill \\
		Given:
		\begin{itemize}
			\item $I = \{7,1,7,1,7,1\}$
			\item $d = 3$
		\end{itemize}
		the function returns $15$. You can schedule the first $4$ tasks the first day for a total
		complexity of $7$. Task at index $4$ and $5$ can be scheduled for days $2$ and $3$
		respectively. 

		Notice that in this case if $d = 2$ then the function would return $8$.
	\end{example}

	\begin{example}
		\hfill \\
		Given:
		\begin{itemize}
			\item $I = \{11,111,22,222,33,333,44,444\}$
			\item $d = 6$
		\end{itemize}
		the function returns $843$. You can schedule tasks $0,1,2,3,4$ in the first $5$ days and the
		rest during the \nth{6}.
		
	\end{example}
\end{exercise}



\section{Clarification Questions}
		
	\begin{QandA}
			\item What should the function return in case where it is not possible to make a valid
			schedule? For instance when $|I| < d$?
		\begin{answered}
			\textit{You can return $-1$ in that case.}
		\end{answered}

		\item It is guaranteed for the complexity values to be positive ($\geq 0$)?
		\begin{answered}
			\textit{Yes you can assume complexities are always positive.}
		\end{answered}
	\end{QandA}

\section{Discussion}
\label{min_difficulty_job_scheduler:sec:discussion}
This problem is a classic example of  a problem that can be easily solvable via dynamic programming but can be super challenging if you try to approach it in a different way.
Fortunately, the statement is full of hints about the fact this problem can be solved using DP.
For instance:
\begin{enumerate*}
	\item it is an optimization problem, and,
	\item you are  not really asked to find an actual schedule, but only the value of the best possible one. 
\end{enumerate*}. Very often those are the two most common ingredients in a DP problem.
We have therefore, be able to spot the hints in the statement so that we think about DP.

se DP. Try to cut the array into d non-empty sub-arrays. Try all possible cuts for the array.
Use dp[i][j] where DP states are i the index of the last cut and j the number of remaining cuts. Complexity is O(n * n * d).

\subsection{Brute-force}
\label{min_difficulty_job_scheduler:sec:bruteforce}
If you do not think about DP right away one of the possible ways of approaching this problem would be to try out all possible schedules, and for each of them calculate its cost, 
and return the smallest. 
The problem explicitelly mention a case where a solution does not exists. This is an easy case as there is only one scenario
where you cannot schedule jobs for $d$ days: when the number of jobs to be scheduled is strictly less than $d$.
The core of the problem is really about the case where $|I| \geq d$. 
You can think about a schedule as way of splitting $I$ into $d$ non-empty sub-arrays.
You can split an array in $d$ parts by placing $d-1$ splitting-points in $I$ at different location.
A different placing of the splitting-points leads univocally to a different schedule. 
There is therefore a one-to-one correnspondence between a subset of size $d-1$ of $\{0,1,2, \ldots, |I|-1\}$(the splitting point locations) and schedules. 
We can therefore generate all possible schedules by generating all possible combinations of $d$ elements from $\{0,1,2, \ldots, |I|-1\}$
where each number of a combination  $\{e_0, \ldots, \_{d-1}\}$ represent a splitting point in $I$.

In order to solve this problem we can calculate the costs for each of the schedule represented by a combination of $d-1$ elements of $\{0,1,2, \ldots, |I|-1\}$, and return the cost of the best (the one having minimum cost overall).
The cost of a schedule, as stated in the problem statement, is the sum of the costs for each of the $d$ day where the cost of a single day is the cost of the most expensive job scheduled for that particular day. 
So given a schedule represented by the combination $e = \{e_1, \ldots, \_{d-1}\}$ we can easily calculate its cost, $C(e)$, by using:
\begin{equation*}
	C(e) = \underbrace{\max(A_0, A_1, \ldots, A_{e_1})}_{\text{cost for the } 1^{st} \text{day}} + \underbrace{\max(A_{e_1+1}, A_{e_1+2}, \ldots, A_{e_2})}_{\text{cost for the } 2^{nd} \text{day}} + \ldots + \underbrace{\max(A_{e_{d-1}+1}, A_{e_{d-1}+2}, \ldots, A_{|I|-1})}_{\text{cost for the } d^{th} \text{day}}
\end{equation*}

\subsubsection{Generate all combinations}
The real challenge at this point is really generating the combinations in groups of $d$ elements. 
We can generate all the combinations one at the time by using a backtracking algorithm where we try to construct a combination of element at the time.
A possible recursive implementation of such algorithm is shown in Listing \ref{list:min_difficulty_job_scheduler:combinations}.
The function \inline{foreach_combination} takes as a input an integer $d$ and a function taking a combination
It works by having a number of recursive calls each of them trying to insert an element at a certain location of the variable \inline{combination}.

\begin{minipage}{\linewidth}
	\lstinputlisting[language=c++, caption={Sample Caption},label=list:min_difficulty_job_scheduler:combinations]{sources/min_difficulty_job_scheduler/generate_combination.cpp}
\end{minipage}



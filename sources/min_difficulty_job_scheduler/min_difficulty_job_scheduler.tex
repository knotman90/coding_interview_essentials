%!TEX root = ../main.tex
%%%%%%%%%%%%%%%%%%%%%%%%%%%%%%%%%%
% Links:https://leetcode.com/problems/minimum-difficulty-of-a-job-schedule/
%
% Difficulty: Companies: 
%%%%%%%%%%%%%%%%%%%%%%%%%%%%%%%%%%

\chapter{Minimum difficulty job schedule}
\label{ch:min_difficulty_job_scheduler}
\section*{Introduction}
Imagine you are part of a team currently busy doing beta testing on your new cool feature. The
testing consists of executing several tasks. Each task has dependencies on other tasks and is
assigned a certain amount of complexity points (a measure of how difficult a task is to be
performed; it is not a measure of time). The dependencies between the tasks have been already worked
out i.e. the order in which the tasks are going to be executed is decided. The problem in this
chapter is about coming up with a schedule plan for the execution of these tasks spanning across a
given number of days. Among all possible schedules, we need to make an effort to calculate the
minimum possible complexity achievable for a schedule that will eventually make sure all tasks are
executed and also that there is at least one task executed every day.


\section{Problem statement}
\begin{exercise}
    Write a function that takes as an input a list of tasks $I$ and an integer $d$. The elements in
    $I$ are dependent on each other and to schedule a certain task $I_i$ all the tasks $I_j \: : j <
    i$ have to be completed. The function should return the minimum complexity among all possible
    schedules of length exactly $d$ days. The complexity of a job is calculated as the sum of the
    complexity of every single day of the schedule. The complexity of a day of the schedule is
    defined as the maximum complexity of the task planned for that day.

    As an additional constraint, you have to make sure that there is at least a task scheduled for
    each day.
        
    \begin{example}
        \hfill \\
        Given:
        \begin{itemize}
            \item $I = \{6,5,4,3,2,1\}$
            \item $d = 2$
        \end{itemize}
        the function returns $7$. You can schedule tasks $0$ to $4$ during the first day and the
        last task during the second day. You cannot just schedule all tasks during the first day
        because then you would have a day in the schedule without planned tasks which is not
        allowed.
    \end{example}
    
    \begin{example}
        \hfill \\
        Given:
        \begin{itemize}
            \item $I = \{10,10,10\}$
            \item $d = 4$
        \end{itemize}
        the function returns $-1$. There is no way to schedule tasks for $4$ days when there are
        only $3$ tasks available for scheduling.
    \end{example}

        
    \begin{example}
        \hfill \\
        Given:
        \begin{itemize}
            \item $I = \{7,1,7,1,7,1\}$
            \item $d = 3$
        \end{itemize}
        the function returns $15$. You can schedule the first $4$ tasks the first day for a total
        complexity of $7$. Task at index $4$ and $5$ can be scheduled for days $2$ and $3$
        respectively. 

        Notice that in this case if $d = 2$ then the function would return $8$.
    \end{example}

    \begin{example}
        \hfill \\
        Given:
        \begin{itemize}
            \item $I = \{11,111,22,222,33,333,44,444\}$
            \item $d = 6$
        \end{itemize}
        the function returns $843$. You can schedule tasks $0,1,2,3,4$ in the first $5$ days and the
        rest during the \nth{6}.
        
    \end{example}
\end{exercise}



\section{Clarification Questions}
        
    \begin{QandA}
            \item What should the function return in the case where it is not possible to make a
            valid schedule? For instance when $|I| < d$?
        \begin{answered}
            \textit{You can return $-1$ in that case.}
        \end{answered}

        \item It is guaranteed for the complexity values to be positive ($\geq 0$)?
        \begin{answered}
            \textit{Yes you can assume complexities are always positive.}
        \end{answered}
    \end{QandA}

\section{Discussion}
\label{min_difficulty_job_scheduler:sec:discussion}
This problem is a classic example of a problem that can be easily solvable via dynamic programming
but can be super challenging if you try to approach it differently. Fortunately, the statement is
full of hints about the fact this problem can be solved using DP. For instance:
\begin{enumerate*}
    \item it is an optimization problem, and,
    \item you are not asked to find an actual schedule, but only the value of the best possible one.
\end{enumerate*}. Very often those are the two most common ingredients in a DP problem. We have,
therefore, to be able to spot the hints in the statement so that we think about DP.



\subsection{Brute-force}
\label{min_difficulty_job_scheduler:sec:bruteforce}
If you do not think about DP right away one of the possible ways of approaching this problem would
be to try out all possible schedules, and for each of them calculate its cost, and return the
smallest. The problem explicitly mentions a case where a solution does not exist. This is an easy
case as there is only one scenario where you cannot schedule jobs for $d$ days: when the number of
jobs to be scheduled is strictly less than $d$. The core of the problem is really about the case
where $|I| \geq d$. You can think about a schedule as a way of splitting $I$ into $d$ non-empty
sub-arrays. You can split an array into $d$ parts by placing $d-1$ splitting-points in $I$ at
different locations. A different placing of the splitting-points leads univocally to a different
schedule. There is, therefore, a one-to-one correspondence between a subset of size $d-1$ of
$\{0,1,2, \ldots, |I|-2\}$ (the splitting point locations) and schedules (see Equation
\ref{eq:min_difficulty_job_scheduler:cost_combination}). We can therefore generate all possible
schedules by generating all possible combinations of $d-1$ elements from $\{0,1,2, \ldots, |I|-2\}$
where each number of a combination  $\{e_0, \ldots, e_{d-1}\}$ represent a splitting point in $I$
and $e_i$ identifies the following subarray of $I$: $\{A_{e_i-1+1}, A_{e_i-1+1}, \ldots ,
A_{e_i}\}$.

In order to solve this problem we can calculate the costs for each of the schedule represented by a
combination of $d-1$ elements of $\{0,1,2, \ldots, |I|-1\}$, and return the cost of the best (the
one having minimum cost overall). The cost of a schedule, as stated in the problem statement, is the
sum of the costs for each of the $d$ day where the cost of a single day is the cost of the most
expensive job scheduled for that particular day. So given a schedule represented by the combination
$e = \{e_1, \ldots, \_{d-1}\}$ we can easily calculate its cost, $C(e)$, by using:
\begin{equation}
    C(e) = \underbrace{\max(A_0, A_1, \ldots, A_{e_1})}_{\text{cost for the } 1^{st} \text{day}} + \underbrace{\max(A_{e_1+1}, A_{e_1+2}, \ldots, A_{e_2})}_{\text{cost for the } 2^{nd} \text{day}} + \ldots + \underbrace{\max(A_{e_{d-1}+1}, A_{e_{d-1}+2}, \ldots, A_{|I|-1})}_{\text{cost for the } d^{th} \text{day}}
    \label{eq:min_difficulty_job_scheduler:cost_combination}
\end{equation}

\subsubsection{Generate all combinations}
The real challenge at this point is really about the generation of combinations in groups of $d-1$
elements. We can generate all the combinations one at the time by using a backtracking algorithm
where we try to construct ane combination of elements at a time. A possible recursive implementation
of such algorithm is shown in Listing \ref{list:min_difficulty_job_scheduler:combinations}. The
function \inline{generate_all_combination} takes as a input two integer $k$ and $l$. $k$ represents
the size of the combination and $l$ identifies the elements of the combinations i.e. $0,1,\ldots,
l-1$. If you had to write a generic function for generating combinations you would also most likely
have a parameter containing the list of elements from which to generate the combinations. In this
case, such list is implicit, as we need to generate combinations of splitting points and can be
uniquely identified by a single integer. For instance \inline{generate_all_combination(3,10)}
generates all combinations of three elements from $\{0,1,2,3,4,5,6,7,8,9\}$ and
\inline{generate_all_combination(2,4)} generates all combinations of $2$ elements from
$\{0,1,2,3\}$. \inline{generate_all_combination_helper} is a recursive function which enumerates all
combinations. It takes the following parameters:
\begin{itemize}
    \item \inline{std::vector<std::vector<int>> & all_combinations}: the output list of all
    generated combination,
    \item \inline{std::vector<int>& combination}: the work-in-progress combination,
    \item \inline{const unsigned k}: the size of the combinations
    \item \inline{const unsigned l}: the last number we can add to the work-in-progress combination
    \item \inline{const unsigned curr_el}: the first number we can add to the combination
\end{itemize}
Each call tries to place a number  in the work-in-progress \inline{combination} at location
specified by $curr_el$. Initially $curr_el = 0$ and each recursive call increases it by $1$.
Eventually $curr_el = d$ and we can stop the recursion and return. At that point the combination is
ready and saved into \inline{all_combinations}. After each recursive call, the last inserted element
is removed from the work-in-progress combination and another number is pushed. The process repeats
until there is no more number to be pushed.

\lstinputlisting[language=c++, caption={Function that generates all the combinations of size $k$ from the elements $\{0,1,\ldots,l\}$},label=list:min_difficulty_job_scheduler:combinations]{sources/min_difficulty_job_scheduler/generate_combination.h}


\subsubsection{Wrapping-up}
Once we are able to generate all the possible schedules we are going to evaluate the cost associated
with each of them, and pick the one with the smallest difficulty overall. All is left to do at this
point is to come up with a way to evaluate a given schedule $c$. We have already seen in Equation
\ref{eq:min_difficulty_job_scheduler:cost_combination} how a certain combination of $d-1$ splitting
points maps directly to subarrays of $I$. The function \inline{calculate\_cost\_schedule} in Listing
\ref{list:min_difficulty_job_scheduler:combinations} uses this idea to evaluate a schedule and
calculate its difficulty by summing up the difficulties of each of the tasks scheduled each day.
Notice that \inline{start} and \inline{finish} identify the elements of $I$ in the following range:
$[start, finish]$ (the element pointed by \inline{finish} is included). The function
\inline{min\_difficulty\_scheduler\_combinations} is the driver that is responsible for keeping
track of the minimum difficulty among all the processed schedules.

\lstinputlisting[language=c++, caption={Brute-force solution that works by evaluating all the possible schedules generated using Listing \ref{list:min_difficulty_job_scheduler:combinations}},label=list:min_difficulty_job_scheduler:solution_combinations]{sources/min_difficulty_job_scheduler/min_difficulty_job_scheduler_solution1.cpp}


\subsection{Dynamic Programming}
The key insight to solve this problem with DP is that given that you have decided on a set of tasks
that are scheduled in the first day, say the first $i$ tasks, than the minimum difficulty of a
schedule having the first $i$ elements scheduled the first day across $d$ days is the sum of the
largest task among the first $i$ tasks plus the cost of the best possible schedule of the last
$|I|-i$ tasks among $d-1$ days. More formally, if $C(I,d)$ is a function returning the minimum cost
of a schedule of the tasks in $I$ across $d$ days , and can be defined as follows:
\begin{equation}
    \begin{cases}
        C(\emptyset, 0) = 0 \; :  \text{the cost of scheduling $0$ task in $0$ days is $0$}\\
        C(\emptyset, d > 0) = +\infty \; : \text{it is impossible to schedule $0$ tasks in $1$ or more days}\\
        C(|I|, 0) = +\infty \: :\text{it is impossible to schedule $1$ or more tasks  in $0$ days}\\\\
		C(|I|, d) = \underbrace{\min_{\forall j \in \{0,1,\ldots,|I-1|\}}}_{\text{ $\forall$ schedule of the $d^{th}$ day}} \Bigg( \max I_j + \underbrace{C\Big(I - \{0,1,\ldots,j\}, d-1\Big)}_{\text{optimal solution to a subproblem}}\Bigg)\\
	 \end{cases}
	\label{eq:min_difficulty_job_scheduler:dpformula}
\end{equation}
$C(I,d)$ has a recursive definition and we can quickly see that the problem has both the properties
a DP problem has:
\begin{description}
    \item[optimal substructure:] can be solved by solving and combining together various
    \textbf{optimal} solutions to \textbf{smaller} subproblems.
    \item[overlapping subproblems:] the same problems are solved over and over again. (try to draw
    the recursion tree for $C$ if you are not entirely convinced)
\end{description}

\subsection{Top-down}
Without any optimization, the function that we obtain by translating the recursive definition of
Equation \ref{eq:min_difficulty_job_scheduler:dpformula} is extremely inefficient due to the fact
problems are recalculated over and over (See Appendix \ref{sect:appendix:DP}). In order to make good
use of DP we can therefore use memoization to avoid unnecessary recomputation. Listing
\ref{list:min_difficulty_job_scheduler:solutiondp} shows a possible implementation of this idea
where a \inline{std::unordered_map} is used to remember the calls to \inline{min_difficulty_helper}
(the equivalent of the function $C$). Notice that given $I$ and $d$ the function can "only"  be invoked in $|I|
\times d$ ways. Therefore,in the worst case, by using memoization  we will never
make more than $|I| \times d$ calls to \inline{min_difficulty_helper}. Because the cost of a single call to \inline{min_difficulty_helper} is linear in $|I|$ the complexity of the whole algorithm is $O(|I|^2 d)$

\lstinputlisting[language=c++, caption={Dynamic programming solution that using the idea that },label=list:min_difficulty_job_scheduler:solutiondp]{sources/min_difficulty_job_scheduler/min_difficulty_job_scheduler_solution2.cpp}

\subsection{Bottom-up}
The same  DP approach can also be implemented in a bottom-up fashion. 

\section{Conclusion}
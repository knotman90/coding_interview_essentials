%!TEX root = ../main.tex
%%%%%%%%%%%%%%%%%%%%%%%%%%%%%%%%%%
% Links:
%
% Difficulty:
% Companies: 
%%%%%%%%%%%%%%%%%%%%%%%%%%%%%%%%%%

\chapter{Palindrome Partitioning \rom{2}}
\label{ch:palindrome_partitioning2}
\section*{Introduction}

\section{Problem statement}
\begin{exercise}
Given a string s, partition s such that every substring of the partition is a palindrome. A partition for a string $s$ is a collection of cutpoints $c_0 < c_1 \ldots < c_k$ that split the string $s$ into $k+1$ sustrings:
\begin{itemize}
	\item $s(0 \ldots c_0)$
	\item $s(c_0+1 \ldots c_1)$
	\item \ldots
	\item $s(k-1 \ldots c_k)$
\end{itemize}

Return the minimum cuts needed s.t. the obtained partitions consist only of palindrome substrings.

	\begin{example}
		\hfill \\
		Input: "aab"
		Output: 1
		Explanation: The palindrome partitioning $["aa","b"]$ could be produced using $1$ cut.
	\end{example}

	\begin{example}
		\hfill \\
		Input: "itopinonavevnonipoti"
		Output: 0
		Explanation: The input is already palindrome
	\end{example}

	\begin{example}
		\hfill \\
		Input: "ababbbabbababa"
		Output: 3
		Explanation: The palindrome partitioning $["a","bbabbbab","b","ababa"]$ could be produced using $3$ cut.

	\end{example}
\end{exercise}
\section{Clarification Questions}

\begin{QandA}
	\item 
	\begin{answered}
		\textit{}
	\end{answered}
	
\end{QandA}

\section{Discussion}
\label{palindrome_partitioning2:sec:discussion}


\subsection{Brute-force}
\label{palindrome_partitioning2:sec:bruteforce}

\lstinputlisting[language=c++, caption={Sample Caption},label=list:palindrome_partitioning2]{sources/palindrome_partitioning2/palindrome_partitioning2_solution1.cpp}


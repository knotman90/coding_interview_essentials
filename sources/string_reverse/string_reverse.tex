%!TEX root = ../main.tex
%%%%%%%%%%%%%%%%%%%%%%%%%%%%%%%%%%
% Links:
%
% Difficulty:
% Companies: 
%%%%%%%%%%%%%%%%%%%%%%%%%%%%%%%%%%

\chapter{String Reverse}
\label{ch:string_reverse}
\section*{Introduction}
Reversing a string is an obiquitous operation and because of that is it very often asked during programming interview. There are basically two variation of this problem:
\begin{enumerate}
  \item in-place.
  \item out-of-place.
\end{enumerate}
and the first thing it should be clarified is which type is the interviewer asking.
Moreover considering the popularity of this problem it is important not to make mistakes and solve the problem in a relatively short time frame. The interviewer most likely is expecting you to have seen this question or solved this challenge yourself more than once in the past.


\section{Problem statement}
The problmen statement is very simple and fits in a single line:
\begin{exercise}
Write a method that takes a string $s$ of length $n$ and reverses it.
\end{exercise}

\begin{example}
	\hfill \\
	Given $s="abcde"$ the function produces $s="edcba"$.
\end{example}

\begin{example}
	\hfill \\
	Given $s="programming"$ the function produces $s="gnimmargorp"$.
\end{example}

\section{Clarification Questions}

\begin{QandA}
	\item Should the function reverses the string in place?
	\begin{answered}
		\textit{Yes, a copy of the input cannot be created.}
	\end{answered}

	\item Is the empty string a valid input?
	\begin{answered}
		\textit{Yes, the input string might be empty.}
	\end{answered}
	
\end{QandA}

\section{Discussion}
\label{string_reverse:sec:discussion}
The problem asks to reverse a string in place. But what does it exactly mean to do it \textit{in-place}? It means that no auxiliary storage is allowed and that the input itself will be processed by the function and will be eventually transformed into the output. Since the original content of the input is lost once the function is terminaed, in-place algorithms are also called \textit{destructive}. Note that no additional storage does not literally mean that not even a single variable can be created. In this case it should be interpreted more as that a copy of the input is disallowed, or that the function should work in $O(1)$ (constant) space.

Let's take a deeper look at what happens to the single letters of the string once they are reversed. For instance consider the string $s="a_0 a_1 a_2 a_3 a_4 a_5"$ which is transformed into $s'="a_5 a_4 a_3 a_2 a_1 a_0"$. The subscript $i$ in $a_i$ identifies the position in which the letter $a_i$ appears in the original input. The core of the problem is to figure our how are letters shuffled from the original to the reversed string. In order to do so one can analyze how the indices are moved during the reverse process. It can clearly be seen from the example above that the indices modifies as follows:
\[
\begin{array}{l}
    0 \rightarrow 5  \\ 
    5 \rightarrow 0  \\ 
    1 \rightarrow 4  \\ 
    4 \rightarrow 1  \\ 
    3 \rightarrow 2  \\ 
    2 \rightarrow 3  \\ 
  \end{array}
\]
The example contains all the information that are necessary to deduce the function that maps the indices of the original string into the indices of the reversed string. An index $i$ gets mappes to an index $j$ s.t. $i+j = n$ (index $0$ goes to $5$). A quick manipulation of that equation shows that $j$ (the index of where the letter at index $i$ in the original string will be in the reversed on) is equal to $j = n-i$. 
This equation effectively says that  each letter at index $i$ in the original input is moved to the $j$'s location.
Also notice from the example above that letter  at index $j$ is itself mapped to index $i$ meaning that letters are positions $i$ and $j$ are effectively swapped.

So to summarize, each character in the input string need to be swapped with its symmetrical sibling (See Listing \ref{list:string_reverse_1}).

\lstinputlisting[language=c++, caption=Iterative solution to the in-place string reverse question.,label=list:string_reverse_1]{sources/string_reverse/string_reverse_solution1.cpp}

Note how the loop terminates at only half the size of the input string. This is necessary because a swap operations on the index $i<\frac{n}{2}$ involves also an element of the second half of the string (its simmetrycal sibling).  If the loop would not terminate at $\frac{n}{2}$ then for each element $i$ two swap operations would be performed. For instacce for the letter at index $0$ the following swaps would occur:
\begin{itemize}
	\item[-] swap(0,n-1)
	\item[-] swap(n-1,0)
\end{itemize}.
Considering that two\footnote{Any even number, to be precise} swap operations on the same indices result in having the original status of the letters (on those indeces) restored, if the loop does not stop at $\frac{n}{2}$ then the function would not modify the original string at all. 


This is considered a good solution because the time and space complexity of Listing \ref{list:string_reverse_1} are $O(n)$ and $O(1)$ respectively, which are optimal. Moreover the solution is short and expressive. 

\section{Common Variation}
\label{string_reverse:sec:variations}

\subsection{Out-of-place solution}
Sometimes the interviewe might ask an easier version of this problem which asks to return a reversed copy of the  original string. This version is easier compared to the in-place version because one can simply construct the reversed string by looping the original string backwards as shown in Listing \ref{list:string_reverse_outplace_rawloop} and Listing \ref{list:string_reverse_outplace_iterators}.

\lstinputlisting[language=c++, caption=Iterative out-of-place solution using raw loops. Both time and space complexity for this version is $O(n)$.,label=list:string_reverse_outplace_rawloop]{sources/string_reverse/string_reverse_solution2.cpp}

\lstinputlisting[language=c++, caption=Alternative implementation of an iterative out-of-place solution using iterators. Both time and space complexity for this version is $O(n)$.,label=list:string_reverse_outplace_iterators]{sources/string_reverse/string_reverse_solution3.cpp}

\subsection{Recursive solution}
Another commonly asked variation of the string reverse problem is the one in which recursion has to be used. The problem is well suited for recursion infact, a look at the iterative solution shows that at any given point in the loop the status of the string is the following: $a_{n-1}a_{n-2} \ldots \underbrace{a_k a_{k+1} \ldots a_l}_\text{untouched} a_{k-1}a_{k-2} \ldots a_0$. There is always a portion of the string delimited by two indices $k$ and $l$, $k>=l$ which is yet to be processed (i.e. with the original content). This can be easily used to derive a recursive solution. At first the $k=0$ and $l=n-1$ and the string can be reversed by swapping $a_k=0$ and $a_l=n-1$ and by recursively reversing the inner part of the string i.e. in the range $k=1$ and $l=n-2$. This can be continuosly be done until $l > k$. At that point the function can simply terminate (See Equation \ref{eq:string_reversal_recursion} and Listing \ref{list:string_reversal_recursion}).

\begin{equation}
	R(s, k, l)=\begin{cases} 
\text{swap}(s[k]s[l]) \: \wedge \: R(s,k+1, l-1) & \text{if } k\geq l\\
\text{return} & \text{otherwise}
\end{cases}
\label{eq:string_reversal_recursion}
\end{equation} 

The complexity analysis for this case can be a bit controversial because one have to consider also the stack space utilized by the recursion which can be $O(n)$. It is therefore important to clarify this aspect with the interviewer. Discussing this might open the possibility for questions related to this, especially regarding Tail Call Optimization(TCO)\footnote{TCO (Tail Call Optimization) is the process by which a smart compiler can make a call to a function and take no additional stack space.The allocation of a new stack frame for a function can be avoided because the calling function will simply return the value that it gets from the called function. The most common use is tail-recursion, where a recursive function written to take advantage of tail-call optimization can use constant stack space.\nopagebreak}, so one needs to be careful while opening this pandora box.

\lstinputlisting[language=c++, caption=Recursive in-place solution to the string reversal problem.,label=list:string_reversal_recursion]{sources/string_reverse/string_reverse_solution4.cpp}
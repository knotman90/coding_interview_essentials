%!TEX root = ../main.tex
%%%%%%%%%%%%%%%%%%%%%%%%%%%%%%%%%%
% Links:
%
% Difficulty: Companies: 
%%%%%%%%%%%%%%%%%%%%%%%%%%%%%%%%%%

\chapter{Find the $K$ closest elements}
\label{ch:find_k_closest_in_array}
\section*{Introduction}
In this chapter we are going to discuss a problem that asks you to return a subset of a given input
array. We will investigate two solutions: one that is based on sorting and the other on binary
search with the latter being more efficient than the former because we will make good use of the
fact that the input is going to be provided already sorted. As we will see the solution based on
sorting is going to be almost trivial to and we will be able to derive it directly from the problem
statement while the solution based on binary search requires slightly more brain work and typing to
get it right. We will present two different implementations of the binary search solution: 
\begin{enumerate*}
	\item the first based entirely the C++ STL,
	\item and the other where we will code the binary search algorithm explicitly. 
\end{enumerate*}

\section{Problem statement}
\begin{exercise}

	Write a function that takes as input
	\begin{itemize*}
		\item a sorted array $I$ and two integers
		\item $k$ and
		\item $x$,
	\end{itemize*}
	and returns an array, sorted in ascending order, containing the $k$ elements that are closest to
	$x$ in $I$. Note that: given two elements $y$, and $z$, $y$ is closer to $x$ than $z$ if:
	\begin{equation}
		|x-y| < |x-z|
	\label{eq:kclosest_in_array:sort_criteria}
	\end{equation}
	 

	\begin{example}
		\hfill \\
		Given
		\begin{itemize*}
			\item $I = \{1,2,3,4,5\}$,
			\item $k=4$ and
			\item $x=3$,
		\end{itemize*} 
		the function returns: $\{2,3,4,5\}$
	\end{example}

	\begin{example}
		\hfill \\
		Given
		\begin{itemize*}
			\item $I = \{1,2,3,4,5\}$,
			\item $k=4$ and
			\item $x=-1$,
		\end{itemize*} 
		the function returns: $\{1,2,3,4\}$
	\end{example}

	\begin{example}
		\hfill \\
		Given
		\begin{itemize*}
			\item $I = \{12,16,26,30,35,39,42,46,48,50,53,55,56\}$,
			\item $k=5$ and
			\item $x=36$,
		\end{itemize*} 
		 the function returns: $\{26,30,35,39,42\}$
	\end{example}
\end{exercise}

\section{Clarification Questions}

\begin{QandA}
	\item What should be the function behavior when resolving ties? What do to when you have two
	elements that are at the same distance from $x$?
	\begin{answered}
		\textit{The function should always favor the smaller element in case of a tie.}
	\end{answered}
	
	\item Is I guaranteed to be sorted in ascending order?
	\begin{answered}
		\textit{Yes you can assume $I$ to  always sorted in ascending order.}
	\end{answered}
	
\end{QandA}

\section{Sorting}
\label{sec:find_k_closest_in_array:sorting}
A solution that almost immediately follows from the problem statement is based on the idea of
sorting the elements of $I$ according to the criteria shown in Equation
\ref{eq:kclosest_in_array:sort_criteria}. The idea is that if $I$ is sorted according to the
absolute value of the different between each number of $I$ and $x$ then, the closest number to $x$
will be located, after the sorting at the front of $I$. All is necessary at that point is to copy
the first $K$ element of $I$ into the return array. Listing \ref{list:find_k_closest_in_array1}
shows a possible implementation of such idea. 
\lstinputlisting[language=c++, caption={Solution to the problem of finding the $k$ closest element using sorting.},label=list:find_k_closest_in_array1]{sources/find_k_closest_in_array/find_k_closest_in_array_solution1.cpp}

Please note that as in all cases where you actually do not need to have the whole array sorted you
can use partial sorting instead of full-fledged sorting. In all cases where $k$ is smaller that
$n$ the complexity is going to be slightly better as we will go from the $O(nlog(n))$ of the normal
sorting to $O(nlog(k))$ of the partial sort. Fortunately, making this change in C++ is extremely
easily and it is only matter of calling \inline{std::partial_sort} instead of \inline{std::sort} as
shown in Listing \ref{list:find_k_closest_in_array2}.

\lstinputlisting[language=c++, caption={Solution to the problem of finding the $k$ closest element using sorting.},label=list:find_k_closest_in_array1]{sources/find_k_closest_in_array/find_k_closest_in_array_solution2.cpp}

\subsection{Binary Search}
\label{find_k_closest_in_array:sec:binary_search}
The problem description clearly state the fact that the input array is sorted but, the solution we
devised in Section \ref{sec:find_k_closest_in_array:sorting} is not taking advantage of it at all.
In fact all it does is invalidating the original ordering so to enforce a different one. Everytime
the problem statement mention that some input is sorted, you should think how to use such constraint
to device a more efficient solution rather than questioning whether that information is useful or
not because no useless information is part of the problem statement. Usually when sorted input is
involved, there are a bunch of algorithms that should come to mind immediately. Out of this set, binary
search is probably going to be one of the firsts. But how can binary search be applied to this
problem?

Let's take a step back and try to analyze the problem for a slightly different angle. In particular,
let's discuss the case where $x \in \: I$. In this case we know for sure that $x$ is going to be
part of the output vector. Because the input is sorted we can use binary search to search for $x$ in
$I$. Once we have identified the index $j$ such that $I_j = x$ we know that the closest element to
$x$ must either be at index $j+1$ or $j-1$ \footnote{If that was not true it would mean that:
\begin{itemize}
	\item $I_k = I_{j+1}$. In this case, picking $K$  would not be an improvement to $j-1$ or $j+1$.
	\item otherwise either:
	\begin{itemize}
		\item $\exists k < j-1 \: : I_k < I_{j-1}$ and  $x-I_k < x-I_{j-1}$ or
		\item $\exists k > j+1 \: : I_k > I_{j+1}$ and  $I_k-x < I_{j+1}-x$ \end{itemize} which is
	impossible because the input is sorted. \end{itemize}}. Therefore once $x$ has been identified
	we can select a range of $k$ elements "centered" at $j$. Said range can be found by using a two
	pointers tecnique. We start by initializing two pointers $l = j$ and $r = j$. THen until $r-l+1
	< k$ we do one of the following operations:
\begin{itemize}
	\item if $l = 0$ then $r = r+1$
	\item if $r = |I|$ then $l = l-1$
	\item if$ x-I_{l-1} > I_{r+1}-x$ then $r = r+1$. The range is enlarged at the its right end
	side.
	\item symmetrically for the left side: if $x-I_{l-1} > I_{r+1}-x$ then $ l + l+1$. The range is
	enlarged at its left end side.
\end{itemize}
In other words once $x$ has been found, we incrementally include elements around it, by always
choosing between the closest numbers to $x$ between the numbers pointed by the two pointers.

This approach can be easily extended to the case where $x$ is not present in the input array as it
also work when we try to build the range of elements to be returned around the closest element to
$x$ in the array. 
Turns out that binary search can be used to find such an element (we even have STL
support for such operation). 
In particular we can use it to identify the index of the first element that is larger or equal than $x$:
a value that is commonly known as \textit{lower bound}.
Armed with this information let's have a look at Listing \ref{list:find_k_closest_in_array2} showing
the implementation of the idea above where we use the STL \inline{std::lower\_bound} function to find
the index $o$ of the first element greater or equal than $x$.
We then compare such value with the value at index $p = o-1$ (if exists)
and we promote $o$ or $p$ to be the index to closest element to $x$ in the array depending on their
absolute difference to $x$.
The closest of the two to $x$ is chosen to be the designated starting value for the algorithm described above.

\lstinputlisting[language=c++, caption={Solution to the problem of finding the $k$ closest element using \inline{std::lower\_bound}.},label=list:find_k_closest_in_array2]{sources/find_k_closest_in_array/find_k_closest_in_array_solution3.cpp}

For completeness, in Listing \ref{list:find_k_closest_in_array:binary_lower_bound} we also show an implementation of "in-house" version of \inline{std::lower\_bound}. You might be asked 
to show you can code binary search. 

\begin{minipage}{\linewidth}
	\lstinputlisting[language=c++, caption={Implementation of a function for the calculation of the \textit{lower\_bound} that can be using in substitution of \inline{std::lower\_bound} in Listing \ref{list:find_k_closest_in_array2}.},label=list:find_k_closest_in_array:binary_lower_bound]{sources/find_k_closest_in_array/my_lower_bound.h}
\end{minipage}
\chapter{Preparation Strategy}

\section{Debug in your mind}
Lesson 1: Destroy that damn RUN CODE button in leetcode. I have prepared the wrong way and probably
all of you too. I would take a problem, think for a bit how to solve, then start writing code,
incrementally complete step by step, run every once in a while to see if the desired output are
coming, then test a few times with few cases, and finally hit SUBMIT. This left me feel so
handicapped in the real interview cuz you cannot RUN your code. You have write once that will JUST
WORK and test with your EYES. This threw off my rhythm because of how I prepared.

HOW YOU PREPARE > HOW MUCH YOU PREPARE
\section{Ready for interruptions}
Lesson 2: Prepare to face interruptions and distractions from the interviewer At the start, the
interviewer mentioned he will be constantly typing as we do the interview and it's normal. But it
definitely threw off my rhythm because when you are at work, or when prearing on leetcode you are in
this silent, zero-distraction mode as you focus on your coding. Now someone asks you random
questions in between, Why is line 21 doing this, or I don't see you pushing to the stack anywhere in
this for loop, and then proceeds to start typing on his computer. He may be trying to help you, or
didn't wait long enough for you to finish up your work properly. But you are now left thinking what
he may be typing instead of completing or fixing your code. This brings us to my next lesson.

\section{Practice common patterns like a Ninja}
Lesson 3: Practice to code common patterns like a Ninja There are certain coding patterns that are
frequently used when solving leet code problems. This can be a binary search, or sort, or traversing
a tree (my case). You want to be ruthlessly clinical about finshing this part of the code quickly. I
had some hiccups performing a in-order-traversal because of lack of practice. You should be able to
complete this part of your code your EYES closed, which only comes with serious prep. This bring us
to the next lesson - Prep Time.

\section{Take your time to prepare}
Lesson 4: Don't be shy to ask the recruiter more time The recuriter offered around two weeks to
prepare, and I borrowed additional two weeks since I was super rusty. Though I prepared for a month,
I fell short. One of our recent interview experiences posted here quoted this, which was so true.

During tough times, You don't rise to the level of expecations, you fall back to your level of
training If you haven't prepared very strongly, then during the interview you cannot magically go
one level up. It's quite the opposite, you will be slightly below your average. Be sure to ask the
recruiter even 3 months time if that's what its going to take to be ready.

\section{Mock interviews are helpful}
Mock interviews were extremely beneficial for building up speed, talking through solutions, and overall confidence.
I remember being more nervous for my first mock session than for my actual first interview. Practice makes perfect!
Use websites like \href{https://www.pramp.com}{PRAMP}\footnote{\url{https://www.pramp.com}}.


\section{Do not code and forget - Spaced repetition}
Flashcards helped me recall problems I had solved months ago. Don't just power through questions - make sure you're retaining knowledge as well. For those interested, 
I used Anki and created flashcards out of each question I solved. Spaced repetition does wonders

On the front would be the question 
(without the title so that I don't memorize based on names),
 and on the back I would write a short explanation of the solution and paste the full code. 
 I'd also have time and space complexity.
 Sometimes there were multiple solutions (maybe with trade-offs between time and space complexity), so I'd include both in the back.

\section{Get referrals}
If possible, try to get referrals for positions you're interested in as they make a HUGE difference with getting your foot in the door

\section{Gather positivity}
Smile and be as positive as possible! Interviewers are not only assessing your technical skills but also your soft/interpersonal skills.
At the end of the day they want to bring on someone they will enjoy working with, so make sure to keep this in mind

\section{Use online courses}
Master the Coding Interview: Data Structures + Algorithms on UDEMY

\section{Do not underestimate Low level design Preparation}
\url{https://cloud.google.com/apis/design/resources#top_of_page}.
% @Author: Davide Spataro
% @Date:   2020-10-25 
% @Last Modified by:   Davide Spataro
% https://www.topcoder.com/community/competitive-programming/tutorials/dynamic-programming-from-novice-to-advanced/
% file:///home/knotman/Downloads/DYNAMIC_PROGRAMMING_-_ITS_PRINCIPLES_APPLICATIONS_.pdf
% http://smo.sogang.ac.kr/doc/bellman.pdf 
\section*{Dynamic Programming}
\label{sect:appendix:DP}

Dynamic programming (DP) is a popular technique for solving a certain class of
optimization problems efficiently and is accredited to the American Scientist
Richard Bellman\cite{bellman1954}. He conied the term DP in the context of
solving problems involving a serie of best decision one after the other. 
The word \textit{programming} can be a bit deceiving for
computer scientist of programmers in general but it has really little to do with
computer programming and it is infact intended as a set of rules to 
follow to solve a certain problem and it is refeered specifically to the
solution to find an optimal military schedule for logistics (and has more or
less the same meaning as linear programming or linear optimization).  These rules can of course be coded and
executed by a computer but can be easily followed on paper for instance. 
Dynamic programming is better thought as an optimization approach rather than an
method or framework where a complex optimization problem is transformed into a sequence of
smaller (and simpler) problems. The very essence of DP is its multi-stage
optimization procedure. DP does not provide directly with the
instruction on how to solve a particular problem, but instead provides a general
framework that requires creativity and non trivial effort/insights so that a
problem formulation can be adapted and casted within the DP framework bounds.
This is possibly the reason why DP is considered a rather hard topic and it is
particularly feared during interviews. 

This chapter is not intended to be a full treatement of DP, and we will
introduce and describe it to the level that is necessary to understand and
better tackle DP interview problems. For a more comprenshive material on DP
please refer to \cite{bellman1954, cormen2009}.

The gist of the DP approach is that we aim at breaking down a problem into
simpler sub-problems recursively. If it is possible to do so, then the problem
at hand is said to have the \textbf{optimal substructure} property i.e. it can
be solved by using optimal solution to subproblems. But having the optimal
substructure property alone is not enough to prefer a DP approach to another
when trying to solve the same problem. This is because DP really shines when a
problem also exposes the \textbf{overlapping subproblems} property i.e. when the
subproblems are reused several times. A classic example if the
Fibonacci Sequence. In order to calculate $F(n)$ we need to solve two subproblems:
$F(n-1)$ and $F(n-2)$ and adding them up. But for solving $F(n-1)$ we need to
solve $F(n-2)$ \textbf{again}. The value for the subproblem $F(n-2)$ is thus
reused and this makes the Fibonacci problem exposed the optimal substructure
property. 
Dynamic programming takes care of this fact by making sure of solving each
subproblem only once. Usually this can be achieved into two ways:
\begin{description}
    \item [Top-down] This is usually the easiest of the two, by being a direct
    derivation from the recursive formulation of the problem. If the problem can
    be formulated recursively in terms of solution then solution to subproblems
    can be \textit{memoized}\footnote{From the latin word \textit{memorandum}
    which means to be remembered. It is basically a way of remembering the
    result of a function for a certain set of inputs call by storing it in a
    cache.} in a cache. 
    When a subproblem is reused then the
    (potentially expensive) recursive call is avoided and the cached result is
    returned instead. 
    \item [Bottom-up] We can try to reformulate the problem by twisting and
    massaging  the  recursive formulation so that the subproblems are solved
    first (thus effectively removing the recursion) and build the solution to
    the bigger problem from the bottom. This is usually done by working in a
    sort of tabular form where entries of the table for larger problems are
    filled by using  entries for solution to smaller problems that we have
    already solved. For instance, when solving the problem of finding the
    $10^{th}$ Fibonacci number $F(10)$, we can start from the known values for
    $F(0)$ and $F(1)$ and working our way up to $F(2)$  by using $F(1)$ and
    $F(2)$. Once F(2) is ready we can move up to F(3), and so on when we have
    the values for $F(8)$ and $F(9)$ we proceed with calculating $F(10)$.
\end{description}

DP has found application in many field of science such as Control theory,
Bioinformatics AI and operations research. There are a number of problems in
computer science that can be solved by using DP such as the 
\begin{itemize}
    \item Longest Common (or increasing) Subsequence
    \item Weighted Interval Scheduling
    \item Chain Matrix Multiplication
    \item Subset sub
    \item String edit distance
    \item Coin change
    \item 0/1 knapsack problem
    \item Graph shortest path
\end{itemize}

In the next section we will shortly review a number of DP problem focusing on
the key ideas that allow a problem to be approached and solved  using DP.

\subsection*{Fibonacci Sequence}
Computing the $n^{th}$ number of the Fibonacci sequence is probably one of the
most common introductionary example of DP. The Fibonacci sequence recursive
formulation is ready to be solved using a top-down DP approach. Listing
\ref{list:app:dp:canonical} shows a \CC function that calculated the $n^{th}$ Fibonacci
number.
\lstinputlisting[language=c++, caption={Canonical recursive \CC implementation of a function returning the $n^{th}$ Fibonacci number.},label=list:app:dp:canonical]{sources/appendices/fibonacci_canonical.cpp}
Notice that for instance when $F(6)$ a call tree is produced where the same call
is repeated more than once as shown in the list below. $F(2)$ has been
calculated $5$ times!
\begin{itemize}
    \item $F(6) = F(5)+F(4)$
    \item $F(6) = (F(4)+F(3)) + (F(3)+F(2))$
    \item $F(6) = ((F(3)+F(2))+(F(2)+F(1))) + ((F(2)+F(1))+(F(1)+F(0)))$
    \item $F(6) = (((F(2)+F(1))+(F(1)+F(0)))+((F(1)+F(0))+F(1))) + (((F(1)+F(0))+F(1))+(F(1)+F(0)))$
    \item $F(6) = ((((F(1)+F(0))+F(1))+(F(1)+F(0)))+((F(1)+F(0))+F(1))) + (((F(1)+F(0))+F(1))+(F(1)+F(0)))$
\end{itemize}

Listing \ref{list:app:dp:fib} can be improved dramatically if we memoize the function calls
that have been already calculated. This way no duplicate work is done. W.r.t the
previous example, from the second time the value of $F(2)$ is needed, no
additional work is done, as the value in the cache is returned.
\lstinputlisting[language=c++, caption={Canonical recursive top-down Dynamic Programming \CC implementation of a function returning the $n^{th}$ Fibonacci number.},label=list:app:dp:fib]{sources/appendices/fibonacci_dp_top_down.cpp}

% @Author: Davide Spataro
% @Date:   2020-10-25 
% @Last Modified by:   Davide Spataro

\section{Dynamic Programming}
\label{sect:appendix:DP}

Dynamic programming (DP) is a popular tecnique for solving a certain class of
optimization problems efficiently and is accredited to the American Scientist
Richard Bellman\cite{bellman1954}. The word \textit{programming} can be a bit deceiving for
computer scientist of programmers in general but it has really little to do with
computer programming and it is infact intended as a set of rules to 
follow to solve a certain problem. These rules can of course be coded and
executed by a computer but can be easily followed on paper for instance. 
Dynamic programming is better thought as an optimization approach rather than an
method or framework where a complex optimization problem is transformed into a sequence of
smaller (and simpler) problems. The very essence of DP is its multi-stage
optimization procedure. DP does not provide directly with the
instruction on how to solve a particular problem, but instead provides a general
framework that requires creativity and non trivial effort/insights so that a
problem formulation can be adapted and casted within the DP framework bounds.
This is possibly the reason why DP is considered a rather hard topic and it is
particularly feared during interviews. 

This chapter is not intended to be a full treatement of DP, and we will
introduce and describe it to the level that is necessary to understand and
better tackle DP interview problems. For a more comprenshive material on DP
please refer to \cite{bellman1954, cormen2009}.


%!TEX root = ../main.tex
%%%%%%%%%%%%%%%%%%%%%%%%%%%%%%%%%%
% Links:
%
% Difficulty: Companies: 
%%%%%%%%%%%%%%%%%%%%%%%%%%%%%%%%%%

\chapter{Jump Game}
\label{ch:can_jump}
\section*{Introduction}
In this problem we are going to investigate whether a solution exists for a game played in an array
where the player starts at the beginning of it and has to get to the last location by jumping from a
cell to another. The array contains information about the length of the jump you can take from a
cell. 

There are a number of possible solutions to this problem and in this Chapter we will discuss two of
them. In particular:
\begin{itemize}
	\item In Section \ref{can_jump:sec:backtracking} we take a look at an approach that is possibly
	 the most intuitive one, where we try all possible jumps in a backtracking-like manner.
	\item In Section \ref{} we will refine the solution of Section \ref{} into an efficient one that
	uses a clever insight to visit the cell efficiently. 
	\item In Sections \ref{} and \ref{} we will have a look at a dynamic programming-based approach
	and we will discuss both the top-down and bottom-up implementations.
	\item Finally, in Section \ref{} we will discuss an efficient and concise greedy solution.
\end{itemize}


\section{Problem statement}
\begin{exercise}
Write a function that takes as input  an array $I$ of non-negative integers. You are initially
positioned at the beginning of the array (at index $0$) and your goal is to jump from cell to cell
to the end of the array (cell $|I|-1$). If you are in the at index $j$ you are allowed to jump to
all cells within the following range: $[j-I_i,j+I_i]$ (each cell of the array contains the
information about the longest jump you can take from there). The function should return true if you
are able to reach the last cell of the array, false otherwise.

	\begin{example}
		\hfill \\
		Given  $I=[2,3,1,1,4]$ the function retuns \textbf{true}. You jump from cell $0$ to $1$ and
		then take a $3$ cells wide jump to the end of the array. 
	\end{example}

	\begin{example}
		\hfill \\
		Given $I=[3,2,1,0,4]$ the function retuns \textbf{false} because it is impossible to reach
		any cells with index higher than $3$.
		
	\end{example}
\end{exercise}

%\section{Clarification Questions}

%\begin{QandA}
	%\item \begin{answered}
		%\textit{}
	%\end{answered}
	
%\end{QandA}

\section{Backtracking}
\label{can_jump:sec:backtracking}
The first solution that we will investigate is one based on an idea similar to the DFS where we $I$
is treated as an implicit graph where each cell (a node) is connected to all the others cells can be
reached by jumping from it. The idea is to use DFS to check whether there the last node of the graph
is connected with the first one. In other words, whether there is a path from the first to the last
node. We can proceed by adopting a recursive approach where we try to visit all the nodes that we
can reach from the node we currently occupy and to continue this process until either we have
reached the last node or there is no more jump we can try, meaning in that case, that there is no
way to reach the last node (the last node is disconnected). Because the implicit graph is not
guaranteed to be acyclic, in order to make this approach work we need to make sure is that we not
jump back and forth from a cell to another in a cycle. This can happen if for instance you jump from
a cell $0$ to cell $1$ and then back to cell $0$. In order to overcome this issue we can only
perform forward jumps so that it will be impossible to be stuck in a cycle and still manage to find
answer. When you jump to a cell $i$ from a cell $j$ s.t. $j < i$ (you performed a forward jump) you
know that you can also visit all cells $ j \leq k \leq i$ (all the cells in between $j$ and $i$). If
you only jump forward you are not going to need to visit any cell $ j \leq k \leq i$ using backward
jumps as they are visited anyway when processing cells $j$ by performing forward jumps from it. An
implementation of this idea is shown in Listing \ref{list:can_jump1_1}. This approach is correct and
it will eventually finds a solution, 	but it is extremely inefficient. Its complexity is exponential
in time as potentially the same cells are visited over and over\footnote{Suppose $W(x)$ is the
number of possible ways you can jump from position $x$ to the end of the array at index $N$. We know
that $T(N) = 1$ (the only way to jump from cell $N$ to itself is not to jump at all). For all other
cells we have that:
	\begin{align*}
		W(x) = \sum_{i=x+1}^N W(i) \\
		 = W(x+1) + \sum_{i=x+2}^N W(i) \\
		 = W(x+1) + W(x+1) \\
	  \end{align*}
	So in order to calculate $W(X)$ we need the values  $W(x+1)$ two times. The recursive tree for
	$W$ is binary and complete and has height $N$ and therefore contains $O(2^N)$ number of nodes.}
	and constant in space\footnote{if we do not consider the spaces utilized by the stack frames
	during the recursive calls, otherwise it is linear.}.

	\lstinputlisting[language=c++, caption={Exponential time solution to the \textit{jump game} problem where only forward jumps are performed.},label=list:can_jump1_1]{sources/can_jump/can_jump_solution1_1.cpp}
 
\section{DFS}
\label{can_jump:sec:DP1}
Another option for solving the cycle problem arising from the algorithm described in Section
\ref{can_jump:sec:backtracking} (this solution can be infact thought as an optimized backtracking)
is to keep track of the cells that we have already visited and everytime we are about to perform a
jump to a cell we first check whether the cell has been visited already in the past and if it had,
the jump is discarded and not performed. No cells is actually visited twice this way and as a
consequence the complexity is in this case $O(|I|^2)$. Listing \ref{list:can_jump1} shows an
implementation of this idea. 


\lstinputlisting[language=c++, caption={Quadratic time and linear space DFS solution to the \textit{jump game} problem using a visited array.},label=list:can_jump1]{sources/can_jump/can_jump_solution1.cpp}

Notice that one optimization from which this solution (and perhaps also Listing \ref{list:can_jump1_1}) can benefit would be to always try to
jump the longest distance possible. Despite this would not change their asymptotic complexity, in
practice this might be faster in certain cases.



\subsection{Greedy}
\label{can_jump:sec:DP1}
There is however a much faster solution to this problem that is based on the idea that we can return true if we 
can jump from the cell at index $0$ to a cell from which we can reach the end of the array.
If we apply the same reasoning to any index $i$ we end up with basically a Dynamic programming approach that, 
given $G(x)$ is $1$ if you can reach the end of the array from the cell $x$ and $0$ otherwise,
is based on the following recursive formula:
\begin{equation}
	\begin{cases}
		G(|I|-1) = 1 \\
		G(x) = 1 \: \: \text{if} \: \: \exists \: y > x \:\: \text{s.t.} \:\: y < (x+I_x) \: \: \text{and} \: \:G(y) = 1\\
		\text{otherwise} \: \: G(x) = 0
	 \end{cases}
	\label{eq:can_jump:dpformula}
\end{equation}
Equation \ref{eq:can_jump:dpformula} shows that a possible implementation would start processing cells from the last to the first
and that for each element a linear time lookup for a suitable cell $y$ might be needed. Therefore the complexity of this solution is quadratic in time.
However we can drastically lower its complexity by noticing that when processing cell $x$ all we care about is whether the closest cell to the right from which you can reach the end of the array
is reachable from $x$. 
We can carry this information into a variable $m$ down from cell $|I|-1$ to cell $0$ and update it after a cell is processed and this would effectively allow us to have a linear time solution.

To summarize the linear time solution for this problem works as follows:
We iterate the array $I$ right-to-left and for each cell $x$ we check whether we can reach $m$ jumping from $x$. If we can then $x$ is the new leftmost cell from which we can reach the end of the array, thus $m = x$.
Otherwise we continue by processing cell $x-1$ in a similar manner. Eventually we will have processed all cells and therefore we can return true if $m = 0$ (meaning cell $0$ is the leftmost cell from which we can jump to location $|I|-1$), and false otherwise.
\lstinputlisting[language=c++, caption={Linear time DFS solution on the implicit graph having indexes as vertices and jump lenghts defining the arcs between them. },label=list:can_jump]{sources/can_jump/can_jump_solution2.cpp}



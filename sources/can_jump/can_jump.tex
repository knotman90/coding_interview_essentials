%!TEX root = ../main.tex
%%%%%%%%%%%%%%%%%%%%%%%%%%%%%%%%%%
% Links:
%
% Difficulty:
% Companies: 
%%%%%%%%%%%%%%%%%%%%%%%%%%%%%%%%%%

\chapter{Jump Game}
\label{ch:can_jump}
\section*{Introduction}

\section{Problem statement}
\begin{exercise}
Given an array of non-negative integers, you are initially positioned at the first index of the array.

Each element in the array represents your maximum jump length at that position.

Determine if you are able to reach the last index.


	\begin{example}
		\hfill \\
		Given the array $A=[2,3,1,1,4]$ the function retuns \textbf{true}
	\end{example}

	\begin{example}
		\hfill \\
		Given the array $A=[3,2,1,0,4]$ the function retuns \textbf{false} because no matter to reach any cells with index higher than $3$ and from $3$ you cannot jump anywhere else.
		
	\end{example}
\end{exercise}

\section{Clarification Questions}

\begin{QandA}
	\item 
	\begin{answered}
		\textit{}
	\end{answered}
	
\end{QandA}

\section{Discussion}
\label{can_jump:sec:discussion}


\subsection{Brute-force - Implicit graph}
\label{can_jump:sec:bruteforce}

\lstinputlisting[language=c++, caption={Quadratic time and linear space DFS solution on the implicit graph having indexes as vertices and jump lenghts defining the arcs between them. },label=list:can_jump]{sources/can_jump/can_jump_solution1.cpp}

\subsection{Linear time solution}
\label{can_jump:sec:linear}

\lstinputlisting[language=c++, caption={Linear time solution to the problem of the jump game.},label=list:can_jump]{sources/can_jump/can_jump_solution2.cpp}



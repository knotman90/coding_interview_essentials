%!TEX root = ../main.tex
%%%%%%%%%%%%%%%%%%%%%%%%%%%%%%%%%%
% Links:
%
% Difficulty: Companies: 
%%%%%%%%%%%%%%%%%%%%%%%%%%%%%%%%%%

\chapter{Counts the items in the containers}
\label{ch:items_in_containers_amazon}
\section*{Introduction}
Imagine you are the owner of a successful online store. You would like to be able to query what is
the number of items you still have in the warehouse. The problem is that you cannot just walk into
the warehouse and count the items as they are stored in closed containers. Thankfully, the warehouse
is equipped with sensors and it is able to produce a string representing the state of the warehouse
and single containers. The problem described in this chapter investigates how we can write an
algorithm that takes such string (the state of all the containers in the warehouse) and is able to
answer queries on the number of elements that are present in some portions of the warehouse itself. 

This problem has been reported to be asked during Amazon interviews and it is considered a medium
difficulty problem. We will investigate two solutions:
\begin{itemize}
	\item brute-force, that is going to have a relatively straightforward logic (blindly count the
	 items in the string) and be easy to code (in Section
	 \ref{items_in_containers_amazon:sec:bruteforce}),
	\item a more sophisticated one with optimal asymptotic complexity where the input string is
	preprocessed so that queries can be answered faster.
\end{itemize}



\section{Problem statement}
\begin{exercise}
	You are given a string $s$ representing the current state of a warehouse. The string contains
	only two kinds of characters: 
	\begin{description}
		\item[\bsq{\textbf{\texttt{*}}}(ASCII 42)]: representing an item 
		\item[\bsq{\textbf{\texttt{|}}}(ASCII 124)]: representing the boundaries of a container.
	\end{description}
	
	A container is a closed space within the warehouse and it is represented in $s$ by a pair of
	\bsq{\texttt{|}}. Items within a container $c$ are represented as \bsq{\texttt{*}} appearing
	withing the two \bsq{\texttt{|}} defining $c$. You are also given an array of pairs $Q = \{(s_0,
	e_0),(s_1, e_2),\ldots,(s_{n-1}, e_{e-1}) : 0 \leq s_i \leq e_i \leq |s|\}$, where
	each pair in $Q$ identifies a substring in $s$. Each element of $Q$ is a query you must answer to.
	
	Your task is to write a function that returns an array $A$ of length $n$, containing the answers
	to all of the queries in $Q$, where each element $A_i$ is the number of items contained in all
	the \textbf{closed} compartments between $(s_i, e_i)$.	

\begin{example}
	\hfill \\
	Given \texttt{s = \bsq{|**|*|*}} and $Q = \{(0,4),(0,5)\}$ the function returns $A=\{2,3\}$. $s$
	has a total of $2$ closed containers the first with $2$ and $1$ item inside respectively.
	
	The first query asks you to find the number of elements in the substring $s[0,4]=$
	\texttt{\bsq{|**|*}} where three items are represented but only two are within a closed
	container (the first two).
	
	The second query refers to the substring $s[0,5]=$ \texttt{\bsq{|**|*|}}. The items are the same
	as in the previous query but this time all of them are in closed containers.
	
\end{example}

\begin{example}
	\hfill \\
	Given \texttt{s = \bsq{*|*|}} and $Q = \{(0,2),(1,3)\}$ the function returns $A=\{0,1\}$. $s$
	has a total of two items and only $1$ closed container containing only a single item.

	The first query refers to the substring $s[0,2]=$ \texttt{\bsq{*|*}}. No closed container are
	represented in such substring thus the answer in this case must be $0$. However, the second
	question refers to  $s[1,3]=$ \texttt{\bsq{|*|}} where we can see we have a valid container. We
	can therefore counts the elements in it.
\end{example}

\end{exercise}
\section{Clarification Questions}

\begin{QandA}
	\item Is it guaranteed for the input string $s$ to only contains valid character?
	\begin{answered}
		\textit{Yes, you do not need to worry about the sanity of the input.}
	\end{answered}
\end{QandA}

\section{Discussion}
\label{items_in_containers_amazon:sec:discussion}



\subsection{Brute-force}
\label{items_in_containers_amazon:sec:bruteforce}
This problem has a straightforward solution that basically loops over all the elements  specified in a query $(s,e) \in Q$
and counts all the elements inside the containers. Because the $|s-e|$ is $O(|s|)$ the complexity of this approach is $O(|s|*|Q|)$. 
Listing \ref{} shows an implementation of such idea.

\lstinputlisting[language=c++, caption={Na\"ive solution to the \textit{items in the container} problem.},label=list:items_in_containers_amazon]{sources/items_in_containers_amazon/items_in_containers_amazon_solution1.cpp}


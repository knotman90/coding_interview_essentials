%!TEX root = ../main.tex
%%%%%%%%%%%%%%%%%%%%%%%%%%%%%%%%%%
% Links:
%
% Difficulty: Medium
% Companies: Facebook paypal yahoo microsoft linkedin mazaon 
%%%%%%%%%%%%%%%%%%%%%%%%%%%%%%%%%%

\chapter{Max sum contiguous subarray}
\label{ch:max_sum_continguous_subarray}
\section*{Introduction}

\section{Problem statement}
\begin{exercise}
Find the largest sum of a contiguous subarray (containing at least one element) within an array, $A$ of length $n$, and return it.
Formally, the task is to find two indices $ 0 \leq i \leq j < n$ s.t. the following sum is as large as possible:
\[
\sum_{x=i}^j A[i]
\]

\end{exercise}


\begin{example}
	\hfill \\
	Given $A=\{-2, -5, \underbrace{6, -2, -3, 1, 5}\text{}, -6\}$ then the answer is $7$ which can be obtained by summing all elements from index $2$ to $6$ i.e. $\sum_{2}^7 A[i] = 7$
\end{example}

\begin{example}
	\hfill \\
	Given $A=\{-2, 1, -3, \underbrace{4, -1, 2, 1}\text{}, -5, 4\}$ then the answer is $6$ which can be obtained by summing all elements from index $3$ to $6$ i.e. $\sum_{3}^6 A[i] = 6$
	
\end{example}

\section{Clarification Questions}

\begin{QandA}
	\item Are the elements all positive or negative?
	\begin{answered}
		\textit{No, the input numbers can be either positive, or negative}
	\end{answered}
	
	\item Is the array sorted?
	\begin{answered}
		\textit{No, the array is not sorted.}
	\end{answered}
	
\end{QandA}

\section{Discussion}
\label{max_sum_continguous_subarray:sec:discussion}
First of all let's start by notiving that 
\begin{itemize}
	\item[-] If the array only contains non-negative numbers then the problem is trivial because the answer is the sum of the whole array. 
	\item[-] if, on the contrary, the array contains only non-positive numbers, then the answer is the sum of a one element subarray which coincides with the largest number.
	\item[-] The answer is unique, but more than one subarray might have sum up to that value. 
\end{itemize}

\subsection{Brute-force}
\label{max_sum_continguous_subarray:sec:bruteforce}
One way to tackle this problem is to look at the sum of all possible subarray and return the largest sum. 
The idea is that for all elements $A[i]$, the sum of all subarrays having it as starting element can be easily calculated  as shown in Listing \ref{list:max_sum_continguous_subarray_bruteforce}. This approach works but it is unnecessarly slow, havig a time complexity of $O(n^3)$. There are $O(n^2)$ ordered pairs each identifying a subarray, and calculating the sum of a single subarray costs $O(O(n)$ for a grand total of $O(n^3)$. 

\lstinputlisting[language=c++, caption=Sample Caption,label=list:max_sum_continguous_subarray_bruteforce]{sources/max_sum_continguous_subarray/max_sum_continguous_subarray_solution1.cpp}

